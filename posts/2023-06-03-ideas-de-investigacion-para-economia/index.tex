\documentclass[
  man,
  floatsintext,
  longtable,
  a4paper,
  nolmodern,
  notxfonts,
  notimes,
  12pt,
  colorlinks=true,linkcolor=blue,citecolor=blue,urlcolor=blue]{apa7}

\usepackage{amsmath}
\usepackage{amssymb}



\usepackage[bidi=default]{babel}
\babelprovide[main,import]{spanish}


% get rid of language-specific shorthands (see #6817):
\let\LanguageShortHands\languageshorthands
\def\languageshorthands#1{}

\RequirePackage{longtable}
\RequirePackage{threeparttablex}

\makeatletter
\renewcommand{\paragraph}{\@startsection{paragraph}{4}{\parindent}%
	{0\baselineskip \@plus 0.2ex \@minus 0.2ex}%
	{-.5em}%
	{\normalfont\normalsize\bfseries\typesectitle}}

\renewcommand{\subparagraph}[1]{\@startsection{subparagraph}{5}{0.5em}%
	{0\baselineskip \@plus 0.2ex \@minus 0.2ex}%
	{-\z@\relax}%
	{\normalfont\normalsize\bfseries\itshape\hspace{\parindent}{#1}\textit{\addperi}}{\relax}}
\makeatother




\usepackage{longtable, booktabs, multirow, multicol, colortbl, hhline, caption, array, float, xpatch}
\usepackage{subcaption}


\renewcommand\thesubfigure{\Alph{subfigure}}
\setcounter{topnumber}{2}
\setcounter{bottomnumber}{2}
\setcounter{totalnumber}{4}
\renewcommand{\topfraction}{0.85}
\renewcommand{\bottomfraction}{0.85}
\renewcommand{\textfraction}{0.15}
\renewcommand{\floatpagefraction}{0.7}

\usepackage{tcolorbox}
\tcbuselibrary{listings,theorems, breakable, skins}
\usepackage{fontawesome5}

\definecolor{quarto-callout-color}{HTML}{909090}
\definecolor{quarto-callout-note-color}{HTML}{0758E5}
\definecolor{quarto-callout-important-color}{HTML}{CC1914}
\definecolor{quarto-callout-warning-color}{HTML}{EB9113}
\definecolor{quarto-callout-tip-color}{HTML}{00A047}
\definecolor{quarto-callout-caution-color}{HTML}{FC5300}
\definecolor{quarto-callout-color-frame}{HTML}{ACACAC}
\definecolor{quarto-callout-note-color-frame}{HTML}{4582EC}
\definecolor{quarto-callout-important-color-frame}{HTML}{D9534F}
\definecolor{quarto-callout-warning-color-frame}{HTML}{F0AD4E}
\definecolor{quarto-callout-tip-color-frame}{HTML}{02B875}
\definecolor{quarto-callout-caution-color-frame}{HTML}{FD7E14}

%\newlength\Oldarrayrulewidth
%\newlength\Oldtabcolsep


\usepackage{hyperref}




\providecommand{\tightlist}{%
  \setlength{\itemsep}{0pt}\setlength{\parskip}{0pt}}
\usepackage{longtable,booktabs,array}
\usepackage{calc} % for calculating minipage widths
% Correct order of tables after \paragraph or \subparagraph
\usepackage{etoolbox}
\makeatletter
\patchcmd\longtable{\par}{\if@noskipsec\mbox{}\fi\par}{}{}
\makeatother
% Allow footnotes in longtable head/foot
\IfFileExists{footnotehyper.sty}{\usepackage{footnotehyper}}{\usepackage{footnote}}
\makesavenoteenv{longtable}

\usepackage{graphicx}
\makeatletter
\newsavebox\pandoc@box
\newcommand*\pandocbounded[1]{% scales image to fit in text height/width
  \sbox\pandoc@box{#1}%
  \Gscale@div\@tempa{\textheight}{\dimexpr\ht\pandoc@box+\dp\pandoc@box\relax}%
  \Gscale@div\@tempb{\linewidth}{\wd\pandoc@box}%
  \ifdim\@tempb\p@<\@tempa\p@\let\@tempa\@tempb\fi% select the smaller of both
  \ifdim\@tempa\p@<\p@\scalebox{\@tempa}{\usebox\pandoc@box}%
  \else\usebox{\pandoc@box}%
  \fi%
}
% Set default figure placement to htbp
\def\fps@figure{htbp}
\makeatother







\usepackage{newtx}

\defaultfontfeatures{Scale=MatchLowercase}
\defaultfontfeatures[\rmfamily]{Ligatures=TeX,Scale=1}





\title{Ideas de investigación en economía}


\shorttitle{Ideas de investigación en economía}


\usepackage{etoolbox}



\ccoppy{\textcopyright~2023}



\author{Edison Achalma}



\affiliation{
{Escuela Profesional de Economía, Universidad Nacional de San Cristóbal
de Huamanga}}




\leftheader{Achalma}

\date{2023-06-03}


\abstract{Este abstract será actualizado una vez que se complete el
contenido final del artículo. }

\keywords{keyword1, keyword2}

\authornote{\par{\addORCIDlink{Edison Achalma}{0000-0001-6996-3364}} 
\par{ }
\par{   El autor no tiene conflictos de interés que revelar.    Los
roles de autor se clasificaron utilizando la taxonomía de roles de
colaborador (CRediT; https://credit.niso.org/) de la siguiente
manera:  Edison Achalma:   conceptualización, metodología, análisis
formal, investigación, recursos, curación de
datos, redacción, visualización, supervisión, administración del
proyecto}
\par{La correspondencia relativa a este artículo debe dirigirse a Edison
Achalma, Escuela Profesional de Economía, Universidad Nacional de San
Cristóbal de Huamanga, Portal Independencia N°
57, Ayacucho, AYA 5001, Perú, Email: \href{mailto:elmer.achalma.09@unsch.edu.pe}{elmer.achalma.09@unsch.edu.pe}}
}

\makeatletter
\let\endoldlt\endlongtable
\def\endlongtable{
\hline
\endoldlt
}
\makeatother

\urlstyle{same}



\makeatletter
\@ifpackageloaded{caption}{}{\usepackage{caption}}
\AtBeginDocument{%
\ifdefined\contentsname
  \renewcommand*\contentsname{Tabla de contenidos}
\else
  \newcommand\contentsname{Tabla de contenidos}
\fi
\ifdefined\listfigurename
  \renewcommand*\listfigurename{Lista de Figuras}
\else
  \newcommand\listfigurename{Lista de Figuras}
\fi
\ifdefined\listtablename
  \renewcommand*\listtablename{Lista de Tablas}
\else
  \newcommand\listtablename{Lista de Tablas}
\fi
\ifdefined\figurename
  \renewcommand*\figurename{Figura}
\else
  \newcommand\figurename{Figura}
\fi
\ifdefined\tablename
  \renewcommand*\tablename{Tabla}
\else
  \newcommand\tablename{Tabla}
\fi
}
\@ifpackageloaded{float}{}{\usepackage{float}}
\floatstyle{ruled}
\@ifundefined{c@chapter}{\newfloat{codelisting}{h}{lop}}{\newfloat{codelisting}{h}{lop}[chapter]}
\floatname{codelisting}{Listado}
\newcommand*\listoflistings{\listof{codelisting}{Lista de Listados}}
\makeatother
\makeatletter
\makeatother
\makeatletter
\@ifpackageloaded{caption}{}{\usepackage{caption}}
\@ifpackageloaded{subcaption}{}{\usepackage{subcaption}}
\makeatother
\makeatletter
\@ifpackageloaded{fontawesome5}{}{\usepackage{fontawesome5}}
\makeatother

% From https://tex.stackexchange.com/a/645996/211326
%%% apa7 doesn't want to add appendix section titles in the toc
%%% let's make it do it
\makeatletter
\xpatchcmd{\appendix}
  {\par}
  {\addcontentsline{toc}{section}{\@currentlabelname}\par}
  {}{}
\makeatother

%% Disable longtable counter
%% https://tex.stackexchange.com/a/248395/211326

\usepackage{etoolbox}

\makeatletter
\patchcmd{\LT@caption}
  {\bgroup}
  {\bgroup\global\LTpatch@captiontrue}
  {}{}
\patchcmd{\longtable}
  {\par}
  {\par\global\LTpatch@captionfalse}
  {}{}
\apptocmd{\endlongtable}
  {\ifLTpatch@caption\else\addtocounter{table}{-1}\fi}
  {}{}
\newif\ifLTpatch@caption
\makeatother

\begin{document}

\maketitle


\hypertarget{toc}{}
\tableofcontents
\newpage
\section[Introduction]{Ideas de investigación en economía}

\setcounter{secnumdepth}{3}

\setlength\LTleft{0pt}




En esta publicación presento algunos temas de investigación en economía

\section{Matriz 1: Gasto Público Social y Desarrollo Humano en
Ayacucho}\label{matriz-1-gasto-puxfablico-social-y-desarrollo-humano-en-ayacucho}

\subsection{Título}\label{tuxedtulo}

\textbf{Influencia del Gasto Público Social en el Desarrollo Humano en
la Región de Ayacucho, Periodo 2008-2021}

\subsection{Datos Generales}\label{datos-generales}

\begin{itemize}
\tightlist
\item
  \textbf{Área de investigación:} Economía Pública
\item
  \textbf{Línea de investigación:} Economía, Finanzas Públicas
\item
  \textbf{Tipo de investigación:} Aplicada, Explicativa, Longitudinal
\item
  \textbf{Enfoque:} Cuantitativo
\end{itemize}

\begin{longtable}[]{@{}
  >{\raggedright\arraybackslash}p{(\linewidth - 4\tabcolsep) * \real{0.3333}}
  >{\raggedright\arraybackslash}p{(\linewidth - 4\tabcolsep) * \real{0.3333}}
  >{\raggedright\arraybackslash}p{(\linewidth - 4\tabcolsep) * \real{0.3333}}@{}}
\toprule\noalign{}
\begin{minipage}[b]{\linewidth}\raggedright
PROBLEMAS
\end{minipage} & \begin{minipage}[b]{\linewidth}\raggedright
OBJETIVOS
\end{minipage} & \begin{minipage}[b]{\linewidth}\raggedright
HIPÓTESIS
\end{minipage} \\
\midrule\noalign{}
\endhead
\bottomrule\noalign{}
\endlastfoot
\textbf{PROBLEMA GENERAL} & \textbf{OBJETIVO GENERAL} &
\textbf{HIPÓTESIS GENERAL} \\
¿Cuál es la influencia del gasto público social en el desarrollo humano
en la región de Ayacucho durante el periodo 2008-2021? & Analizar la
influencia del gasto público social en el desarrollo humano en la región
de Ayacucho durante el periodo 2008-2021. & El gasto público social
influye significativa y positivamente en el desarrollo humano en la
región de Ayacucho durante el periodo 2008-2021. \\
\textbf{PROBLEMAS ESPECÍFICOS} & \textbf{OBJETIVOS ESPECÍFICOS} &
\textbf{HIPÓTESIS ESPECÍFICAS} \\
PE1: ¿Cuál es la influencia del gasto público en educación sobre el
índice de desarrollo humano en Ayacucho durante el periodo 2008-2021? &
OE1: Determinar la influencia del gasto público en educación sobre el
índice de desarrollo humano en Ayacucho durante el periodo 2008-2021. &
HE1: El gasto público en educación influye positiva y significativamente
en el índice de desarrollo humano en Ayacucho durante el periodo
2008-2021. \\
PE2: ¿Cuál es la influencia del gasto público en salud sobre el índice
de desarrollo humano en Ayacucho durante el periodo 2008-2021? & OE2:
Evaluar la influencia del gasto público en salud sobre el índice de
desarrollo humano en Ayacucho durante el periodo 2008-2021. & HE2: El
gasto público en salud influye positiva y significativamente en el
índice de desarrollo humano en Ayacucho durante el periodo 2008-2021. \\
PE3: ¿Cuál es la influencia de los programas sociales sobre el índice de
desarrollo humano en Ayacucho durante el periodo 2008-2021? & OE3:
Establecer la influencia de los programas sociales sobre el índice de
desarrollo humano en Ayacucho durante el periodo 2008-2021. & HE3: Los
programas sociales influyen positiva y significativamente en el índice
de desarrollo humano en Ayacucho durante el periodo 2008-2021. \\
\end{longtable}

\subsection{Variables e Indicadores}\label{variables-e-indicadores}

\begin{longtable}[]{@{}
  >{\raggedright\arraybackslash}p{(\linewidth - 8\tabcolsep) * \real{0.2000}}
  >{\raggedright\arraybackslash}p{(\linewidth - 8\tabcolsep) * \real{0.1200}}
  >{\raggedright\arraybackslash}p{(\linewidth - 8\tabcolsep) * \real{0.2600}}
  >{\raggedright\arraybackslash}p{(\linewidth - 8\tabcolsep) * \real{0.2600}}
  >{\raggedright\arraybackslash}p{(\linewidth - 8\tabcolsep) * \real{0.1600}}@{}}
\toprule\noalign{}
\begin{minipage}[b]{\linewidth}\raggedright
VARIABLE
\end{minipage} & \begin{minipage}[b]{\linewidth}\raggedright
TIPO
\end{minipage} & \begin{minipage}[b]{\linewidth}\raggedright
DIMENSIONES
\end{minipage} & \begin{minipage}[b]{\linewidth}\raggedright
INDICADORES
\end{minipage} & \begin{minipage}[b]{\linewidth}\raggedright
ESCALA
\end{minipage} \\
\midrule\noalign{}
\endhead
\bottomrule\noalign{}
\endlastfoot
\textbf{Gasto Público Social} & Independiente & Gasto en Educación & -
Gasto per cápita en educación- Porcentaje del PBI regional destinado a
educación- Gasto por estudiante & Razón (Millones de soles) \\
& & Gasto en Salud & - Gasto per cápita en salud- Porcentaje del PBI
regional destinado a salud- Gasto por beneficiario & Razón (Millones de
soles) \\
& & Programas Sociales & - Presupuesto de programas sociales- Número de
beneficiarios- Cobertura de programas (Juntos, Qali Warma, Pensión 65) &
Razón (Millones de soles) \\
\textbf{Desarrollo Humano} & Dependiente & Educación & - Tasa de
alfabetización- Años promedio de escolaridad- Tasa de asistencia escolar
& Razón (Porcentaje) \\
& & Salud & - Esperanza de vida al nacer- Tasa de mortalidad infantil-
Desnutrición crónica infantil & Razón (Años/Porcentaje) \\
& & Nivel de Vida & - Ingreso per cápita- Acceso a servicios básicos-
Índice de Desarrollo Humano (IDH) & Razón (Soles/Índice) \\
\end{longtable}

\subsection{Metodología}\label{metodologuxeda}

\begin{longtable}[]{@{}
  >{\raggedright\arraybackslash}p{(\linewidth - 2\tabcolsep) * \real{0.4091}}
  >{\raggedright\arraybackslash}p{(\linewidth - 2\tabcolsep) * \real{0.5909}}@{}}
\toprule\noalign{}
\begin{minipage}[b]{\linewidth}\raggedright
ASPECTO
\end{minipage} & \begin{minipage}[b]{\linewidth}\raggedright
DESCRIPCIÓN
\end{minipage} \\
\midrule\noalign{}
\endhead
\bottomrule\noalign{}
\endlastfoot
\textbf{Tipo de Investigación} & - Aplicada- Explicativa-
Longitudinal \\
\textbf{Diseño} & No experimental, longitudinal de tendencia \\
\textbf{Nivel} & Explicativo-correlacional \\
\textbf{Enfoque} & Cuantitativo \\
\textbf{Método} & Hipotético-deductivo, análisis estadístico \\
\textbf{Población} & Región de Ayacucho (provincias y distritos) durante
el periodo 2008-2021 \\
\textbf{Muestra} & Datos agregados a nivel regional y provincial (14
años de observación) \\
\textbf{Técnicas de Recolección} & - Análisis documental- Revisión de
bases de datos secundarias \\
\textbf{Instrumentos} & - Fichas de registro de datos- Base de datos en
Excel/SPSS/Stata \\
\textbf{Fuentes de Datos} & - MEF (Portal de Transparencia Económica)-
INEI (Compendio Estadístico)- PNUD (Informes de Desarrollo Humano)-
Gobiernos regionales y locales- SIAF (Sistema Integrado de
Administración Financiera) \\
\textbf{Técnicas de Análisis} & - Estadística descriptiva (medias,
desviaciones)- Análisis de series de tiempo- Modelos de regresión
múltiple- Modelos econométricos de panel- Pruebas de cointegración-
Análisis de causalidad (Granger) \\
\textbf{Software Estadístico} & Stata, SPSS, R, EViews \\
\end{longtable}

\subsection{Marco Teórico (Conceptos
Clave)}\label{marco-teuxf3rico-conceptos-clave}

\textbf{Gasto Público Social:} Conjunto de erogaciones del Estado
destinadas a mejorar el bienestar de la población, particularmente en
áreas de educación, salud y protección social.

\textbf{Desarrollo Humano:} Proceso de ampliación de las opciones de las
personas, enfocado en tres dimensiones esenciales: vida larga y
saludable, conocimientos y nivel de vida digno (PNUD).

\textbf{Índice de Desarrollo Humano (IDH):} Medida compuesta que evalúa
el promedio de los logros de un país o región en tres dimensiones
básicas del desarrollo humano: salud, educación e ingresos.

\subsection{Referencias Principales}\label{referencias-principales}

\begin{itemize}
\tightlist
\item
  UNDAC. (2024). \emph{Influencia del gasto público social en el
  desarrollo humano en las regiones del Perú, periodo 2008-2021}.
  \url{http://repositorio.undac.edu.pe/bitstream/undac/5363/1/T026_71063724_T.pdf}
\item
  PNUD. (2022). \emph{Informe sobre Desarrollo Humano Perú 2022}. Lima:
  PNUD.
\item
  MEF. (2021). \emph{Distribución del presupuesto por sectores y
  regiones}. Lima: Ministerio de Economía y Finanzas.
\end{itemize}

\section{Matriz 2: Gasto Público y Desigualdad de Ingreso en
Perú}\label{matriz-2-gasto-puxfablico-y-desigualdad-de-ingreso-en-peruxfa}

\subsection{Título}\label{tuxedtulo-1}

\textbf{Influencia del Gasto Público en la Desigualdad de Ingreso en el
Perú, Periodo 2000-2024}

\subsection{Datos Generales}\label{datos-generales-1}

\begin{itemize}
\tightlist
\item
  \textbf{Área de investigación:} Economía Pública, Distribución del
  Ingreso
\item
  \textbf{Línea de investigación:} Economía, Finanzas Públicas,
  Desigualdad
\item
  \textbf{Tipo de investigación:} Aplicada, Explicativa, Longitudinal
\item
  \textbf{Enfoque:} Cuantitativo
\end{itemize}

\begin{longtable}[]{@{}
  >{\raggedright\arraybackslash}p{(\linewidth - 4\tabcolsep) * \real{0.3333}}
  >{\raggedright\arraybackslash}p{(\linewidth - 4\tabcolsep) * \real{0.3333}}
  >{\raggedright\arraybackslash}p{(\linewidth - 4\tabcolsep) * \real{0.3333}}@{}}
\toprule\noalign{}
\begin{minipage}[b]{\linewidth}\raggedright
PROBLEMAS
\end{minipage} & \begin{minipage}[b]{\linewidth}\raggedright
OBJETIVOS
\end{minipage} & \begin{minipage}[b]{\linewidth}\raggedright
HIPÓTESIS
\end{minipage} \\
\midrule\noalign{}
\endhead
\bottomrule\noalign{}
\endlastfoot
\textbf{PROBLEMA GENERAL} & \textbf{OBJETIVO GENERAL} &
\textbf{HIPÓTESIS GENERAL} \\
¿Cuál es la influencia del gasto público en la desigualdad de ingreso en
el Perú durante el periodo 2000-2024? & Determinar la influencia del
gasto público en la desigualdad de ingreso en el Perú durante el periodo
2000-2024. & El gasto público redistributivo influye significativa y
negativamente en la desigualdad de ingreso en el Perú durante el periodo
2000-2024. \\
\textbf{PROBLEMAS ESPECÍFICOS} & \textbf{OBJETIVOS ESPECÍFICOS} &
\textbf{HIPÓTESIS ESPECÍFICAS} \\
PE1: ¿Cuál es el efecto del gasto social en el coeficiente de Gini en el
Perú durante el periodo 2000-2024? & OE1: Analizar el efecto del gasto
social en el coeficiente de Gini en el Perú durante el periodo
2000-2024. & HE1: El gasto social tiene un efecto negativo y
significativo en el coeficiente de Gini en el Perú durante el periodo
2000-2024. \\
PE2: ¿Cuál es el efecto de los programas de transferencias condicionadas
en la distribución del ingreso en el Perú durante el periodo 2000-2024?
& OE2: Evaluar el efecto de los programas de transferencias
condicionadas en la distribución del ingreso en el Perú durante el
periodo 2000-2024. & HE2: Los programas de transferencias condicionadas
reducen significativamente la desigualdad de ingreso en el Perú durante
el periodo 2000-2024. \\
PE3: ¿Cuál es la relación entre la inversión pública en capital humano y
la movilidad social en el Perú durante el periodo 2000-2024? & OE3:
Establecer la relación entre la inversión pública en capital humano y la
movilidad social en el Perú durante el periodo 2000-2024. & HE3: La
inversión pública en capital humano se relaciona positiva y
significativamente con la movilidad social en el Perú durante el periodo
2000-2024. \\
\end{longtable}

\subsection{Variables e Indicadores}\label{variables-e-indicadores-1}

\begin{longtable}[]{@{}
  >{\raggedright\arraybackslash}p{(\linewidth - 8\tabcolsep) * \real{0.2000}}
  >{\raggedright\arraybackslash}p{(\linewidth - 8\tabcolsep) * \real{0.1200}}
  >{\raggedright\arraybackslash}p{(\linewidth - 8\tabcolsep) * \real{0.2600}}
  >{\raggedright\arraybackslash}p{(\linewidth - 8\tabcolsep) * \real{0.2600}}
  >{\raggedright\arraybackslash}p{(\linewidth - 8\tabcolsep) * \real{0.1600}}@{}}
\toprule\noalign{}
\begin{minipage}[b]{\linewidth}\raggedright
VARIABLE
\end{minipage} & \begin{minipage}[b]{\linewidth}\raggedright
TIPO
\end{minipage} & \begin{minipage}[b]{\linewidth}\raggedright
DIMENSIONES
\end{minipage} & \begin{minipage}[b]{\linewidth}\raggedright
INDICADORES
\end{minipage} & \begin{minipage}[b]{\linewidth}\raggedright
ESCALA
\end{minipage} \\
\midrule\noalign{}
\endhead
\bottomrule\noalign{}
\endlastfoot
\textbf{Gasto Público} & Independiente & Gasto Social & - Gasto en
educación (\% PBI)- Gasto en salud (\% PBI)- Gasto en protección social
(\% PBI) & Razón (Porcentaje) \\
& & Transferencias & - Presupuesto de programas sociales- Juntos
(millones de soles)- Pensión 65 (millones de soles)- Qali Warma
(millones de soles) & Razón (Millones de soles) \\
& & Inversión en Capital Humano & - Inversión en educación técnica-
Inversión en capacitación laboral- Becas y programas de formación &
Razón (Millones de soles) \\
\textbf{Desigualdad de Ingreso} & Dependiente & Concentración del
Ingreso & - Coeficiente de Gini- Índice de Palma- Ratio 20/20 (quintil
superior/inferior) & Razón (Índice 0-1) \\
& & Pobreza y Distribución & - Tasa de pobreza monetaria- Brecha de
pobreza- Severidad de la pobreza & Razón (Porcentaje) \\
& & Movilidad Social & - Índice de movilidad intergeneracional- Acceso a
servicios públicos- Oportunidades educativas & Razón/Ordinal \\
\end{longtable}

\subsection{Metodología}\label{metodologuxeda-1}

\begin{longtable}[]{@{}
  >{\raggedright\arraybackslash}p{(\linewidth - 2\tabcolsep) * \real{0.4091}}
  >{\raggedright\arraybackslash}p{(\linewidth - 2\tabcolsep) * \real{0.5909}}@{}}
\toprule\noalign{}
\begin{minipage}[b]{\linewidth}\raggedright
ASPECTO
\end{minipage} & \begin{minipage}[b]{\linewidth}\raggedright
DESCRIPCIÓN
\end{minipage} \\
\midrule\noalign{}
\endhead
\bottomrule\noalign{}
\endlastfoot
\textbf{Tipo de Investigación} & - Aplicada- Explicativa-
Longitudinal \\
\textbf{Diseño} & No experimental, longitudinal de tendencia (panel de
datos) \\
\textbf{Nivel} & Explicativo-correlacional \\
\textbf{Enfoque} & Cuantitativo \\
\textbf{Método} & Hipotético-deductivo, econométrico \\
\textbf{Población} & República del Perú (24 departamentos + Lima
Metropolitana) durante el periodo 2000-2024 \\
\textbf{Muestra} & Datos de panel: 25 unidades territoriales × 25 años =
625 observaciones \\
\textbf{Técnicas de Recolección} & - Análisis documental- Revisión de
bases de datos oficiales \\
\textbf{Instrumentos} & - Fichas de registro- Matrices de datos
longitudinales \\
\textbf{Fuentes de Datos} & - CEPAL (Base de datos estadísticos)- INEI
(ENAHO, PBI, distribución del ingreso)- MEF (Transparencia Económica)-
Banco Mundial (Indicadores de desarrollo)- BCR (Series estadísticas) \\
\textbf{Técnicas de Análisis} & - Estadística descriptiva- Modelos de
datos de panel (efectos fijos y aleatorios)- Prueba de Hausman- Modelos
de series de tiempo (VAR, VECM)- Análisis de cointegración- Regresión
cuantílica \\
\textbf{Software Estadístico} & Stata, R, EViews, Python \\
\end{longtable}

\subsection{Marco Teórico (Conceptos
Clave)}\label{marco-teuxf3rico-conceptos-clave-1}

\textbf{Coeficiente de Gini:} Medida de desigualdad en la distribución
del ingreso que varía entre 0 (igualdad perfecta) y 1 (desigualdad
máxima).

\textbf{Gasto Redistributivo:} Componente del gasto público diseñado
para modificar la distribución del ingreso mediante transferencias y
provisión de servicios públicos.

\textbf{Transferencias Condicionadas:} Programas de asistencia social
que otorgan recursos monetarios a hogares pobres condicionados al
cumplimiento de ciertos comportamientos (educación, salud).

\subsection{Referencias Principales}\label{referencias-principales-1}

\begin{itemize}
\tightlist
\item
  CEPAL. (2024). \emph{Base de datos estadísticos para América Latina}.
  \url{https://statistics.cepal.org/portal/databank/}
\item
  Lustig, N. (2017). \emph{El impacto del sistema tributario y el gasto
  social en la distribución del ingreso y la pobreza en América Latina}.
  CEQ Working Paper Series.
\item
  INEI. (2023). \emph{Evolución de la pobreza monetaria 2000-2023}.
  Lima: Instituto Nacional de Estadística e Informática.
\end{itemize}

\section{Matriz 3: Eficiencia del Gasto Público en Educación en
Ayacucho}\label{matriz-3-eficiencia-del-gasto-puxfablico-en-educaciuxf3n-en-ayacucho}

\subsection{Título}\label{tuxedtulo-2}

\textbf{Evaluación de la Eficiencia del Gasto Público en Educación en
los Distritos de Ayacucho, Periodo 2015-2023}

\subsection{Datos Generales}\label{datos-generales-2}

\begin{itemize}
\tightlist
\item
  \textbf{Área de investigación:} Economía de la Educación, Finanzas
  Públicas
\item
  \textbf{Línea de investigación:} Economía, Administración Pública
\item
  \textbf{Tipo de investigación:} Aplicada, Evaluativa, Transversal
\item
  \textbf{Enfoque:} Cuantitativo
\end{itemize}

\begin{longtable}[]{@{}
  >{\raggedright\arraybackslash}p{(\linewidth - 4\tabcolsep) * \real{0.3333}}
  >{\raggedright\arraybackslash}p{(\linewidth - 4\tabcolsep) * \real{0.3333}}
  >{\raggedright\arraybackslash}p{(\linewidth - 4\tabcolsep) * \real{0.3333}}@{}}
\toprule\noalign{}
\begin{minipage}[b]{\linewidth}\raggedright
PROBLEMAS
\end{minipage} & \begin{minipage}[b]{\linewidth}\raggedright
OBJETIVOS
\end{minipage} & \begin{minipage}[b]{\linewidth}\raggedright
HIPÓTESIS
\end{minipage} \\
\midrule\noalign{}
\endhead
\bottomrule\noalign{}
\endlastfoot
\textbf{PROBLEMA GENERAL} & \textbf{OBJETIVO GENERAL} &
\textbf{HIPÓTESIS GENERAL} \\
¿Cuál es el nivel de eficiencia del gasto público en educación en los
distritos de Ayacucho durante el periodo 2015-2023? & Evaluar el nivel
de eficiencia del gasto público en educación en los distritos de
Ayacucho durante el periodo 2015-2023. & El gasto público en educación
en los distritos de Ayacucho presenta niveles heterogéneos de eficiencia
durante el periodo 2015-2023, con mayor eficiencia en distritos
urbanos. \\
\textbf{PROBLEMAS ESPECÍFICOS} & \textbf{OBJETIVOS ESPECÍFICOS} &
\textbf{HIPÓTESIS ESPECÍFICAS} \\
PE1: ¿Cuál es la relación entre el gasto por estudiante y el rendimiento
académico en los distritos de Ayacucho? & OE1: Determinar la relación
entre el gasto por estudiante y el rendimiento académico en los
distritos de Ayacucho. & HE1: Existe una relación positiva pero con
rendimientos decrecientes entre el gasto por estudiante y el rendimiento
académico en los distritos de Ayacucho. \\
PE2: ¿Cuáles son los factores que explican las diferencias en eficiencia
del gasto educativo entre distritos urbanos y rurales de Ayacucho? &
OE2: Identificar los factores que explican las diferencias en eficiencia
del gasto educativo entre distritos urbanos y rurales de Ayacucho. &
HE2: Las diferencias en infraestructura, disponibilidad de docentes
calificados y acceso a tecnología explican las brechas de eficiencia
entre distritos urbanos y rurales. \\
PE3: ¿Qué distritos presentan mejores prácticas en la gestión del gasto
educativo en Ayacucho? & OE3: Identificar los distritos que presentan
mejores prácticas en la gestión del gasto educativo en Ayacucho. & HE3:
Los distritos con mayor eficiencia implementan prácticas de gestión
orientadas a resultados y monitoreo continuo. \\
\end{longtable}

\subsection{Variables e Indicadores}\label{variables-e-indicadores-2}

\begin{longtable}[]{@{}
  >{\raggedright\arraybackslash}p{(\linewidth - 8\tabcolsep) * \real{0.2000}}
  >{\raggedright\arraybackslash}p{(\linewidth - 8\tabcolsep) * \real{0.1200}}
  >{\raggedright\arraybackslash}p{(\linewidth - 8\tabcolsep) * \real{0.2600}}
  >{\raggedright\arraybackslash}p{(\linewidth - 8\tabcolsep) * \real{0.2600}}
  >{\raggedright\arraybackslash}p{(\linewidth - 8\tabcolsep) * \real{0.1600}}@{}}
\toprule\noalign{}
\begin{minipage}[b]{\linewidth}\raggedright
VARIABLE
\end{minipage} & \begin{minipage}[b]{\linewidth}\raggedright
TIPO
\end{minipage} & \begin{minipage}[b]{\linewidth}\raggedright
DIMENSIONES
\end{minipage} & \begin{minipage}[b]{\linewidth}\raggedright
INDICADORES
\end{minipage} & \begin{minipage}[b]{\linewidth}\raggedright
ESCALA
\end{minipage} \\
\midrule\noalign{}
\endhead
\bottomrule\noalign{}
\endlastfoot
\textbf{Gasto Público en Educación} & Independiente & Recursos
Financieros & - Gasto por estudiante (soles)- Presupuesto ejecutado en
educación- Porcentaje de ejecución presupuestal & Razón
(Soles/Porcentaje) \\
& & Recursos Humanos & - Número de docentes- Ratio estudiante/docente-
Docentes con formación continua & Razón \\
& & Infraestructura & - Inversión en infraestructura educativa- Escuelas
con servicios básicos completos- Aulas en buen estado &
Razón/Proporción \\
\textbf{Eficiencia Educativa} & Dependiente & Resultados Académicos & -
Puntaje promedio en ECE (2° y 4° primaria)- Porcentaje de estudiantes en
nivel satisfactorio- Tasa de aprobación & Razón (Puntaje/Porcentaje) \\
& & Cobertura y Acceso & - Tasa de matrícula- Tasa de asistencia- Tasa
de deserción escolar & Razón (Porcentaje) \\
& & Equidad & - Brecha urbano-rural en rendimiento- Equidad de género en
logros- Acceso a servicios complementarios & Razón/Proporción \\
\end{longtable}

\subsection{Metodología}\label{metodologuxeda-2}

\begin{longtable}[]{@{}
  >{\raggedright\arraybackslash}p{(\linewidth - 2\tabcolsep) * \real{0.4091}}
  >{\raggedright\arraybackslash}p{(\linewidth - 2\tabcolsep) * \real{0.5909}}@{}}
\toprule\noalign{}
\begin{minipage}[b]{\linewidth}\raggedright
ASPECTO
\end{minipage} & \begin{minipage}[b]{\linewidth}\raggedright
DESCRIPCIÓN
\end{minipage} \\
\midrule\noalign{}
\endhead
\bottomrule\noalign{}
\endlastfoot
\textbf{Tipo de Investigación} & - Aplicada- Evaluativa- Transversal
comparativa \\
\textbf{Diseño} & No experimental, transversal comparativo \\
\textbf{Nivel} & Descriptivo-explicativo \\
\textbf{Enfoque} & Cuantitativo \\
\textbf{Método} & Análisis Envolvente de Datos (DEA), análisis
comparativo \\
\textbf{Población} & 119 distritos de la región de Ayacucho \\
\textbf{Muestra} & Muestra censal (todos los distritos) o estratificada
por ubicación geográfica y tamaño \\
\textbf{Técnicas de Recolección} & - Análisis documental- Revisión de
bases de datos administrativas \\
\textbf{Instrumentos} & - Fichas de registro- Matrices de datos
distritales \\
\textbf{Fuentes de Datos} & - MINEDU (ESCALE, ECE)- MEF (SIAF,
Transparencia Económica)- UGEL Ayacucho- Censo Escolar- INEI (población
estudiantil) \\
\textbf{Técnicas de Análisis} & - Análisis Envolvente de Datos (DEA)-
Índices de eficiencia técnica- Análisis de regresión múltiple-
Estadística descriptiva comparativa- Análisis de correlación-
Benchmarking \\
\textbf{Software Estadístico} & DEA-Solver, SPSS, Stata, R, MaxDEA \\
\end{longtable}

\subsection{Marco Teórico (Conceptos
Clave)}\label{marco-teuxf3rico-conceptos-clave-2}

\textbf{Eficiencia en Educación:} Relación entre los recursos invertidos
(inputs) y los resultados educativos obtenidos (outputs), buscando
maximizar logros con recursos dados o minimizar recursos para alcanzar
resultados deseados.

\textbf{Análisis Envolvente de Datos (DEA):} Técnica no paramétrica que
mide la eficiencia relativa de unidades tomadoras de decisión
(distritos) comparando sus desempeños en múltiples inputs y outputs.

\textbf{Evaluación Censal de Estudiantes (ECE):} Evaluación
estandarizada que aplica el MINEDU para medir logros de aprendizaje en
comunicación y matemática.

\subsection{Enfoque Metodológico DEA}\label{enfoque-metodoluxf3gico-dea}

El modelo DEA propuesto considera:

\textbf{Inputs:}

\begin{itemize}
\tightlist
\item
  Gasto por estudiante
\item
  Número de docentes
\item
  Infraestructura educativa
\end{itemize}

\textbf{Outputs:}

\begin{itemize}
\tightlist
\item
  Puntaje promedio ECE
\item
  Tasa de aprobación
\item
  Cobertura educativa
\end{itemize}

\textbf{Modelo:} DEA-CRS (retornos constantes a escala) o DEA-VRS
(retornos variables a escala)

\subsection{Referencias Principales}\label{referencias-principales-2}

\begin{itemize}
\tightlist
\item
  Hanushek, E. A., \& Woessmann, L. (2020). \emph{The Economic Impacts
  of Learning Losses}. OECD Education Working Papers.
\item
  MINEDU. (2023). \emph{Resultados de la Evaluación Censal de
  Estudiantes 2023}. Lima: Ministerio de Educación.
\item
  Afonso, A., \& St.~Aubyn, M. (2005). \emph{Non-parametric approaches
  to education and health efficiency in OECD countries}. Journal of
  Applied Economics, 8(2), 227-246.
\end{itemize}

\section{Matriz 4: Inversión Pública en Infraestructura Vial y
Crecimiento
Económico}\label{matriz-4-inversiuxf3n-puxfablica-en-infraestructura-vial-y-crecimiento-econuxf3mico}

\subsection{Título}\label{tuxedtulo-3}

\textbf{Efectos de la Inversión Pública en Infraestructura Vial en el
Crecimiento Económico de los Distritos de Ayacucho, Periodo 2010-2023}

\subsection{Datos Generales}\label{datos-generales-3}

\begin{itemize}
\tightlist
\item
  \textbf{Área de investigación:} Economía del Desarrollo,
  Infraestructura
\item
  \textbf{Línea de investigación:} Economía, Geografía Económica
\item
  \textbf{Tipo de investigación:} Aplicada, Explicativa, Longitudinal
\item
  \textbf{Enfoque:} Cuantitativo
\end{itemize}

\begin{longtable}[]{@{}
  >{\raggedright\arraybackslash}p{(\linewidth - 4\tabcolsep) * \real{0.3333}}
  >{\raggedright\arraybackslash}p{(\linewidth - 4\tabcolsep) * \real{0.3333}}
  >{\raggedright\arraybackslash}p{(\linewidth - 4\tabcolsep) * \real{0.3333}}@{}}
\toprule\noalign{}
\begin{minipage}[b]{\linewidth}\raggedright
PROBLEMAS
\end{minipage} & \begin{minipage}[b]{\linewidth}\raggedright
OBJETIVOS
\end{minipage} & \begin{minipage}[b]{\linewidth}\raggedright
HIPÓTESIS
\end{minipage} \\
\midrule\noalign{}
\endhead
\bottomrule\noalign{}
\endlastfoot
\textbf{PROBLEMA GENERAL} & \textbf{OBJETIVO GENERAL} &
\textbf{HIPÓTESIS GENERAL} \\
¿Cuáles son los efectos de la inversión pública en infraestructura vial
en el crecimiento económico de los distritos de Ayacucho durante el
periodo 2010-2023? & Determinar los efectos de la inversión pública en
infraestructura vial en el crecimiento económico de los distritos de
Ayacucho durante el periodo 2010-2023. & La inversión pública en
infraestructura vial genera efectos positivos y significativos en el
crecimiento económico de los distritos de Ayacucho durante el periodo
2010-2023. \\
\textbf{PROBLEMAS ESPECÍFICOS} & \textbf{OBJETIVOS ESPECÍFICOS} &
\textbf{HIPÓTESIS ESPECÍFICAS} \\
PE1: ¿Cuál es el impacto de la inversión vial en el Producto Bruto
Interno distrital en Ayacucho? & OE1: Cuantificar el impacto de la
inversión vial en el Producto Bruto Interno distrital en Ayacucho. &
HE1: La inversión vial impacta positiva y significativamente en el
Producto Bruto Interno distrital en Ayacucho. \\
PE2: ¿Cómo afecta la mejora de carreteras en el empleo y la actividad
económica en los distritos de Ayacucho? & OE2: Analizar cómo la mejora
de carreteras afecta el empleo y la actividad económica en los distritos
de Ayacucho. & HE2: La mejora de carreteras aumenta significativamente
el empleo y dinamiza la actividad económica en los distritos de
Ayacucho. \\
PE3: ¿Cuál es la relación entre conectividad vial y acceso a mercados en
los distritos rurales de Ayacucho? & OE3: Establecer la relación entre
conectividad vial y acceso a mercados en los distritos rurales de
Ayacucho. & HE3: Existe una relación positiva y significativa entre
conectividad vial y acceso a mercados en los distritos rurales de
Ayacucho. \\
\end{longtable}

\subsection{Variables e Indicadores}\label{variables-e-indicadores-3}

\begin{longtable}[]{@{}
  >{\raggedright\arraybackslash}p{(\linewidth - 8\tabcolsep) * \real{0.2000}}
  >{\raggedright\arraybackslash}p{(\linewidth - 8\tabcolsep) * \real{0.1200}}
  >{\raggedright\arraybackslash}p{(\linewidth - 8\tabcolsep) * \real{0.2600}}
  >{\raggedright\arraybackslash}p{(\linewidth - 8\tabcolsep) * \real{0.2600}}
  >{\raggedright\arraybackslash}p{(\linewidth - 8\tabcolsep) * \real{0.1600}}@{}}
\toprule\noalign{}
\begin{minipage}[b]{\linewidth}\raggedright
VARIABLE
\end{minipage} & \begin{minipage}[b]{\linewidth}\raggedright
TIPO
\end{minipage} & \begin{minipage}[b]{\linewidth}\raggedright
DIMENSIONES
\end{minipage} & \begin{minipage}[b]{\linewidth}\raggedright
INDICADORES
\end{minipage} & \begin{minipage}[b]{\linewidth}\raggedright
ESCALA
\end{minipage} \\
\midrule\noalign{}
\endhead
\bottomrule\noalign{}
\endlastfoot
\textbf{Inversión en Infraestructura Vial} & Independiente & Inversión
Financiera & - Monto de inversión en carreteras (millones de soles)-
Inversión per cápita en vialidad- Porcentaje del presupuesto destinado a
infraestructura vial & Razón (Millones de soles) \\
& & Infraestructura Física & - Kilómetros de carreteras pavimentadas-
Kilómetros de carreteras afirmadas- Número de puentes
construidos/rehabilitados & Razón (Kilómetros/Número) \\
& & Conectividad & - Índice de conectividad vial- Tiempo promedio de
viaje a capital provincial- Densidad vial (km/km²) & Razón/Índice \\
\textbf{Crecimiento Económico} & Dependiente & Producción & - Producto
Bruto Interno distrital- Tasa de crecimiento del PBI- Valor Agregado
Bruto sectorial & Razón (Millones de soles/Porcentaje) \\
& & Empleo & - Tasa de empleo adecuado- PEA ocupada- Empleo por sectores
económicos & Razón (Porcentaje) \\
& & Actividad Económica & - Número de empresas formales- Volumen de
comercio- Producción agrícola comercializada & Razón/Número \\
& & Acceso a Mercados & - Distancia a mercados principales- Costo de
transporte- Frecuencia de transporte público & Razón
(Km/Soles/Número) \\
\end{longtable}

\subsection{Metodología}\label{metodologuxeda-3}

\begin{longtable}[]{@{}
  >{\raggedright\arraybackslash}p{(\linewidth - 2\tabcolsep) * \real{0.4091}}
  >{\raggedright\arraybackslash}p{(\linewidth - 2\tabcolsep) * \real{0.5909}}@{}}
\toprule\noalign{}
\begin{minipage}[b]{\linewidth}\raggedright
ASPECTO
\end{minipage} & \begin{minipage}[b]{\linewidth}\raggedright
DESCRIPCIÓN
\end{minipage} \\
\midrule\noalign{}
\endhead
\bottomrule\noalign{}
\endlastfoot
\textbf{Tipo de Investigación} & - Aplicada- Explicativa-
Longitudinal \\
\textbf{Diseño} & No experimental, longitudinal de panel \\
\textbf{Nivel} & Explicativo-correlacional \\
\textbf{Enfoque} & Cuantitativo \\
\textbf{Método} & Econométrico, análisis de panel de datos \\
\textbf{Población} & 119 distritos de la región de Ayacucho durante el
periodo 2010-2023 \\
\textbf{Muestra} & Panel balanceado: 119 distritos × 14 años = 1,666
observaciones \\
\textbf{Técnicas de Recolección} & - Análisis documental- Revisión de
bases de datos administrativas- Georeferenciación de proyectos viales \\
\textbf{Instrumentos} & - Fichas de registro- Bases de datos
geoespaciales- Matrices de panel \\
\textbf{Fuentes de Datos} & - MEF (Consulta de Inversión Pública, SIAF)-
MTC (Inventario Vial)- INEI (PBI distrital, empleo, censos)- Gobiernos
regionales y provinciales- PROVIAS (Proyectos viales) \\
\textbf{Técnicas de Análisis} & - Modelos de datos de panel (efectos
fijos y aleatorios)- Prueba de Hausman- Modelo de diferencias en
diferencias (DID)- Variables instrumentales- Análisis de impacto
cuasi-experimental- Estadística espacial \\
\textbf{Software Estadístico} & Stata, R, QGIS (análisis espacial),
EViews \\
\end{longtable}

\subsection{Marco Teórico (Conceptos
Clave)}\label{marco-teuxf3rico-conceptos-clave-3}

\textbf{Inversión en Infraestructura Vial:} Gasto de capital destinado a
la construcción, rehabilitación y mantenimiento de carreteras, caminos
rurales, puentes y vías de comunicación terrestre.

\textbf{Crecimiento Económico:} Incremento sostenido de la producción de
bienes y servicios en una economía, medido típicamente por el aumento
del Producto Bruto Interno.

\textbf{Conectividad:} Grado de integración física y funcional de un
territorio con centros económicos, determinado por la calidad y
accesibilidad de la red vial.

\subsection{Modelo Econométrico
Propuesto}\label{modelo-economuxe9trico-propuesto}

\textbf{Especificación básica (Panel de datos):}

\begin{verbatim}
ln(PBI_it) = β₀ + β₁·ln(InvVial_it) + β₂·X_it + α_i + τ_t + ε_it
\end{verbatim}

Donde:

\begin{itemize}
\tightlist
\item
  PBI\_it: Producto Bruto Interno del distrito i en el año t
\item
  InvVial\_it: Inversión en infraestructura vial
\item
  X\_it: Vector de variables de control (población, educación, etc.)
\item
  α\_i: Efectos fijos distritales
\item
  τ\_t: Efectos fijos temporales
\item
  ε\_it: Término de error
\end{itemize}

\subsection{Referencias Principales}\label{referencias-principales-3}

\begin{itemize}
\tightlist
\item
  Calderón, C., \& Servén, L. (2010). \emph{Infrastructure in Latin
  America}. World Bank Policy Research Working Paper No.~5317.
\item
  Escobal, J., \& Ponce, C. (2008). \emph{Economic impact of rural road
  infrastructure in Peru}. GRADE Working Paper.
\item
  Banco Mundial. (2020). \emph{Infraestructura para el desarrollo:
  Evidencia del impacto económico}. Washington DC: Banco Mundial.
\end{itemize}

\section{Matriz 5: Inversión Pública y Crecimiento Económico - Series de
Tiempo}\label{matriz-5-inversiuxf3n-puxfablica-y-crecimiento-econuxf3mico---series-de-tiempo}

\subsection{Título}\label{tuxedtulo-4}

\textbf{Inversión Pública y el Crecimiento Económico en Ayacucho: Un
Análisis de Series de Tiempo, Periodo 2000-2023}

\subsection{Datos Generales}\label{datos-generales-4}

\begin{itemize}
\tightlist
\item
  \textbf{Área de investigación:} Macroeconomía, Finanzas Públicas
\item
  \textbf{Línea de investigación:} Econometría, Economía
\item
  \textbf{Tipo de investigación:} Aplicada, Explicativa, Longitudinal
\item
  \textbf{Enfoque:} Cuantitativo con análisis de series temporales
\end{itemize}

\begin{longtable}[]{@{}
  >{\raggedright\arraybackslash}p{(\linewidth - 4\tabcolsep) * \real{0.3333}}
  >{\raggedright\arraybackslash}p{(\linewidth - 4\tabcolsep) * \real{0.3333}}
  >{\raggedright\arraybackslash}p{(\linewidth - 4\tabcolsep) * \real{0.3333}}@{}}
\toprule\noalign{}
\begin{minipage}[b]{\linewidth}\raggedright
PROBLEMAS
\end{minipage} & \begin{minipage}[b]{\linewidth}\raggedright
OBJETIVOS
\end{minipage} & \begin{minipage}[b]{\linewidth}\raggedright
HIPÓTESIS
\end{minipage} \\
\midrule\noalign{}
\endhead
\bottomrule\noalign{}
\endlastfoot
\textbf{PROBLEMA GENERAL} & \textbf{OBJETIVO GENERAL} &
\textbf{HIPÓTESIS GENERAL} \\
¿Cuál es la relación de largo plazo entre la inversión pública y el
crecimiento económico en Ayacucho durante el periodo 2000-2023? &
Determinar la relación de largo plazo entre la inversión pública y el
crecimiento económico en Ayacucho durante el periodo 2000-2023. & Existe
una relación de cointegración positiva y significativa entre la
inversión pública y el crecimiento económico en Ayacucho durante el
periodo 2000-2023. \\
\textbf{PROBLEMAS ESPECÍFICOS} & \textbf{OBJETIVOS ESPECÍFICOS} &
\textbf{HIPÓTESIS ESPECÍFICAS} \\
PE1: ¿Existe causalidad en el sentido de Granger entre la inversión
pública y el PBI regional de Ayacucho? & OE1: Evaluar la causalidad en
el sentido de Granger entre la inversión pública y el PBI regional de
Ayacucho. & HE1: Existe causalidad unidireccional de la inversión
pública hacia el PBI regional de Ayacucho. \\
PE2: ¿Cuál es el efecto de corto y largo plazo de la inversión pública
sobre el crecimiento económico en Ayacucho? & OE2: Estimar el efecto de
corto y largo plazo de la inversión pública sobre el crecimiento
económico en Ayacucho. & HE2: La inversión pública tiene efectos
positivos tanto en el corto como en el largo plazo sobre el crecimiento
económico en Ayacucho. \\
PE3: ¿Qué tipo de inversión pública (social o productiva) tiene mayor
impacto en el crecimiento económico de Ayacucho? & OE3: Comparar el
impacto de la inversión pública social versus productiva en el
crecimiento económico de Ayacucho. & HE3: La inversión pública
productiva tiene un mayor impacto en el crecimiento económico de
Ayacucho que la inversión social. \\
\end{longtable}

\subsection{Variables e Indicadores}\label{variables-e-indicadores-4}

\begin{longtable}[]{@{}
  >{\raggedright\arraybackslash}p{(\linewidth - 10\tabcolsep) * \real{0.1613}}
  >{\raggedright\arraybackslash}p{(\linewidth - 10\tabcolsep) * \real{0.0968}}
  >{\raggedright\arraybackslash}p{(\linewidth - 10\tabcolsep) * \real{0.2097}}
  >{\raggedright\arraybackslash}p{(\linewidth - 10\tabcolsep) * \real{0.2097}}
  >{\raggedright\arraybackslash}p{(\linewidth - 10\tabcolsep) * \real{0.1290}}
  >{\raggedright\arraybackslash}p{(\linewidth - 10\tabcolsep) * \real{0.1935}}@{}}
\toprule\noalign{}
\begin{minipage}[b]{\linewidth}\raggedright
VARIABLE
\end{minipage} & \begin{minipage}[b]{\linewidth}\raggedright
TIPO
\end{minipage} & \begin{minipage}[b]{\linewidth}\raggedright
DIMENSIONES
\end{minipage} & \begin{minipage}[b]{\linewidth}\raggedright
INDICADORES
\end{minipage} & \begin{minipage}[b]{\linewidth}\raggedright
ESCALA
\end{minipage} & \begin{minipage}[b]{\linewidth}\raggedright
FRECUENCIA
\end{minipage} \\
\midrule\noalign{}
\endhead
\bottomrule\noalign{}
\endlastfoot
\textbf{Inversión Pública} & Independiente & Inversión Total & -
Inversión pública total (millones de soles constantes)- Inversión
pública per cápita- Inversión como \% del PBI regional & Razón &
Trimestral/Anual \\
& & Inversión por Tipo & - Inversión en infraestructura productiva-
Inversión en capital humano (educación, salud)- Inversión en servicios
básicos & Razón & Trimestral/Anual \\
& & Ejecución & - Porcentaje de ejecución del presupuesto- Monto
devengado- Proyectos culminados & Razón/Porcentaje & Trimestral/Anual \\
\textbf{Crecimiento Económico} & Dependiente & Producto Bruto & - PBI
regional (millones de soles constantes)- Tasa de crecimiento del PBI-
PBI per cápita & Razón & Trimestral/Anual \\
& & Sectores Económicos & - VAB sector primario- VAB sector secundario-
VAB sector terciario & Razón & Anual \\
\textbf{Variables de Control} & Control & Contexto Económico & -
Inflación regional- PBI nacional- Tipo de cambio- Términos de
intercambio & Razón & Mensual/Trimestral \\
\end{longtable}

\subsection{Metodología}\label{metodologuxeda-4}

\begin{longtable}[]{@{}
  >{\raggedright\arraybackslash}p{(\linewidth - 2\tabcolsep) * \real{0.4091}}
  >{\raggedright\arraybackslash}p{(\linewidth - 2\tabcolsep) * \real{0.5909}}@{}}
\toprule\noalign{}
\begin{minipage}[b]{\linewidth}\raggedright
ASPECTO
\end{minipage} & \begin{minipage}[b]{\linewidth}\raggedright
DESCRIPCIÓN
\end{minipage} \\
\midrule\noalign{}
\endhead
\bottomrule\noalign{}
\endlastfoot
\textbf{Tipo de Investigación} & - Aplicada- Explicativa-
Longitudinal \\
\textbf{Diseño} & No experimental, longitudinal de series de tiempo \\
\textbf{Nivel} & Explicativo-correlacional \\
\textbf{Enfoque} & Cuantitativo con modelamiento econométrico \\
\textbf{Método} & Series de tiempo, análisis econométrico \\
\textbf{Población} & Serie temporal de datos económicos de Ayacucho
(2000-2023) \\
\textbf{Muestra} & Serie completa: 24 años de observaciones anuales o 96
trimestres \\
\textbf{Técnicas de Recolección} & - Análisis documental- Extracción de
bases de datos estadísticas \\
\textbf{Instrumentos} & - Fichas de registro- Bases de datos en series
temporales \\
\textbf{Fuentes de Datos} & - BCRP (PBI regional, inflación)- MEF
(Portal de Transparencia, inversión pública)- INEI (PBI regional, censos
económicos)- Gobierno Regional de Ayacucho \\
\textbf{Técnicas de Análisis} & - Análisis de estacionariedad (ADF, PP,
KPSS)- Cointegración (Engle-Granger, Johansen)- Modelos ARIMA, VAR,
VECM- Causalidad de Granger- Funciones impulso-respuesta- Descomposición
de varianza \\
\textbf{Software Estadístico} & EViews, Stata, R (forecast, vars, urca),
Python (statsmodels) \\
\end{longtable}

\subsection{Marco Teórico (Conceptos
Clave)}\label{marco-teuxf3rico-conceptos-clave-4}

\textbf{Cointegración:} Relación de equilibrio de largo plazo entre dos
o más series temporales no estacionarias.

\textbf{Modelo de Vectores Autorregresivos (VAR):} Sistema de ecuaciones
que permite analizar la interdependencia dinámica entre múltiples series
temporales.

\textbf{Modelo de Corrección de Errores (VECM):} Extensión del modelo
VAR que incorpora relaciones de cointegración para capturar dinámicas de
corto y largo plazo.

\textbf{Causalidad de Granger:} Prueba estadística que determina si una
serie temporal ayuda a predecir otra, indicando precedencia temporal
pero no causalidad en sentido estricto.

\subsection{Estrategia Econométrica}\label{estrategia-economuxe9trica}

\textbf{Paso 1: Análisis de Estacionariedad}

\begin{itemize}
\tightlist
\item
  Pruebas ADF (Augmented Dickey-Fuller)
\item
  Pruebas Phillips-Perron
\item
  Pruebas KPSS
\end{itemize}

\textbf{Paso 2: Pruebas de Cointegración}

\begin{itemize}
\tightlist
\item
  Test de Engle-Granger (dos variables)
\item
  Test de Johansen (múltiples variables)
\end{itemize}

\textbf{Paso 3: Estimación de Modelos}

\begin{itemize}
\tightlist
\item
  Si hay cointegración: VECM
\item
  Si no hay cointegración: VAR en diferencias
\end{itemize}

\textbf{Paso 4: Análisis de Causalidad}

\begin{itemize}
\tightlist
\item
  Test de causalidad de Granger
\end{itemize}

\textbf{Paso 5: Análisis Dinámico}

\begin{itemize}
\tightlist
\item
  Funciones impulso-respuesta
\item
  Descomposición de varianza
\end{itemize}

\subsection{Modelo Econométrico
Propuesto}\label{modelo-economuxe9trico-propuesto-1}

\textbf{Modelo VECM (si existe cointegración):}

\begin{verbatim}
ΔY_t = α(Y_{t-1} - βX_{t-1}) + Σγ_iΔY_{t-i} + Σδ_iΔX_{t-i} + ε_t
\end{verbatim}

Donde: - Y\_t: ln(PBI regional) - X\_t: ln(Inversión pública) - α:
Velocidad de ajuste - β: Coeficiente de cointegración (relación de largo
plazo) - γ, δ: Efectos de corto plazo

\textbf{Modelo VAR (si no hay cointegración):}

\begin{verbatim}
Y_t = c + ΣA_i·Y_{t-i} + ε_t
\end{verbatim}

Donde Y\_t es un vector que incluye PBI e inversión pública.

\subsection{Referencias Principales}\label{referencias-principales-4}

\begin{itemize}
\tightlist
\item
  Aschauer, D. A. (1989). \emph{Is public expenditure productive?}
  Journal of Monetary Economics, 23(2), 177-200.
\item
  Calderón, C., \& Servén, L. (2003). \emph{The output cost of Latin
  America's infrastructure gap}. Central Bank of Chile Working Paper
  No.~186.
\item
  Enders, W. (2015). \emph{Applied Econometric Time Series} (4th ed.).
  Wiley.
\end{itemize}

\section{Matriz 6: Gestión Pública y Desarrollo
Sostenible}\label{matriz-6-gestiuxf3n-puxfablica-y-desarrollo-sostenible}

\subsection{Título}\label{tuxedtulo-5}

\textbf{La Gestión Pública y su Incidencia en el Desarrollo Sostenible
del Distrito de Santo Tomás, Chumbivilcas, Periodo 2020-2023}

\subsection{Datos Generales}\label{datos-generales-5}

\begin{itemize}
\tightlist
\item
  \textbf{Área de investigación:} Administración Pública, Desarrollo
  Sostenible
\item
  \textbf{Línea de investigación:} Administración Pública, Ciencias
  Ambientales
\item
  \textbf{Tipo de investigación:} Aplicada, Explicativa, Transversal
\item
  \textbf{Enfoque:} Mixto (cuantitativo-cualitativo)
\end{itemize}

\begin{longtable}[]{@{}
  >{\raggedright\arraybackslash}p{(\linewidth - 4\tabcolsep) * \real{0.3333}}
  >{\raggedright\arraybackslash}p{(\linewidth - 4\tabcolsep) * \real{0.3333}}
  >{\raggedright\arraybackslash}p{(\linewidth - 4\tabcolsep) * \real{0.3333}}@{}}
\toprule\noalign{}
\begin{minipage}[b]{\linewidth}\raggedright
PROBLEMAS
\end{minipage} & \begin{minipage}[b]{\linewidth}\raggedright
OBJETIVOS
\end{minipage} & \begin{minipage}[b]{\linewidth}\raggedright
HIPÓTESIS
\end{minipage} \\
\midrule\noalign{}
\endhead
\bottomrule\noalign{}
\endlastfoot
\textbf{PROBLEMA GENERAL} & \textbf{OBJETIVO GENERAL} &
\textbf{HIPÓTESIS GENERAL} \\
¿Cuál es la incidencia de la gestión pública en el desarrollo sostenible
del distrito de Santo Tomás, Chumbivilcas, durante el periodo 2020-2023?
& Determinar la incidencia de la gestión pública en el desarrollo
sostenible del distrito de Santo Tomás, Chumbivilcas, durante el periodo
2020-2023. & La gestión pública incide significativamente en el
desarrollo sostenible del distrito de Santo Tomás, Chumbivilcas, durante
el periodo 2020-2023. \\
\textbf{PROBLEMAS ESPECÍFICOS} & \textbf{OBJETIVOS ESPECÍFICOS} &
\textbf{HIPÓTESIS ESPECÍFICAS} \\
PE1: ¿Cómo incide la planificación estratégica en la dimensión ambiental
del desarrollo sostenible en el distrito? & OE1: Analizar la incidencia
de la planificación estratégica en la dimensión ambiental del desarrollo
sostenible en el distrito. & HE1: La planificación estratégica incide
positivamente en la dimensión ambiental del desarrollo sostenible del
distrito. \\
PE2: ¿Cómo incide la gestión presupuestaria en la dimensión económica
del desarrollo sostenible en el distrito? & OE2: Evaluar la incidencia
de la gestión presupuestaria en la dimensión económica del desarrollo
sostenible en el distrito. & HE2: La gestión presupuestaria incide
significativamente en la dimensión económica del desarrollo sostenible
del distrito. \\
PE3: ¿Cómo incide la participación ciudadana en la dimensión social del
desarrollo sostenible en el distrito? & OE3: Establecer la incidencia de
la participación ciudadana en la dimensión social del desarrollo
sostenible en el distrito. & HE3: La participación ciudadana incide
positivamente en la dimensión social del desarrollo sostenible del
distrito. \\
\end{longtable}

\subsection{Variables e Indicadores}\label{variables-e-indicadores-5}

\begin{longtable}[]{@{}
  >{\raggedright\arraybackslash}p{(\linewidth - 8\tabcolsep) * \real{0.2000}}
  >{\raggedright\arraybackslash}p{(\linewidth - 8\tabcolsep) * \real{0.1200}}
  >{\raggedright\arraybackslash}p{(\linewidth - 8\tabcolsep) * \real{0.2600}}
  >{\raggedright\arraybackslash}p{(\linewidth - 8\tabcolsep) * \real{0.2600}}
  >{\raggedright\arraybackslash}p{(\linewidth - 8\tabcolsep) * \real{0.1600}}@{}}
\toprule\noalign{}
\begin{minipage}[b]{\linewidth}\raggedright
VARIABLE
\end{minipage} & \begin{minipage}[b]{\linewidth}\raggedright
TIPO
\end{minipage} & \begin{minipage}[b]{\linewidth}\raggedright
DIMENSIONES
\end{minipage} & \begin{minipage}[b]{\linewidth}\raggedright
INDICADORES
\end{minipage} & \begin{minipage}[b]{\linewidth}\raggedright
ESCALA
\end{minipage} \\
\midrule\noalign{}
\endhead
\bottomrule\noalign{}
\endlastfoot
\textbf{Gestión Pública} & Independiente & Planificación Estratégica & -
Existencia de Plan de Desarrollo Concertado- Alineamiento con ODS-
Cumplimiento de metas estratégicas- Instrumentos de planificación
actualizados & Ordinal/Nominal \\
& & Gestión Presupuestaria & - Porcentaje de ejecución presupuestal-
Eficiencia en el uso de recursos- Transparencia presupuestaria-
Inversión en proyectos sostenibles & Razón (Porcentaje) \\
& & Participación Ciudadana & - Mecanismos de participación
implementados- Porcentaje de participación en presupuesto participativo-
Satisfacción ciudadana con la gestión- Número de organizaciones sociales
activas & Razón/Ordinal \\
& & Gestión Institucional & - Modernización administrativa- Capacitación
de servidores públicos- Sistemas de información implementados-
Cumplimiento normativo & Ordinal \\
\textbf{Desarrollo Sostenible} & Dependiente & Dimensión Ambiental & -
Calidad del agua y aire- Cobertura de áreas verdes- Gestión de residuos
sólidos- Conservación de recursos naturales- Proyectos ambientales
ejecutados & Razón/Ordinal \\
& & Dimensión Económica & - Ingreso per cápita- Tasa de empleo- Número
de MYPE formalizadas- Diversificación productiva- Acceso a servicios
financieros & Razón (Soles/Porcentaje) \\
& & Dimensión Social & - Índice de Desarrollo Humano- Acceso a servicios
básicos (agua, luz, saneamiento)- Tasa de pobreza- Cobertura en
educación y salud- Equidad de género & Razón/Índice \\
\end{longtable}

\subsection{Metodología}\label{metodologuxeda-5}

\begin{longtable}[]{@{}
  >{\raggedright\arraybackslash}p{(\linewidth - 2\tabcolsep) * \real{0.4091}}
  >{\raggedright\arraybackslash}p{(\linewidth - 2\tabcolsep) * \real{0.5909}}@{}}
\toprule\noalign{}
\begin{minipage}[b]{\linewidth}\raggedright
ASPECTO
\end{minipage} & \begin{minipage}[b]{\linewidth}\raggedright
DESCRIPCIÓN
\end{minipage} \\
\midrule\noalign{}
\endhead
\bottomrule\noalign{}
\endlastfoot
\textbf{Tipo de Investigación} & - Aplicada- Explicativa- Transversal \\
\textbf{Diseño} & No experimental, transversal correlacional-causal \\
\textbf{Nivel} & Explicativo-correlacional \\
\textbf{Enfoque} & Mixto (cuantitativo-cualitativo) \\
\textbf{Método} & Hipotético-deductivo, análisis estadístico, análisis
documental \\
\textbf{Población} & - Funcionarios de la Municipalidad Distrital de
Santo Tomás (N ≈ 50)- Población del distrito (para encuestas ciudadanas)
(N ≈ 8,000 habitantes) \\
\textbf{Muestra} & - Funcionarios: Censo (50)- Ciudadanos: Muestreo
probabilístico estratificado (n ≈ 367, con IC 95\%, e=5\%) \\
\textbf{Técnicas de Recolección} & - Encuesta- Entrevista
semi-estructurada- Análisis documental- Observación \\
\textbf{Instrumentos} & - Cuestionarios sobre gestión pública (escala
Likert)- Cuestionario sobre desarrollo sostenible- Guía de entrevista-
Ficha de análisis documental- Lista de cotejo \\
\textbf{Fuentes de Datos} & - Municipalidad Distrital de Santo Tomás-
Portal de Transparencia (MEF)- INEI (Censo, ENAHO)- MINAM (indicadores
ambientales)- SINIA (Sistema Nacional de Información Ambiental) \\
\textbf{Técnicas de Análisis} & - Estadística descriptiva (frecuencias,
medias, desviaciones)- Análisis de correlación (Pearson/Spearman)-
Regresión lineal múltiple- Análisis factorial- Chi-cuadrado- Análisis de
contenido (datos cualitativos) \\
\textbf{Software Estadístico} & SPSS, Excel, Atlas.ti (análisis
cualitativo) \\
\end{longtable}

\subsection{Marco Teórico (Conceptos
Clave)}\label{marco-teuxf3rico-conceptos-clave-5}

\textbf{Gestión Pública:} Conjunto de procesos y acciones mediante los
cuales las entidades del sector público planifican, organizan, dirigen y
controlan los recursos y actividades para el logro de objetivos
institucionales y el bienestar ciudadano.

\textbf{Desarrollo Sostenible:} Desarrollo que satisface las necesidades
del presente sin comprometer la capacidad de las futuras generaciones
para satisfacer sus propias necesidades, integrando dimensiones
económicas, sociales y ambientales (Brundtland, 1987).

\textbf{Objetivos de Desarrollo Sostenible (ODS):} Conjunto de 17
objetivos globales establecidos por las Naciones Unidas en 2015 para
erradicar la pobreza, proteger el planeta y asegurar prosperidad para
todos.

\subsection{Validación de
Instrumentos}\label{validaciuxf3n-de-instrumentos}

\begin{itemize}
\tightlist
\item
  \textbf{Validez de contenido:} Juicio de expertos (3-5 expertos en
  gestión pública y desarrollo sostenible)
\item
  \textbf{Confiabilidad:} Alfa de Cronbach (α ≥ 0.70 aceptable)
\item
  \textbf{Prueba piloto:} Aplicación a muestra pequeña (n=30) para
  ajustar instrumentos
\end{itemize}

\subsection{Referencias Principales}\label{referencias-principales-5}

\begin{itemize}
\tightlist
\item
  CEPLAN. (2019). \emph{Guía para el planeamiento institucional}. Lima:
  Centro Nacional de Planeamiento Estratégico.
\item
  Naciones Unidas. (2015). \emph{Transformar nuestro mundo: la Agenda
  2030 para el Desarrollo Sostenible}. Nueva York: ONU.
\item
  Secretaría de Gestión Pública. (2021). \emph{Política Nacional de
  Modernización de la Gestión Pública}. Lima: PCM.
\end{itemize}

\section{Matriz 7: Presupuesto Participativo y Ejecución
Presupuestaria}\label{matriz-7-presupuesto-participativo-y-ejecuciuxf3n-presupuestaria}

\subsection{Título}\label{tuxedtulo-6}

\textbf{El Presupuesto Participativo y su Relación con la Ejecución
Presupuestaria en la Municipalidad Distrital de San Jerónimo, Cusco,
Periodo 2024}

\subsection{Datos Generales}\label{datos-generales-6}

\begin{itemize}
\tightlist
\item
  \textbf{Área de investigación:} Administración Pública, Finanzas
  Públicas
\item
  \textbf{Línea de investigación:} Administración Pública, Gestión Local
\item
  \textbf{Tipo de investigación:} Aplicada, Descriptiva-Correlacional,
  Transversal
\item
  \textbf{Enfoque:} Cuantitativo
\end{itemize}

\begin{longtable}[]{@{}
  >{\raggedright\arraybackslash}p{(\linewidth - 4\tabcolsep) * \real{0.3333}}
  >{\raggedright\arraybackslash}p{(\linewidth - 4\tabcolsep) * \real{0.3333}}
  >{\raggedright\arraybackslash}p{(\linewidth - 4\tabcolsep) * \real{0.3333}}@{}}
\toprule\noalign{}
\begin{minipage}[b]{\linewidth}\raggedright
PROBLEMAS
\end{minipage} & \begin{minipage}[b]{\linewidth}\raggedright
OBJETIVOS
\end{minipage} & \begin{minipage}[b]{\linewidth}\raggedright
HIPÓTESIS
\end{minipage} \\
\midrule\noalign{}
\endhead
\bottomrule\noalign{}
\endlastfoot
\textbf{PROBLEMA GENERAL} & \textbf{OBJETIVO GENERAL} &
\textbf{HIPÓTESIS GENERAL} \\
¿Cuál es la relación entre el presupuesto participativo y la ejecución
presupuestaria en la Municipalidad Distrital de San Jerónimo, Cusco, en
el año 2024? & Determinar la relación entre el presupuesto participativo
y la ejecución presupuestaria en la Municipalidad Distrital de San
Jerónimo, Cusco, en el año 2024. & Existe una relación positiva y
significativa entre el presupuesto participativo y la ejecución
presupuestaria en la Municipalidad Distrital de San Jerónimo, Cusco, en
el año 2024. \\
\textbf{PROBLEMAS ESPECÍFICOS} & \textbf{OBJETIVOS ESPECÍFICOS} &
\textbf{HIPÓTESIS ESPECÍFICAS} \\
PE1: ¿Cuál es la relación entre la convocatoria del presupuesto
participativo y la ejecución del gasto de inversión? & OE1: Establecer
la relación entre la convocatoria del presupuesto participativo y la
ejecución del gasto de inversión. & HE1: Existe una relación positiva y
significativa entre la convocatoria del presupuesto participativo y la
ejecución del gasto de inversión. \\
PE2: ¿Cuál es la relación entre la identificación de proyectos
participativos y el cumplimiento de metas presupuestarias? & OE2:
Analizar la relación entre la identificación de proyectos participativos
y el cumplimiento de metas presupuestarias. & HE2: Existe una relación
positiva y significativa entre la identificación de proyectos
participativos y el cumplimiento de metas presupuestarias. \\
PE3: ¿Cuál es la relación entre el seguimiento y rendición de cuentas
del presupuesto participativo y la transparencia presupuestaria? & OE3:
Evaluar la relación entre el seguimiento y rendición de cuentas del
presupuesto participativo y la transparencia presupuestaria. & HE3:
Existe una relación positiva y significativa entre el seguimiento y
rendición de cuentas del presupuesto participativo y la transparencia
presupuestaria. \\
\end{longtable}

\subsection{Variables e Indicadores}\label{variables-e-indicadores-6}

\begin{longtable}[]{@{}
  >{\raggedright\arraybackslash}p{(\linewidth - 8\tabcolsep) * \real{0.2000}}
  >{\raggedright\arraybackslash}p{(\linewidth - 8\tabcolsep) * \real{0.1200}}
  >{\raggedright\arraybackslash}p{(\linewidth - 8\tabcolsep) * \real{0.2600}}
  >{\raggedright\arraybackslash}p{(\linewidth - 8\tabcolsep) * \real{0.2600}}
  >{\raggedright\arraybackslash}p{(\linewidth - 8\tabcolsep) * \real{0.1600}}@{}}
\toprule\noalign{}
\begin{minipage}[b]{\linewidth}\raggedright
VARIABLE
\end{minipage} & \begin{minipage}[b]{\linewidth}\raggedright
TIPO
\end{minipage} & \begin{minipage}[b]{\linewidth}\raggedright
DIMENSIONES
\end{minipage} & \begin{minipage}[b]{\linewidth}\raggedright
INDICADORES
\end{minipage} & \begin{minipage}[b]{\linewidth}\raggedright
ESCALA
\end{minipage} \\
\midrule\noalign{}
\endhead
\bottomrule\noalign{}
\endlastfoot
\textbf{Presupuesto Participativo} & Independiente & Convocatoria & -
Número de talleres de convocatoria- Porcentaje de participación
ciudadana- Difusión de la convocatoria (medios utilizados)- Inclusión de
organizaciones sociales & Razón/Ordinal \\
& & Identificación de Proyectos & - Número de proyectos propuestos-
Pertinencia de proyectos (alineados con PDC)- Priorización democrática-
Diversidad sectorial de proyectos & Razón/Ordinal \\
& & Seguimiento y Rendición de Cuentas & - Frecuencia de rendiciones de
cuentas- Mecanismos de vigilancia ciudadana- Cumplimiento de acuerdos-
Satisfacción con el proceso & Razón/Ordinal \\
\textbf{Ejecución Presupuestaria} & Dependiente & Gasto de Inversión & -
Porcentaje de ejecución en proyectos de inversión- Monto devengado/PIM-
Proyectos culminados- Eficiencia del gasto & Razón (Porcentaje/Soles) \\
& & Cumplimiento de Metas & - Porcentaje de metas físicas alcanzadas-
Porcentaje de metas financieras alcanzadas- Oportunidad en la ejecución-
Calidad de obras ejecutadas & Razón (Porcentaje) \\
& & Transparencia Presupuestaria & - Publicación de información
presupuestaria- Accesibilidad de la información- Cumplimiento de
normativa de transparencia- Índice de transparencia municipal &
Ordinal/Razón \\
\end{longtable}

\subsection{Metodología}\label{metodologuxeda-6}

\begin{longtable}[]{@{}
  >{\raggedright\arraybackslash}p{(\linewidth - 2\tabcolsep) * \real{0.4091}}
  >{\raggedright\arraybackslash}p{(\linewidth - 2\tabcolsep) * \real{0.5909}}@{}}
\toprule\noalign{}
\begin{minipage}[b]{\linewidth}\raggedright
ASPECTO
\end{minipage} & \begin{minipage}[b]{\linewidth}\raggedright
DESCRIPCIÓN
\end{minipage} \\
\midrule\noalign{}
\endhead
\bottomrule\noalign{}
\endlastfoot
\textbf{Tipo de Investigación} & - Aplicada- Descriptiva-Correlacional-
Transversal \\
\textbf{Diseño} & No experimental, transversal correlacional \\
\textbf{Nivel} & Descriptivo-correlacional \\
\textbf{Enfoque} & Cuantitativo \\
\textbf{Método} & Descriptivo, correlacional \\
\textbf{Población} & - Funcionarios municipales (N ≈ 80)- Agentes
participantes del presupuesto participativo 2024 (N ≈ 150) \\
\textbf{Muestra} & - Funcionarios: Censo o muestra estratificada (n ≈
66)- Agentes participantes: Censo o muestra aleatoria simple (n ≈
108) \\
\textbf{Técnicas de Recolección} & - Encuesta- Análisis documental-
Revisión de registros administrativos \\
\textbf{Instrumentos} & - Cuestionario sobre presupuesto participativo
(escala Likert)- Cuestionario sobre ejecución presupuestaria- Ficha de
análisis documental \\
\textbf{Fuentes de Datos} & - Municipalidad Distrital de San Jerónimo-
Portal de Transparencia (MEF)- SIAF-GL- Actas de presupuesto
participativo- Informes de gestión \\
\textbf{Técnicas de Análisis} & - Estadística descriptiva (frecuencias,
tablas, gráficos)- Coeficiente de correlación de Pearson- Coeficiente de
correlación de Spearman- Regresión lineal simple- Chi-cuadrado \\
\textbf{Software Estadístico} & SPSS, Excel, Stata \\
\end{longtable}

\subsection{Marco Teórico (Conceptos
Clave)}\label{marco-teuxf3rico-conceptos-clave-6}

\textbf{Presupuesto Participativo:} Mecanismo de democracia directa y
participativa mediante el cual la población decide sobre la asignación
de recursos públicos municipales, particularmente en inversiones y
proyectos de desarrollo local.

\textbf{Ejecución Presupuestaria:} Proceso de utilización de los
recursos financieros asignados en el presupuesto público para la
adquisición de bienes, servicios y ejecución de proyectos, comprendiendo
las fases de compromiso, devengado y pago.

\textbf{Marco Legal:} Ley N° 28056 (Ley Marco del Presupuesto
Participativo), Decreto Supremo N° 097-2009-EF (Reglamento), Ley N°
27972 (Ley Orgánica de Municipalidades).

\subsection{Validación de
Instrumentos}\label{validaciuxf3n-de-instrumentos-1}

\begin{itemize}
\tightlist
\item
  \textbf{Validez de contenido:} Juicio de 3 expertos en gestión pública
\item
  \textbf{Confiabilidad:} Alfa de Cronbach ≥ 0.70
\item
  \textbf{Prueba piloto:} n=20 (no incluidos en la muestra final)
\end{itemize}

\subsection{Esquema de Correlación
Esperada}\label{esquema-de-correlaciuxf3n-esperada}

Se espera encontrar:

\begin{itemize}
\tightlist
\item
  Correlación positiva moderada-alta (r = 0.50-0.75) entre presupuesto
  participativo y ejecución presupuestaria
\item
  Significancia estadística (p \textless{} 0.05)
\end{itemize}

\subsection{Referencias Principales}\label{referencias-principales-6}

\begin{itemize}
\tightlist
\item
  Congreso de la República. (2003). \emph{Ley N° 28056: Ley Marco del
  Presupuesto Participativo}. Lima: Congreso del Perú.
\item
  Goldfrank, B. (2006). \emph{Los procesos de ``presupuesto
  participativo'' en América Latina: éxito, fracaso y cambio}. Revista
  de Ciencia Política, 26(2), 03-28.
\item
  Cabannes, Y. (2004). \emph{Participatory budgeting: a significant
  contribution to participatory democracy}. Environment and
  Urbanization, 16(1), 27-46.
\end{itemize}

\section{Matriz 8: Políticas Públicas y Reducción de Pobreza
Rural}\label{matriz-8-poluxedticas-puxfablicas-y-reducciuxf3n-de-pobreza-rural}

\subsection{Título}\label{tuxedtulo-7}

\textbf{Impacto de las Políticas Públicas en la Reducción de la Pobreza
Rural en la Región de Ayacucho, Periodo 2010-2023}

\subsection{Datos Generales}\label{datos-generales-7}

\begin{itemize}
\tightlist
\item
  \textbf{Área de investigación:} Economía del Desarrollo, Políticas
  Públicas
\item
  \textbf{Línea de investigación:} Economía, Ciencias Sociales
  Interdisciplinarias
\item
  \textbf{Tipo de investigación:} Aplicada, Explicativa, Longitudinal
\item
  \textbf{Enfoque:} Mixto (predominantemente cuantitativo)
\end{itemize}

\begin{longtable}[]{@{}
  >{\raggedright\arraybackslash}p{(\linewidth - 4\tabcolsep) * \real{0.3333}}
  >{\raggedright\arraybackslash}p{(\linewidth - 4\tabcolsep) * \real{0.3333}}
  >{\raggedright\arraybackslash}p{(\linewidth - 4\tabcolsep) * \real{0.3333}}@{}}
\toprule\noalign{}
\begin{minipage}[b]{\linewidth}\raggedright
PROBLEMAS
\end{minipage} & \begin{minipage}[b]{\linewidth}\raggedright
OBJETIVOS
\end{minipage} & \begin{minipage}[b]{\linewidth}\raggedright
HIPÓTESIS
\end{minipage} \\
\midrule\noalign{}
\endhead
\bottomrule\noalign{}
\endlastfoot
\textbf{PROBLEMA GENERAL} & \textbf{OBJETIVO GENERAL} &
\textbf{HIPÓTESIS GENERAL} \\
¿Cuál es el impacto de las políticas públicas en la reducción de la
pobreza rural en la región de Ayacucho durante el periodo 2010-2023? &
Evaluar el impacto de las políticas públicas en la reducción de la
pobreza rural en la región de Ayacucho durante el periodo 2010-2023. &
Las políticas públicas implementadas en Ayacucho han tenido un impacto
significativo y positivo en la reducción de la pobreza rural durante el
periodo 2010-2023. \\
\textbf{PROBLEMAS ESPECÍFICOS} & \textbf{OBJETIVOS ESPECÍFICOS} &
\textbf{HIPÓTESIS ESPECÍFICAS} \\
PE1: ¿Cuál es el impacto del programa Juntos en la reducción de la
pobreza rural en Ayacucho? & OE1: Determinar el impacto del programa
Juntos en la reducción de la pobreza rural en Ayacucho. & HE1: El
programa Juntos ha tenido un impacto significativo y positivo en la
reducción de la pobreza rural en Ayacucho. \\
PE2: ¿Cuál es el efecto de los programas productivos (Haku Wiñay, Mi
Riego) en el ingreso de las familias rurales de Ayacucho? & OE2:
Analizar el efecto de los programas productivos en el ingreso de las
familias rurales de Ayacucho. & HE2: Los programas productivos han
incrementado significativamente los ingresos de las familias rurales en
Ayacucho. \\
PE3: ¿Cuáles son las características de los hogares rurales que han
salido de la pobreza en Ayacucho? & OE3: Identificar las características
de los hogares rurales que han salido de la pobreza en Ayacucho. & HE3:
Los hogares que salieron de la pobreza presentan mayor acceso a
programas sociales, educación y activos productivos. \\
\end{longtable}

\subsection{Variables e Indicadores}\label{variables-e-indicadores-7}

\begin{longtable}[]{@{}
  >{\raggedright\arraybackslash}p{(\linewidth - 8\tabcolsep) * \real{0.2000}}
  >{\raggedright\arraybackslash}p{(\linewidth - 8\tabcolsep) * \real{0.1200}}
  >{\raggedright\arraybackslash}p{(\linewidth - 8\tabcolsep) * \real{0.2600}}
  >{\raggedright\arraybackslash}p{(\linewidth - 8\tabcolsep) * \real{0.2600}}
  >{\raggedright\arraybackslash}p{(\linewidth - 8\tabcolsep) * \real{0.1600}}@{}}
\toprule\noalign{}
\begin{minipage}[b]{\linewidth}\raggedright
VARIABLE
\end{minipage} & \begin{minipage}[b]{\linewidth}\raggedright
TIPO
\end{minipage} & \begin{minipage}[b]{\linewidth}\raggedright
DIMENSIONES
\end{minipage} & \begin{minipage}[b]{\linewidth}\raggedright
INDICADORES
\end{minipage} & \begin{minipage}[b]{\linewidth}\raggedright
ESCALA
\end{minipage} \\
\midrule\noalign{}
\endhead
\bottomrule\noalign{}
\endlastfoot
\textbf{Políticas Públicas} & Independiente & Programas Sociales & -
Cobertura del programa Juntos (\%)- Monto de transferencias (soles)-
Años de permanencia en el programa- Cobertura de Qali Warma, Pensión 65
& Razón \\
& & Programas Productivos & - Participación en Haku Wiñay- Acceso a Mi
Riego- Créditos del programa Agro Rural- Asistencia técnica recibida &
Nominal/Razón \\
& & Infraestructura Social & - Acceso a agua potable- Acceso a
electricidad- Acceso a servicios de salud- Infraestructura educativa &
Nominal/Razón \\
\textbf{Pobreza Rural} & Dependiente & Pobreza Monetaria & - Tasa de
pobreza rural (\%)- Tasa de pobreza extrema rural (\%)- Brecha de
pobreza- Severidad de la pobreza & Razón (Porcentaje) \\
& & Ingreso & - Ingreso per cápita mensual- Gasto per cápita mensual-
Diversificación de fuentes de ingreso & Razón (Soles) \\
& & Condiciones de Vida & - Índice de Necesidades Básicas Insatisfechas
(NBI)- Acceso a servicios básicos- Calidad de vivienda- Hacinamiento &
Razón/Ordinal \\
& & Capital Humano & - Años de educación del jefe de hogar- Desnutrición
crónica infantil- Acceso a seguro de salud & Razón/Ordinal \\
\end{longtable}

\subsection{Metodología}\label{metodologuxeda-7}

\begin{longtable}[]{@{}
  >{\raggedright\arraybackslash}p{(\linewidth - 2\tabcolsep) * \real{0.4091}}
  >{\raggedright\arraybackslash}p{(\linewidth - 2\tabcolsep) * \real{0.5909}}@{}}
\toprule\noalign{}
\begin{minipage}[b]{\linewidth}\raggedright
ASPECTO
\end{minipage} & \begin{minipage}[b]{\linewidth}\raggedright
DESCRIPCIÓN
\end{minipage} \\
\midrule\noalign{}
\endhead
\bottomrule\noalign{}
\endlastfoot
\textbf{Tipo de Investigación} & - Aplicada- Explicativa-
Longitudinal \\
\textbf{Diseño} & Cuasi-experimental, longitudinal (evaluación de
impacto) \\
\textbf{Nivel} & Explicativo \\
\textbf{Enfoque} & Mixto (predominantemente cuantitativo) \\
\textbf{Método} & Evaluación de impacto, análisis econométrico, estudios
de caso \\
\textbf{Población} & Hogares rurales de la región de Ayacucho
(aproximadamente 150,000 hogares) \\
\textbf{Muestra} & Muestra de la ENAHO: Hogares rurales de Ayacucho
(panel 2010-2023) ≈ 800-1,200 hogares/año \\
\textbf{Técnicas de Recolección} & - Análisis de datos secundarios
(ENAHO)- Entrevistas semi-estructuradas (cualitativo)- Grupos focales en
comunidades seleccionadas \\
\textbf{Instrumentos} & - Base de datos ENAHO- Guía de entrevista- Guía
de grupo focal \\
\textbf{Fuentes de Datos} & - INEI (ENAHO panel 2010-2023)- MIDIS
(Padrón de beneficiarios de programas sociales)- MEF (Inversión
pública)- SIAF (Ejecución de programas)- Censos nacionales \\
\textbf{Técnicas de Análisis} & - \textbf{Cuantitativo:} • Propensity
Score Matching (PSM) • Diferencias en Diferencias (DID) • Regresión con
discontinuidad (RDD) • Modelos de panel de datos • Análisis de
supervivencia (análisis de duración)- \textbf{Cualitativo:} • Análisis
de contenido • Triangulación de datos \\
\textbf{Software Estadístico} & Stata, R, SPSS, Atlas.ti
(cualitativo) \\
\end{longtable}

\subsection{Marco Teórico (Conceptos
Clave)}\label{marco-teuxf3rico-conceptos-clave-7}

\textbf{Pobreza Rural:} Condición de carencia de recursos económicos y
acceso limitado a oportunidades y servicios básicos que afecta a hogares
en zonas rurales, caracterizada por bajos ingresos, vulnerabilidad y
exclusión.

\textbf{Programa Juntos:} Programa de transferencias monetarias
condicionadas del gobierno peruano que busca reducir la pobreza y
pobreza extrema mediante incentivos económicos a hogares en condición de
pobreza, condicionados al cumplimiento de compromisos en salud,
nutrición y educación.

\textbf{Haku Wiñay:} Programa del FONCODES orientado a mejorar la
seguridad alimentaria y los ingresos económicos de hogares rurales en
pobreza mediante asistencia técnica, transferencia de activos
productivos y fortalecimiento de capacidades.

\subsection{Estrategia de Evaluación de
Impacto}\label{estrategia-de-evaluaciuxf3n-de-impacto}

\textbf{1. Propensity Score Matching (PSM):}

\begin{itemize}
\tightlist
\item
  Comparación entre beneficiarios y no beneficiarios con características
  similares
\item
  Control de sesgo de selección
\end{itemize}

\textbf{2. Diferencias en Diferencias (DID):}

\begin{itemize}
\tightlist
\item
  Comparación antes-después entre grupo tratamiento y control
\item
  Control de efectos temporales
\end{itemize}

\textbf{3. Regresión con Discontinuidad (si aplica):}

\begin{itemize}
\tightlist
\item
  Aprovecha umbrales de elegibilidad de programas
\end{itemize}

\textbf{Indicador de Impacto Principal:}

\begin{itemize}
\tightlist
\item
  Average Treatment Effect on the Treated (ATT)
\end{itemize}

\subsection{Componente Cualitativo}\label{componente-cualitativo}

\textbf{Objetivo:} Comprender mecanismos y experiencias de salida de la
pobreza

\textbf{Técnicas:}

\begin{itemize}
\tightlist
\item
  15-20 entrevistas en profundidad a hogares que salieron de la pobreza
\item
  4-6 grupos focales en comunidades rurales
\item
  Historias de vida
\end{itemize}

\subsection{Referencias Principales}\label{referencias-principales-7}

\begin{itemize}
\tightlist
\item
  Escobal, J., \& Ponce, C. (2012). \emph{Evaluación de impacto del
  Programa Juntos}. GRADE-MIDIS.
\item
  Trivelli, C., \& Vargas, S. (2014). \emph{Entre el discurso y la
  acción: Desafíos, decisiones y dilemas en el marco de la reforma de
  los programas sociales en el Perú}. IEP.
\item
  MIDIS. (2023). \emph{Informe de evaluación de programas sociales}.
  Lima: Ministerio de Desarrollo e Inclusión Social.
\end{itemize}

\section{Matriz 9: Modernización de la Gestión Pública y Gasto
Público}\label{matriz-9-modernizaciuxf3n-de-la-gestiuxf3n-puxfablica-y-gasto-puxfablico}

\subsection{Título}\label{tuxedtulo-8}

\textbf{Modernización de la Gestión Pública y su Relación con el Gasto
Público en la Municipalidad Distrital de Pillco Marca, Huánuco, Periodo
2021}

\subsection{Datos Generales}\label{datos-generales-8}

\begin{itemize}
\tightlist
\item
  \textbf{Área de investigación:} Administración Pública
\item
  \textbf{Línea de investigación:} Administración Pública, Gestión Local
\item
  \textbf{Tipo de investigación:} Aplicada, Descriptiva-Correlacional,
  Transversal
\item
  \textbf{Enfoque:} Cuantitativo
\end{itemize}

\begin{longtable}[]{@{}
  >{\raggedright\arraybackslash}p{(\linewidth - 4\tabcolsep) * \real{0.3333}}
  >{\raggedright\arraybackslash}p{(\linewidth - 4\tabcolsep) * \real{0.3333}}
  >{\raggedright\arraybackslash}p{(\linewidth - 4\tabcolsep) * \real{0.3333}}@{}}
\toprule\noalign{}
\begin{minipage}[b]{\linewidth}\raggedright
PROBLEMAS
\end{minipage} & \begin{minipage}[b]{\linewidth}\raggedright
OBJETIVOS
\end{minipage} & \begin{minipage}[b]{\linewidth}\raggedright
HIPÓTESIS
\end{minipage} \\
\midrule\noalign{}
\endhead
\bottomrule\noalign{}
\endlastfoot
\textbf{PROBLEMA GENERAL} & \textbf{OBJETIVO GENERAL} &
\textbf{HIPÓTESIS GENERAL} \\
¿Cuál es la relación entre la modernización de la gestión pública y el
gasto público en la Municipalidad Distrital de Pillco Marca, Huánuco, en
el año 2021? & Determinar la relación entre la modernización de la
gestión pública y el gasto público en la Municipalidad Distrital de
Pillco Marca, Huánuco, en el año 2021. & Existe una relación positiva y
significativa entre la modernización de la gestión pública y la
eficiencia del gasto público en la Municipalidad Distrital de Pillco
Marca, Huánuco, en el año 2021. \\
\textbf{PROBLEMAS ESPECÍFICOS} & \textbf{OBJETIVOS ESPECÍFICOS} &
\textbf{HIPÓTESIS ESPECÍFICAS} \\
PE1: ¿Cuál es la relación entre la gestión por procesos y la ejecución
del gasto corriente? & OE1: Establecer la relación entre la gestión por
procesos y la ejecución del gasto corriente. & HE1: Existe una relación
positiva y significativa entre la gestión por procesos y la eficiencia
en la ejecución del gasto corriente. \\
PE2: ¿Cuál es la relación entre el gobierno electrónico y la
transparencia del gasto público? & OE2: Analizar la relación entre el
gobierno electrónico y la transparencia del gasto público. & HE2: Existe
una relación positiva y significativa entre el gobierno electrónico y la
transparencia del gasto público. \\
PE3: ¿Cuál es la relación entre la gestión de recursos humanos y la
eficiencia del gasto en personal? & OE3: Evaluar la relación entre la
gestión de recursos humanos y la eficiencia del gasto en personal. &
HE3: Existe una relación positiva y significativa entre la gestión de
recursos humanos por competencias y la eficiencia del gasto en
personal. \\
\end{longtable}

\subsection{Variables e Indicadores}\label{variables-e-indicadores-8}

\begin{longtable}[]{@{}
  >{\raggedright\arraybackslash}p{(\linewidth - 8\tabcolsep) * \real{0.2000}}
  >{\raggedright\arraybackslash}p{(\linewidth - 8\tabcolsep) * \real{0.1200}}
  >{\raggedright\arraybackslash}p{(\linewidth - 8\tabcolsep) * \real{0.2600}}
  >{\raggedright\arraybackslash}p{(\linewidth - 8\tabcolsep) * \real{0.2600}}
  >{\raggedright\arraybackslash}p{(\linewidth - 8\tabcolsep) * \real{0.1600}}@{}}
\toprule\noalign{}
\begin{minipage}[b]{\linewidth}\raggedright
VARIABLE
\end{minipage} & \begin{minipage}[b]{\linewidth}\raggedright
TIPO
\end{minipage} & \begin{minipage}[b]{\linewidth}\raggedright
DIMENSIONES
\end{minipage} & \begin{minipage}[b]{\linewidth}\raggedright
INDICADORES
\end{minipage} & \begin{minipage}[b]{\linewidth}\raggedright
ESCALA
\end{minipage} \\
\midrule\noalign{}
\endhead
\bottomrule\noalign{}
\endlastfoot
\textbf{Modernización de la Gestión Pública} & Independiente & Gestión
por Procesos & - Procesos identificados y documentados- Procesos
optimizados- Manual de procesos actualizado- Indicadores de desempeño
por proceso & Ordinal/Razón \\
& & Gobierno Electrónico & - Servicios digitales implementados- Trámites
en línea disponibles- Portal web actualizado- Uso de sistemas de
información & Ordinal/Razón \\
& & Servicio Civil Meritocrático & - Personal por meritocracia (\%)-
Capacitación del personal (horas/año)- Evaluación de desempeño
implementada- Gestión por competencias & Razón/Ordinal \\
& & Planeamiento Estratégico & - POI alineado con Plan de Desarrollo-
Seguimiento de metas estratégicas- Sistema de monitoreo y evaluación-
Cumplimiento de objetivos estratégicos & Ordinal/Razón \\
\textbf{Gasto Público} & Dependiente & Ejecución del Gasto & -
Porcentaje de ejecución presupuestal total- Ejecución del gasto
corriente- Ejecución del gasto de capital- Oportunidad del gasto & Razón
(Porcentaje) \\
& & Eficiencia del Gasto & - Costo-efectividad de servicios-
Productividad por empleado- Tiempo de atención al ciudadano- Calidad del
gasto & Razón \\
& & Transparencia y Rendición de Cuentas & - Publicación de información
presupuestaria- Cumplimiento de normativa de transparencia- Rendiciones
de cuentas realizadas- Accesibilidad de información & Ordinal \\
\end{longtable}

\subsection{Metodología}\label{metodologuxeda-8}

\begin{longtable}[]{@{}
  >{\raggedright\arraybackslash}p{(\linewidth - 2\tabcolsep) * \real{0.4091}}
  >{\raggedright\arraybackslash}p{(\linewidth - 2\tabcolsep) * \real{0.5909}}@{}}
\toprule\noalign{}
\begin{minipage}[b]{\linewidth}\raggedright
ASPECTO
\end{minipage} & \begin{minipage}[b]{\linewidth}\raggedright
DESCRIPCIÓN
\end{minipage} \\
\midrule\noalign{}
\endhead
\bottomrule\noalign{}
\endlastfoot
\textbf{Tipo de Investigación} & - Aplicada- Descriptiva-Correlacional-
Transversal \\
\textbf{Diseño} & No experimental, transversal correlacional \\
\textbf{Nivel} & Descriptivo-correlacional \\
\textbf{Enfoque} & Cuantitativo \\
\textbf{Método} & Descriptivo, correlacional \\
\textbf{Población} & - Funcionarios municipales (N ≈ 120)- Ciudadanos
usuarios de servicios (para percepción) \\
\textbf{Muestra} & - Funcionarios: Muestra estratificada (n ≈ 92, IC
95\%, e=5\%)- Ciudadanos: Muestra aleatoria simple (n ≈ 384) \\
\textbf{Técnicas de Recolección} & - Encuesta- Análisis documental-
Revisión de registros administrativos \\
\textbf{Instrumentos} & - Cuestionario sobre modernización de gestión
pública (escala Likert)- Cuestionario sobre gasto público- Ficha de
análisis documental \\
\textbf{Fuentes de Datos} & - Municipalidad Distrital de Pillco Marca-
Portal de Transparencia (MEF)- SIAF-GL- SERVIR (datos de gestión de
recursos humanos)- Informes de gestión \\
\textbf{Técnicas de Análisis} & - Estadística descriptiva- Coeficiente
de correlación de Pearson- Coeficiente de correlación de Spearman-
Regresión lineal múltiple- Tablas de contingencia y Chi-cuadrado \\
\textbf{Software Estadístico} & SPSS, Excel, Stata \\
\end{longtable}

\subsection{Marco Teórico (Conceptos
Clave)}\label{marco-teuxf3rico-conceptos-clave-8}

\textbf{Modernización de la Gestión Pública:} Proceso de transformación
de las entidades públicas orientado a incrementar sus niveles de
eficacia, eficiencia y transparencia en el uso de recursos, mediante la
implementación de modelos de gestión por resultados, gobierno
electrónico y servicio civil meritocrático.

\textbf{Política Nacional de Modernización de la Gestión Pública:} Marco
establecido por el D.S. N° 004-2013-PCM que define pilares centrales
como gobierno abierto, gobierno electrónico, articulación
intergubernamental, gestión por procesos y servicio civil.

\textbf{Gasto Público:} Conjunto de erogaciones realizadas por el Estado
en el cumplimiento de sus funciones, clasificado en gasto corriente
(operativo) y gasto de capital (inversión).

\subsection{Validación de
Instrumentos}\label{validaciuxf3n-de-instrumentos-2}

\begin{itemize}
\tightlist
\item
  \textbf{Validez de contenido:} Juicio de 3-5 expertos en gestión
  pública
\item
  \textbf{Confiabilidad:} Alfa de Cronbach ≥ 0.75
\item
  \textbf{Prueba piloto:} n=30
\end{itemize}

\subsection{Referencias Principales}\label{referencias-principales-8}

\begin{itemize}
\tightlist
\item
  Secretaría de Gestión Pública. (2013). \emph{Política Nacional de
  Modernización de la Gestión Pública al 2021}. D.S. N° 004-2013-PCM.
  Lima: PCM.
\item
  SERVIR. (2020). \emph{Reforma del servicio civil en el Perú}. Lima:
  Autoridad Nacional del Servicio Civil.
\item
  OCDE. (2016). \emph{Estudio de la OCDE sobre la Gobernanza Pública en
  el Perú}. París: OECD Publishing.
\end{itemize}

\section{Matriz 10: Determinantes Microeconómicos de la Pobreza en
Ayacucho}\label{matriz-10-determinantes-microeconuxf3micos-de-la-pobreza-en-ayacucho}

\subsection{Título}\label{tuxedtulo-9}

\textbf{Determinantes Microeconómicos de la Pobreza en el Departamento
de Ayacucho, Periodo 2000-2020}

\subsection{Datos Generales}\label{datos-generales-9}

\begin{itemize}
\tightlist
\item
  \textbf{Área de investigación:} Economía del Desarrollo
\item
  \textbf{Línea de investigación:} Economía, Econometría
\item
  \textbf{Tipo de investigación:} Aplicada, Explicativa, Longitudinal
\item
  \textbf{Enfoque:} Cuantitativo
\end{itemize}

\begin{longtable}[]{@{}
  >{\raggedright\arraybackslash}p{(\linewidth - 4\tabcolsep) * \real{0.3333}}
  >{\raggedright\arraybackslash}p{(\linewidth - 4\tabcolsep) * \real{0.3333}}
  >{\raggedright\arraybackslash}p{(\linewidth - 4\tabcolsep) * \real{0.3333}}@{}}
\toprule\noalign{}
\begin{minipage}[b]{\linewidth}\raggedright
PROBLEMAS
\end{minipage} & \begin{minipage}[b]{\linewidth}\raggedright
OBJETIVOS
\end{minipage} & \begin{minipage}[b]{\linewidth}\raggedright
HIPÓTESIS
\end{minipage} \\
\midrule\noalign{}
\endhead
\bottomrule\noalign{}
\endlastfoot
\textbf{PROBLEMA GENERAL} & \textbf{OBJETIVO GENERAL} &
\textbf{HIPÓTESIS GENERAL} \\
¿Cuáles son los determinantes microeconómicos de la probabilidad de ser
pobre en el departamento de Ayacucho durante el periodo 2000-2020? &
Identificar los determinantes microeconómicos de la probabilidad de ser
pobre en el departamento de Ayacucho durante el periodo 2000-2020,
mediante un modelo de probabilidad. & Las variables demográficas, de
capital humano y de actividad económica son determinantes significativos
de la probabilidad de que un hogar sea pobre en Ayacucho durante el
periodo 2000-2020. \\
\textbf{PROBLEMAS ESPECÍFICOS} & \textbf{OBJETIVOS ESPECÍFICOS} &
\textbf{HIPÓTESIS ESPECÍFICAS} \\
PE1: ¿Cuál es la incidencia de las variables demográficas sobre la
probabilidad de que el hogar sea pobre en Ayacucho? & OE1: Determinar la
incidencia de las variables demográficas sobre la probabilidad de que el
hogar sea pobre en Ayacucho. & HE1: Las variables demográficas (edad del
jefe de hogar, tamaño del hogar, área de residencia) inciden
significativamente en la probabilidad de ser pobre. A menor edad del
jefe y mayor tamaño del hogar, mayor probabilidad de pobreza. \\
PE2: ¿Cuál es la incidencia de las variables de capital humano sobre la
probabilidad de que el hogar sea pobre en Ayacucho? & OE2: Precisar la
incidencia de las variables de capital humano sobre la probabilidad de
que el hogar sea pobre en Ayacucho. & HE2: El capital humano (educación
del jefe, capacitación laboral) incide negativamente en la probabilidad
de ser pobre. A mayor educación, menor probabilidad de pobreza. \\
PE3: ¿Cuál es la incidencia de la actividad económica sobre la
probabilidad de que el hogar sea pobre en Ayacucho? & OE3: Mostrar la
incidencia de las variables del hogar y actividad económica sobre la
probabilidad de que el hogar sea pobre en Ayacucho. & HE3: La rama de
actividad económica y el número de perceptores de ingreso inciden
significativamente en la probabilidad de pobreza. La informalidad
aumenta la probabilidad de ser pobre. \\
\end{longtable}

\subsection{Variables e Indicadores}\label{variables-e-indicadores-9}

\begin{longtable}[]{@{}
  >{\raggedright\arraybackslash}p{(\linewidth - 8\tabcolsep) * \real{0.2000}}
  >{\raggedright\arraybackslash}p{(\linewidth - 8\tabcolsep) * \real{0.1200}}
  >{\raggedright\arraybackslash}p{(\linewidth - 8\tabcolsep) * \real{0.2600}}
  >{\raggedright\arraybackslash}p{(\linewidth - 8\tabcolsep) * \real{0.2600}}
  >{\raggedright\arraybackslash}p{(\linewidth - 8\tabcolsep) * \real{0.1600}}@{}}
\toprule\noalign{}
\begin{minipage}[b]{\linewidth}\raggedright
VARIABLE
\end{minipage} & \begin{minipage}[b]{\linewidth}\raggedright
TIPO
\end{minipage} & \begin{minipage}[b]{\linewidth}\raggedright
DIMENSIONES
\end{minipage} & \begin{minipage}[b]{\linewidth}\raggedright
INDICADORES
\end{minipage} & \begin{minipage}[b]{\linewidth}\raggedright
ESCALA
\end{minipage} \\
\midrule\noalign{}
\endhead
\bottomrule\noalign{}
\endlastfoot
\textbf{Pobreza} & Dependiente & Condición de Pobreza & - Variable
dummy: 1 = pobre, 0 = no pobre- Según línea de pobreza monetaria del
INEI & Nominal (dicotómica) \\
\textbf{Variables Demográficas} & Independiente & Características del
Jefe de Hogar & - Edad del jefe de hogar (años)- Sexo (1=hombre,
0=mujer)- Estado civil (casado/conviviente vs.~otros) & Razón/Nominal \\
& & Características del Hogar & - Tamaño del hogar (número de miembros)-
Número de niños menores de 12 años- Razón de dependencia & Razón \\
& & Localización & - Área de residencia (1=urbano, 0=rural)- Región
natural (costa, sierra, selva) & Nominal \\
\textbf{Capital Humano} & Independiente & Educación & - Años de
educación del jefe de hogar- Nivel educativo alcanzado (categorías)-
Tasa de analfabetismo del hogar & Razón/Ordinal \\
& & Capacitación & - Acceso a capacitación laboral- Educación técnica o
superior & Nominal \\
\textbf{Actividad Económica} & Independiente & Empleo & - Número de
perceptores de ingreso- Rama de actividad económica del jefe- Categoría
ocupacional (asalariado, independiente, etc.) & Razón/Nominal \\
& & Informalidad & - Condición de informalidad (sin beneficios
laborales)- Acceso a seguridad social & Nominal \\
& & Activos & - Propiedad de la vivienda- Acceso a activos productivos &
Nominal \\
\end{longtable}

\subsection{Metodología}\label{metodologuxeda-9}

\begin{longtable}[]{@{}
  >{\raggedright\arraybackslash}p{(\linewidth - 2\tabcolsep) * \real{0.4091}}
  >{\raggedright\arraybackslash}p{(\linewidth - 2\tabcolsep) * \real{0.5909}}@{}}
\toprule\noalign{}
\begin{minipage}[b]{\linewidth}\raggedright
ASPECTO
\end{minipage} & \begin{minipage}[b]{\linewidth}\raggedright
DESCRIPCIÓN
\end{minipage} \\
\midrule\noalign{}
\endhead
\bottomrule\noalign{}
\endlastfoot
\textbf{Tipo de Investigación} & - Aplicada- Explicativa-
Longitudinal \\
\textbf{Diseño} & No experimental, longitudinal retrospectivo (análisis
de datos secundarios) \\
\textbf{Nivel} & Explicativo \\
\textbf{Enfoque} & Cuantitativo \\
\textbf{Método} & Econométrico, modelos de probabilidad \\
\textbf{Población} & Hogares del departamento de Ayacucho durante el
periodo 2000-2020 \\
\textbf{Muestra} & Base de datos de la ENAHO: Panel de hogares de
Ayacucho (aproximadamente 1,000-1,500 hogares/año) \\
\textbf{Técnicas de Recolección} & - Análisis de datos secundarios
(ENAHO) \\
\textbf{Instrumentos} & - Base de datos ENAHO (INEI) \\
\textbf{Fuentes de Datos} & - INEI (Encuesta Nacional de Hogares - ENAHO
2000-2020)- Microdatos disponibles en portal del INEI \\
\textbf{Técnicas de Análisis} & - Estadística descriptiva- Modelos de
probabilidad: • Modelo Logit • Modelo Probit- Cálculo de efectos
marginales- Pruebas de bondad de ajuste- Análisis de sensibilidad \\
\textbf{Software Estadístico} & Stata, R, SPSS, Python \\
\end{longtable}

\subsection{Marco Teórico (Conceptos
Clave)}\label{marco-teuxf3rico-conceptos-clave-9}

\textbf{Pobreza Monetaria:} Situación en la cual el gasto per cápita de
un hogar es inferior al costo de una canasta básica de consumo (línea de
pobreza), definida por el INEI.

\textbf{Modelos de Probabilidad:} Técnicas econométricas que permiten
modelar variables dependientes binarias (pobre/no pobre), estimando la
probabilidad de pertenecer a una categoría en función de variables
explicativas.

\textbf{Modelo Logit:} Modelo de regresión que utiliza la función
logística para estimar probabilidades, asumiendo que los errores siguen
una distribución logística.

\textbf{Efectos Marginales:} Cambio en la probabilidad predicha ante un
cambio marginal en una variable explicativa, manteniendo constantes las
demás variables.

\subsection{Modelo Econométrico
Propuesto}\label{modelo-economuxe9trico-propuesto-2}

\textbf{Modelo Logit:}

\begin{verbatim}
P(Pobre_i = 1) = Λ(β₀ + β₁·Edad_i + β₂·Educación_i + β₃·TamañoHogar_i + 
                     β₄·RamaActividad_i + β₅·Informalidad_i + β₆·Área_i + ε_i)
\end{verbatim}

Donde:

\begin{itemize}
\tightlist
\item
  Λ = Función logística
\item
  Pobre\_i = 1 si el hogar es pobre, 0 si no
\item
  Edad, Educación, etc. = Variables explicativas
\end{itemize}

\textbf{Especificación alternativa: Modelo Probit}

\textbf{Interpretación:}

\begin{itemize}
\tightlist
\item
  Coeficientes β: Efecto en el logit (log-odds)
\item
  Efectos marginales: Cambio en probabilidad ante cambio unitario en X
\end{itemize}

\subsection{Hipótesis de Signo
Esperado}\label{hipuxf3tesis-de-signo-esperado}

\begin{itemize}
\tightlist
\item
  Edad del jefe: Relación no lineal (U invertida)
\item
  Educación del jefe: Negativo (↓ educación → ↑ pobreza)
\item
  Tamaño del hogar: Positivo (↑ miembros → ↑ pobreza)
\item
  Número de perceptores: Negativo (↑ perceptores → ↓ pobreza)
\item
  Informalidad: Positivo (informal → ↑ pobreza)
\item
  Área urbana: Negativo (urbano → ↓ pobreza)
\end{itemize}

\subsection{Referencias Principales}\label{referencias-principales-9}

\begin{itemize}
\tightlist
\item
  Herrera, J. (2017). \emph{Pobreza y desigualdad en el Perú 2016}.
  Lima: INEI.
\item
  Wooldridge, J. M. (2010). \emph{Econometric Analysis of Cross Section
  and Panel Data} (2nd ed.). MIT Press.
\item
  Ravallion, M. (1996). \emph{Issues in measuring and modelling
  poverty}. The Economic Journal, 106(438), 1328-1343.
\end{itemize}

\section{MATRIZ 10: Determinantes Microeconómicos de la
Pobreza}\label{matriz-10-determinantes-microeconuxf3micos-de-la-pobreza}

\subsection{Título}\label{tuxedtulo-10}

\textbf{Determinantes Microeconómicos de la Pobreza en el Departamento
de Ayacucho, Periodo 2000-2020}

\begin{longtable}[]{@{}
  >{\raggedright\arraybackslash}p{(\linewidth - 2\tabcolsep) * \real{0.4762}}
  >{\raggedright\arraybackslash}p{(\linewidth - 2\tabcolsep) * \real{0.5238}}@{}}
\toprule\noalign{}
\begin{minipage}[b]{\linewidth}\raggedright
ELEMENTO
\end{minipage} & \begin{minipage}[b]{\linewidth}\raggedright
CONTENIDO
\end{minipage} \\
\midrule\noalign{}
\endhead
\bottomrule\noalign{}
\endlastfoot
\textbf{PROBLEMA GENERAL} & ¿Cuáles son los determinantes
microeconómicos de la probabilidad de ser pobre en Ayacucho durante el
periodo 2000-2020? \\
\textbf{OBJETIVO GENERAL} & Identificar los determinantes
microeconómicos de la probabilidad de ser pobre en Ayacucho mediante un
modelo de probabilidad. \\
\textbf{HIPÓTESIS GENERAL} & Las variables demográficas, de capital
humano y de actividad económica son determinantes significativos de la
probabilidad de ser pobre. \\
\end{longtable}

\subsection{Operacionalización de
Variables}\label{operacionalizaciuxf3n-de-variables}

\begin{longtable}[]{@{}
  >{\raggedright\arraybackslash}p{(\linewidth - 6\tabcolsep) * \real{0.2222}}
  >{\raggedright\arraybackslash}p{(\linewidth - 6\tabcolsep) * \real{0.2444}}
  >{\raggedright\arraybackslash}p{(\linewidth - 6\tabcolsep) * \real{0.2444}}
  >{\raggedright\arraybackslash}p{(\linewidth - 6\tabcolsep) * \real{0.2889}}@{}}
\toprule\noalign{}
\begin{minipage}[b]{\linewidth}\raggedright
VARIABLE
\end{minipage} & \begin{minipage}[b]{\linewidth}\raggedright
DIMENSIÓN
\end{minipage} & \begin{minipage}[b]{\linewidth}\raggedright
INDICADOR
\end{minipage} & \begin{minipage}[b]{\linewidth}\raggedright
INSTRUMENTO
\end{minipage} \\
\midrule\noalign{}
\endhead
\bottomrule\noalign{}
\endlastfoot
\textbf{V.D: Pobreza} & Condición & - Variable dummy (1=pobre, 0=no
pobre) & Análisis ENAHO (INEI) \\
\textbf{V.I: Demográficas} & Jefe de hogar & - Edad- Sexo- Estado civil
& Análisis ENAHO \\
& Hogar & - Tamaño del hogar- N° niños menores 12 años & Análisis
ENAHO \\
& Localización & - Área (urbano/rural)- Región natural & Análisis
ENAHO \\
\textbf{V.I: Capital Humano} & Educación & - Años de educación jefe-
Nivel educativo & Análisis ENAHO \\
\textbf{V.I: Actividad Económica} & Empleo & - N° perceptores de
ingreso- Rama de actividad- Informalidad & Análisis ENAHO \\
\end{longtable}

\subsection{Metodología}\label{metodologuxeda-10}

\begin{itemize}
\tightlist
\item
  \textbf{Tipo:} Aplicada, Explicativa
\item
  \textbf{Diseño:} Retrospectivo, panel de hogares
\item
  \textbf{Técnicas:} Modelo Logit/Probit, efectos marginales
\item
  \textbf{Software:} Stata, R, SPSS
\end{itemize}

\section{MATRIZ 11: Factores de Desnutrición
Infantil}\label{matriz-11-factores-de-desnutriciuxf3n-infantil}

\subsection{Título}\label{tuxedtulo-11}

\textbf{Factores Asociados a la Desnutrición en Niños Menores de 5 Años
en el Perú}

\begin{longtable}[]{@{}
  >{\raggedright\arraybackslash}p{(\linewidth - 2\tabcolsep) * \real{0.4762}}
  >{\raggedright\arraybackslash}p{(\linewidth - 2\tabcolsep) * \real{0.5238}}@{}}
\toprule\noalign{}
\begin{minipage}[b]{\linewidth}\raggedright
ELEMENTO
\end{minipage} & \begin{minipage}[b]{\linewidth}\raggedright
CONTENIDO
\end{minipage} \\
\midrule\noalign{}
\endhead
\bottomrule\noalign{}
\endlastfoot
\textbf{PROBLEMA GENERAL} & ¿De qué manera la pobreza y el analfabetismo
influyen en la desnutrición de niños menores de 5 años en los
departamentos del Perú (2009-2019)? \\
\textbf{OBJETIVO GENERAL} & Determinar la influencia de la pobreza y el
analfabetismo en la desnutrición de niños menores de 5 años en los
departamentos del Perú. \\
\textbf{HIPÓTESIS GENERAL} & La pobreza y el analfabetismo influyen
significativamente en la desnutrición de niños menores de 5 años en el
Perú. \\
\end{longtable}

\subsection{Operacionalización de
Variables}\label{operacionalizaciuxf3n-de-variables-1}

\begin{longtable}[]{@{}
  >{\raggedright\arraybackslash}p{(\linewidth - 6\tabcolsep) * \real{0.2632}}
  >{\raggedright\arraybackslash}p{(\linewidth - 6\tabcolsep) * \real{0.1579}}
  >{\raggedright\arraybackslash}p{(\linewidth - 6\tabcolsep) * \real{0.2895}}
  >{\raggedright\arraybackslash}p{(\linewidth - 6\tabcolsep) * \real{0.2895}}@{}}
\toprule\noalign{}
\begin{minipage}[b]{\linewidth}\raggedright
VARIABLE
\end{minipage} & \begin{minipage}[b]{\linewidth}\raggedright
TIPO
\end{minipage} & \begin{minipage}[b]{\linewidth}\raggedright
DIMENSIÓN
\end{minipage} & \begin{minipage}[b]{\linewidth}\raggedright
INDICADOR
\end{minipage} \\
\midrule\noalign{}
\endhead
\bottomrule\noalign{}
\endlastfoot
\textbf{Desnutrición} & Dependiente & Nutricional & - Tasa de
desnutrición crónica (\%) \\
\textbf{Pobreza} & Independiente & Económica & - Tasa de pobreza (\%) \\
\textbf{Analfabetismo} & Independiente & Educativa & - Tasa de
analfabetismo (\%) \\
\textbf{Inflación} & Control & Económica & - IPC (Índice de Precios al
Consumidor) \\
\end{longtable}

\subsection{Metodología}\label{metodologuxeda-11}

\begin{itemize}
\tightlist
\item
  \textbf{Tipo:} Aplicada, Explicativa
\item
  \textbf{Diseño:} Descriptivo-correlacional, longitudinal
\item
  \textbf{Población:} 24 departamentos del Perú (2009-2019)
\item
  \textbf{Análisis:} Correlación, regresión múltiple
\item
  \textbf{Fuentes:} INEI (ENDES, ENAHO), MINSA
\item
  \textbf{Software:} SPSS, Stata
\end{itemize}

\section{MATRIZ 12: Recaudación de Impuestos
Prediales}\label{matriz-12-recaudaciuxf3n-de-impuestos-prediales}

\subsection{Título}\label{tuxedtulo-12}

\textbf{Influencia de la Recaudación de Impuestos Prediales en la
Recaudación Tributaria de la Municipalidad Distrital de Huamanga, 2021}

\begin{longtable}[]{@{}
  >{\raggedright\arraybackslash}p{(\linewidth - 2\tabcolsep) * \real{0.4762}}
  >{\raggedright\arraybackslash}p{(\linewidth - 2\tabcolsep) * \real{0.5238}}@{}}
\toprule\noalign{}
\begin{minipage}[b]{\linewidth}\raggedright
ELEMENTO
\end{minipage} & \begin{minipage}[b]{\linewidth}\raggedright
CONTENIDO
\end{minipage} \\
\midrule\noalign{}
\endhead
\bottomrule\noalign{}
\endlastfoot
\textbf{PROBLEMA GENERAL} & ¿De qué manera el impuesto predial se
relaciona con la recaudación tributaria en la Municipalidad Distrital de
Huamanga-2021? \\
\textbf{OBJETIVO GENERAL} & Analizar si el impuesto predial se relaciona
con la recaudación tributaria en la municipalidad distrital de
Huamanga-2021. \\
\textbf{HIPÓTESIS GENERAL} & El impuesto predial se relaciona
significativamente con la recaudación tributaria en la Municipalidad
Distrital de Huamanga-2021. \\
\end{longtable}

\subsection{Operacionalización de
Variables}\label{operacionalizaciuxf3n-de-variables-2}

\begin{longtable}[]{@{}
  >{\raggedright\arraybackslash}p{(\linewidth - 6\tabcolsep) * \real{0.2222}}
  >{\raggedright\arraybackslash}p{(\linewidth - 6\tabcolsep) * \real{0.2444}}
  >{\raggedright\arraybackslash}p{(\linewidth - 6\tabcolsep) * \real{0.2444}}
  >{\raggedright\arraybackslash}p{(\linewidth - 6\tabcolsep) * \real{0.2889}}@{}}
\toprule\noalign{}
\begin{minipage}[b]{\linewidth}\raggedright
VARIABLE
\end{minipage} & \begin{minipage}[b]{\linewidth}\raggedright
DIMENSIÓN
\end{minipage} & \begin{minipage}[b]{\linewidth}\raggedright
INDICADOR
\end{minipage} & \begin{minipage}[b]{\linewidth}\raggedright
INSTRUMENTO
\end{minipage} \\
\midrule\noalign{}
\endhead
\bottomrule\noalign{}
\endlastfoot
\textbf{V.I: Impuesto Predial} & Recaudación & - Monto recaudado- N°
contribuyentes & Análisis documental (Municipalidad) \\
& Valorización & - Auto valúo promedio- N° predios registrados &
Análisis documental \\
\textbf{V.D: Recaudación Tributaria} & Impuestos municipales & - Monto
total impuestos- \% ejecución & Análisis documental (SIAF-GL) \\
& Tasas & - Licencias de funcionamiento- Otras tasas & Análisis
documental \\
\end{longtable}

\subsection{Metodología}\label{metodologuxeda-12}

\begin{itemize}
\tightlist
\item
  \textbf{Tipo:} Descriptivo-correlacional
\item
  \textbf{Diseño:} No experimental, transversal
\item
  \textbf{Población:} Contribuyentes de Huamanga
\item
  \textbf{Técnicas:} Correlación, regresión
\item
  \textbf{Fuentes:} Municipalidad, MEF, BCRP
\item
  \textbf{Software:} SPSS, Excel
\end{itemize}

\section{MATRIZ 13: Choques Externos e Ingresos
Tributarios}\label{matriz-13-choques-externos-e-ingresos-tributarios}

\subsection{Título}\label{tuxedtulo-13}

\textbf{Choques Externos e Internos sobre la Dinámica de los Ingresos
Tributarios en el Perú, Periodo 2000-2019}

\begin{longtable}[]{@{}
  >{\raggedright\arraybackslash}p{(\linewidth - 2\tabcolsep) * \real{0.4762}}
  >{\raggedright\arraybackslash}p{(\linewidth - 2\tabcolsep) * \real{0.5238}}@{}}
\toprule\noalign{}
\begin{minipage}[b]{\linewidth}\raggedright
ELEMENTO
\end{minipage} & \begin{minipage}[b]{\linewidth}\raggedright
CONTENIDO
\end{minipage} \\
\midrule\noalign{}
\endhead
\bottomrule\noalign{}
\endlastfoot
\textbf{PROBLEMA GENERAL} & ¿De qué manera los shocks externos e
internos afectan los ingresos tributarios en Perú (2000-2019)? \\
\textbf{OBJETIVO GENERAL} & Evaluar la influencia del PBI y los precios
de exportación en los ingresos tributarios de Perú (2000-2019). \\
\textbf{HIPÓTESIS GENERAL} & Los indicadores del PBI e IPX han influido
positivamente en la dinámica de los ingresos tributarios de Perú
(2000-2019). \\
\end{longtable}

\subsection{Operacionalización de
Variables}\label{operacionalizaciuxf3n-de-variables-3}

\begin{longtable}[]{@{}
  >{\raggedright\arraybackslash}p{(\linewidth - 6\tabcolsep) * \real{0.2500}}
  >{\raggedright\arraybackslash}p{(\linewidth - 6\tabcolsep) * \real{0.1500}}
  >{\raggedright\arraybackslash}p{(\linewidth - 6\tabcolsep) * \real{0.2750}}
  >{\raggedright\arraybackslash}p{(\linewidth - 6\tabcolsep) * \real{0.3250}}@{}}
\toprule\noalign{}
\begin{minipage}[b]{\linewidth}\raggedright
VARIABLE
\end{minipage} & \begin{minipage}[b]{\linewidth}\raggedright
TIPO
\end{minipage} & \begin{minipage}[b]{\linewidth}\raggedright
INDICADOR
\end{minipage} & \begin{minipage}[b]{\linewidth}\raggedright
INSTRUMENTO
\end{minipage} \\
\midrule\noalign{}
\endhead
\bottomrule\noalign{}
\endlastfoot
\textbf{V.D: Ingresos Tributarios} & Dependiente & - Ingresos
tributarios del gobierno central (millones S/.) & Análisis documental
(SUNAT, MEF, BCRP) \\
\textbf{V.I: Índice de Precios de Exportación} & Independiente & - IPX
(índice base 100) & Análisis documental (BCRP) \\
\textbf{V.I: PBI Real} & Independiente & - PBI (millones S/.
constantes)- Tasa de crecimiento & Análisis documental (BCRP, INEI) \\
\end{longtable}

\subsection{Metodología}\label{metodologuxeda-13}

\begin{itemize}
\tightlist
\item
  \textbf{Tipo:} No experimental, longitudinal
\item
  \textbf{Diseño:} Series de tiempo (20 años)
\item
  \textbf{Técnicas:} VAR, VECM, Causalidad de Granger, Impulso-respuesta
\item
  \textbf{Fuentes:} BCRP, INEI, SUNAT
\item
  \textbf{Software:} EViews, Stata, R
\end{itemize}

\section{Fuentes de Datos}\label{fuentes-de-datos}

\subsection{Instituciones Nacionales}\label{instituciones-nacionales}

\begin{itemize}
\tightlist
\item
  \textbf{INEI:} ENAHO, ENDES, Censos, PBI, pobreza
\item
  \textbf{MEF:} Portal de Transparencia, SIAF, presupuesto público
\item
  \textbf{BCRP:} Series macroeconómicas, PBI regional
\item
  \textbf{SUNAT:} Recaudación tributaria
\item
  \textbf{MIDIS:} Programas sociales, padrón de beneficiarios
\item
  \textbf{MINEDU:} ESCALE, ECE, indicadores educativos
\item
  \textbf{MINSA:} Indicadores de salud
\end{itemize}

\subsection{Instituciones
Internacionales}\label{instituciones-internacionales}

\begin{itemize}
\tightlist
\item
  \textbf{CEPAL:} Estadísticas socioeconómicas de América Latina
\item
  \textbf{PNUD:} Índice de Desarrollo Humano
\item
  \textbf{Banco Mundial:} Indicadores de desarrollo
\end{itemize}

\begin{itemize}
\tightlist
\item
  Wooldridge, J. M. (2010). \emph{Econometric Analysis of Cross Section
  and Panel Data}. MIT Press.
\item
  Gujarati, D. N., \& Porter, D. C. (2009). \emph{Econometría} (5ta
  ed.). McGraw-Hill.
\item
  Enders, W. (2015). \emph{Applied Econometric Time Series} (4th ed.).
  Wiley.
\item
  Angrist, J. D., \& Pischke, J. S. (2009). \emph{Mostly Harmless
  Econometrics}. Princeton University Press.
\item
  Cameron, A. C., \& Trivedi, P. K. (2005). \emph{Microeconometrics}.
  Cambridge University Press.
\item
  Gertler, P., et al.~(2016). \emph{La evaluación de impacto en la
  práctica} (2da ed.). Banco Mundial.
\item
  Khandker, S., Koolwal, G., \& Samad, H. (2010). \emph{Handbook on
  Impact Evaluation}. World Bank.
\end{itemize}

\appendix

\section{Publicaciones Similares}\label{apx-publicaciones-similares}

Si te interesó este artículo, te recomendamos que explores otros blogs y
recursos relacionados que pueden ampliar tus conocimientos. Aquí te dejo
algunas sugerencias:

\begin{enumerate}
\def\labelenumi{\arabic{enumi}.}
\tightlist
\item
  \href{https://methodica.netlify.app/posts/2023-06-03-ideas-de-investigacion-para-economia/index.pdf}{\faIcon{file-pdf}}
  \href{https://methodica.netlify.app/posts/2023-06-03-ideas-de-investigacion-para-economia}{Ideas
  De Investigacion Para Economia}
\item
  \href{https://methodica.netlify.app/posts/2023-06-03-pautas-de-presentacion-del-informe-de-investigacion/index.pdf}{\faIcon{file-pdf}}
  \href{https://methodica.netlify.app/posts/2023-06-03-pautas-de-presentacion-del-informe-de-investigacion}{Pautas
  De Presentacion Del Informe De Investigacion}
\item
  \href{https://methodica.netlify.app/posts/2025-01-12-recursos-de-bibliografia-y-documentacion/index.pdf}{\faIcon{file-pdf}}
  \href{https://methodica.netlify.app/posts/2025-01-12-recursos-de-bibliografia-y-documentacion}{Recursos
  De Bibliografia Y Documentacion}
\item
  \href{https://methodica.netlify.app/posts/2025-02-09-recursos-para-traducción-y-correccion/index.pdf}{\faIcon{file-pdf}}
  \href{https://methodica.netlify.app/posts/2025-02-09-recursos-para-traducción-y-correccion}{Recursos
  Para Traducción Y Correccion}
\item
  \href{https://methodica.netlify.app/posts/2025-04-23-tipos-de-elementos-en-zotero/index.pdf}{\faIcon{file-pdf}}
  \href{https://methodica.netlify.app/posts/2025-04-23-tipos-de-elementos-en-zotero}{Tipos
  De Elementos En Zotero}
\end{enumerate}

Esperamos que encuentres estas publicaciones igualmente interesantes y
útiles. ¡Disfruta de la lectura!






\end{document}
