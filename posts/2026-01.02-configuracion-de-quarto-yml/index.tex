\documentclass[
  doc,
  floatsintext,
  longtable,
  a4paper,
  nolmodern,
  notxfonts,
  notimes,
  12pt,
  colorlinks=true,linkcolor=blue,citecolor=blue,urlcolor=blue]{apa7}

\usepackage{amsmath}
\usepackage{amssymb}

\geometry{inner=1in, outer=1in}
\fancyhfoffset[LE,RO]{0cm}


\usepackage[bidi=default]{babel}
\babelprovide[main,import]{spanish}


% get rid of language-specific shorthands (see #6817):
\let\LanguageShortHands\languageshorthands
\def\languageshorthands#1{}

\RequirePackage{longtable}
\RequirePackage{threeparttablex}

\makeatletter
\renewcommand{\paragraph}{\@startsection{paragraph}{4}{\parindent}%
	{0\baselineskip \@plus 0.2ex \@minus 0.2ex}%
	{-.5em}%
	{\normalfont\normalsize\bfseries\typesectitle}}

\renewcommand{\subparagraph}[1]{\@startsection{subparagraph}{5}{0.5em}%
	{0\baselineskip \@plus 0.2ex \@minus 0.2ex}%
	{-\z@\relax}%
	{\normalfont\normalsize\bfseries\itshape\hspace{\parindent}{#1}\textit{\addperi}}{\relax}}
\makeatother




\usepackage{longtable, booktabs, multirow, multicol, colortbl, hhline, caption, array, float, xpatch}
\usepackage{subcaption}


\renewcommand\thesubfigure{\Alph{subfigure}}
\setcounter{topnumber}{2}
\setcounter{bottomnumber}{2}
\setcounter{totalnumber}{4}
\renewcommand{\topfraction}{0.85}
\renewcommand{\bottomfraction}{0.85}
\renewcommand{\textfraction}{0.15}
\renewcommand{\floatpagefraction}{0.7}

\usepackage{tcolorbox}
\tcbuselibrary{listings,theorems, breakable, skins}
\usepackage{fontawesome5}

\definecolor{quarto-callout-color}{HTML}{909090}
\definecolor{quarto-callout-note-color}{HTML}{0758E5}
\definecolor{quarto-callout-important-color}{HTML}{CC1914}
\definecolor{quarto-callout-warning-color}{HTML}{EB9113}
\definecolor{quarto-callout-tip-color}{HTML}{00A047}
\definecolor{quarto-callout-caution-color}{HTML}{FC5300}
\definecolor{quarto-callout-color-frame}{HTML}{ACACAC}
\definecolor{quarto-callout-note-color-frame}{HTML}{4582EC}
\definecolor{quarto-callout-important-color-frame}{HTML}{D9534F}
\definecolor{quarto-callout-warning-color-frame}{HTML}{F0AD4E}
\definecolor{quarto-callout-tip-color-frame}{HTML}{02B875}
\definecolor{quarto-callout-caution-color-frame}{HTML}{FD7E14}

%\newlength\Oldarrayrulewidth
%\newlength\Oldtabcolsep


\usepackage{hyperref}



\usepackage{color}
\usepackage{fancyvrb}
\newcommand{\VerbBar}{|}
\newcommand{\VERB}{\Verb[commandchars=\\\{\}]}
\DefineVerbatimEnvironment{Highlighting}{Verbatim}{commandchars=\\\{\}}
% Add ',fontsize=\small' for more characters per line
\usepackage{framed}
\definecolor{shadecolor}{RGB}{241,243,245}
\newenvironment{Shaded}{\begin{snugshade}}{\end{snugshade}}
\newcommand{\AlertTok}[1]{\textcolor[rgb]{0.68,0.00,0.00}{#1}}
\newcommand{\AnnotationTok}[1]{\textcolor[rgb]{0.37,0.37,0.37}{#1}}
\newcommand{\AttributeTok}[1]{\textcolor[rgb]{0.40,0.45,0.13}{#1}}
\newcommand{\BaseNTok}[1]{\textcolor[rgb]{0.68,0.00,0.00}{#1}}
\newcommand{\BuiltInTok}[1]{\textcolor[rgb]{0.00,0.23,0.31}{#1}}
\newcommand{\CharTok}[1]{\textcolor[rgb]{0.13,0.47,0.30}{#1}}
\newcommand{\CommentTok}[1]{\textcolor[rgb]{0.37,0.37,0.37}{#1}}
\newcommand{\CommentVarTok}[1]{\textcolor[rgb]{0.37,0.37,0.37}{\textit{#1}}}
\newcommand{\ConstantTok}[1]{\textcolor[rgb]{0.56,0.35,0.01}{#1}}
\newcommand{\ControlFlowTok}[1]{\textcolor[rgb]{0.00,0.23,0.31}{\textbf{#1}}}
\newcommand{\DataTypeTok}[1]{\textcolor[rgb]{0.68,0.00,0.00}{#1}}
\newcommand{\DecValTok}[1]{\textcolor[rgb]{0.68,0.00,0.00}{#1}}
\newcommand{\DocumentationTok}[1]{\textcolor[rgb]{0.37,0.37,0.37}{\textit{#1}}}
\newcommand{\ErrorTok}[1]{\textcolor[rgb]{0.68,0.00,0.00}{#1}}
\newcommand{\ExtensionTok}[1]{\textcolor[rgb]{0.00,0.23,0.31}{#1}}
\newcommand{\FloatTok}[1]{\textcolor[rgb]{0.68,0.00,0.00}{#1}}
\newcommand{\FunctionTok}[1]{\textcolor[rgb]{0.28,0.35,0.67}{#1}}
\newcommand{\ImportTok}[1]{\textcolor[rgb]{0.00,0.46,0.62}{#1}}
\newcommand{\InformationTok}[1]{\textcolor[rgb]{0.37,0.37,0.37}{#1}}
\newcommand{\KeywordTok}[1]{\textcolor[rgb]{0.00,0.23,0.31}{\textbf{#1}}}
\newcommand{\NormalTok}[1]{\textcolor[rgb]{0.00,0.23,0.31}{#1}}
\newcommand{\OperatorTok}[1]{\textcolor[rgb]{0.37,0.37,0.37}{#1}}
\newcommand{\OtherTok}[1]{\textcolor[rgb]{0.00,0.23,0.31}{#1}}
\newcommand{\PreprocessorTok}[1]{\textcolor[rgb]{0.68,0.00,0.00}{#1}}
\newcommand{\RegionMarkerTok}[1]{\textcolor[rgb]{0.00,0.23,0.31}{#1}}
\newcommand{\SpecialCharTok}[1]{\textcolor[rgb]{0.37,0.37,0.37}{#1}}
\newcommand{\SpecialStringTok}[1]{\textcolor[rgb]{0.13,0.47,0.30}{#1}}
\newcommand{\StringTok}[1]{\textcolor[rgb]{0.13,0.47,0.30}{#1}}
\newcommand{\VariableTok}[1]{\textcolor[rgb]{0.07,0.07,0.07}{#1}}
\newcommand{\VerbatimStringTok}[1]{\textcolor[rgb]{0.13,0.47,0.30}{#1}}
\newcommand{\WarningTok}[1]{\textcolor[rgb]{0.37,0.37,0.37}{\textit{#1}}}

\providecommand{\tightlist}{%
  \setlength{\itemsep}{0pt}\setlength{\parskip}{0pt}}
\usepackage{longtable,booktabs,array}
\usepackage{calc} % for calculating minipage widths
% Correct order of tables after \paragraph or \subparagraph
\usepackage{etoolbox}
\makeatletter
\patchcmd\longtable{\par}{\if@noskipsec\mbox{}\fi\par}{}{}
\makeatother
% Allow footnotes in longtable head/foot
\IfFileExists{footnotehyper.sty}{\usepackage{footnotehyper}}{\usepackage{footnote}}
\makesavenoteenv{longtable}

\usepackage{graphicx}
\makeatletter
\newsavebox\pandoc@box
\newcommand*\pandocbounded[1]{% scales image to fit in text height/width
  \sbox\pandoc@box{#1}%
  \Gscale@div\@tempa{\textheight}{\dimexpr\ht\pandoc@box+\dp\pandoc@box\relax}%
  \Gscale@div\@tempb{\linewidth}{\wd\pandoc@box}%
  \ifdim\@tempb\p@<\@tempa\p@\let\@tempa\@tempb\fi% select the smaller of both
  \ifdim\@tempa\p@<\p@\scalebox{\@tempa}{\usebox\pandoc@box}%
  \else\usebox{\pandoc@box}%
  \fi%
}
% Set default figure placement to htbp
\def\fps@figure{htbp}
\makeatother







\usepackage{newtx}

\defaultfontfeatures{Scale=MatchLowercase}
\defaultfontfeatures[\rmfamily]{Ligatures=TeX,Scale=1}





\title{Configuración de \texttt{\_quarto.yml} para Proyectos Quarto}


\shorttitle{Configuración de \texttt{\_quarto.yml} para Proyectos
Quarto}


\usepackage{etoolbox}



\ccoppy{\textcopyright~2026}



\author{Edison Achalma}



\affiliation{
{Escuela Profesional de Economía, Universidad Nacional de San Cristóbal
de Huamanga}}




\leftheader{Achalma}

\date{2026-01-02}


\abstract{Esta publiación proporciona una guía sobre la configuración
del archivo \_quarto.yml en proyectos Quarto, que actúa como el archivo
principal para definir opciones globales aplicables a sitios web
académicos, libros o proyectos simples. Se explica su diferencia con
\_metadata.yml, destacando el alcance global del primero frente al
directorio específico del segundo. Se detallan las funciones
principales, como el control del tipo de proyecto, estructura del sitio
(navegación, footer, sidebar), apariencia (temas, CSS), metadatos SEO,
integración de herramientas externas (Google Analytics, comentarios) y
formatos de salida (HTML, PDF, DOCX). Se analiza la jerarquía de
configuración, con énfasis en la prioridad (del front-matter del
documento a los valores por defecto) y la herencia/combinación de
opciones. La estructura del archivo se desglosa en secciones clave;
project (tipo, output-dir, resources), website (título, navbar, sidebar,
footer, anuncios, Open Graph, Twitter Cards), y format (opciones por
formato). Además, se incluyen configuraciones avanzadas, un análisis de
una configuración ejemplo, sugerencias de optimización, casos de uso
prácticos (blog académico, documentación técnica, portfolio), mejores
prácticas, troubleshooting y comandos útiles. El objetivo es capacitar a
investigadores y desarrolladores para crear sitios web académicos y
profesionales eficientes con Quarto, priorizando accesibilidad, SEO y
privacidad (RGPD). Se enfatiza la validación YAML, testing y recursos
adicionales para una implementación óptima. }

\keywords{Quarto, \_metadata.yml, Metadatos compartidos}

\authornote{\par{\addORCIDlink{Edison Achalma}{0000-0001-6996-3364}} 
\par{ }
\par{   El autor no tiene conflictos de interés que revelar.    Los
roles de autor se clasificaron utilizando la taxonomía de roles de
colaborador (CRediT; https://credit.niso.org/) de la siguiente
manera:  Edison Achalma:   conceptualización, metodología, análisis
formal, investigación, recursos, curación de
datos, redacción, visualización, supervisión, administración del
proyecto}
\par{La correspondencia relativa a este artículo debe dirigirse a Edison
Achalma, Escuela Profesional de Economía, Universidad Nacional de San
Cristóbal de Huamanga, Portal Independencia N°
57, Ayacucho, AYA 5001, Perú, Email: \href{mailto:elmer.achalma.09@unsch.edu.pe}{elmer.achalma.09@unsch.edu.pe}}
}

\makeatletter
\let\endoldlt\endlongtable
\def\endlongtable{
\hline
\endoldlt
}
\makeatother

\urlstyle{same}



\usepackage{hyperref}
\hypersetup{
  pdftitle={Configuración de _quarto.yml para Proyectos Quarto},
  pdfauthor={Edison Achalma},
  pdfsubject={Esta publiación proporciona una guía sobre la configuración del archivo _quarto.yml en proyectos Quarto, que actúa como el archivo principal para definir opciones globales aplicables a sitios web académicos, libros o proyectos simples. Se explica su diferencia con _metadata.yml, destacando el alcance global del primero frente al directorio específico del segundo. Se detallan las funciones principales, como el control del tipo de proyecto, estructura del sitio (navegación, footer, sidebar), apariencia (temas, CSS), metadatos SEO, integración de herramientas externas (Google Analytics, comentarios) y formatos de salida (HTML, PDF, DOCX). Se analiza la jerarquía de configuración, con énfasis en la prioridad (del front-matter del documento a los valores por defecto) y la herencia/combinación de opciones. La estructura del archivo se desglosa en secciones clave; project (tipo, output-dir, resources), website (título, navbar, sidebar, footer, anuncios, Open Graph, Twitter Cards), y format (opciones por formato). Además, se incluyen configuraciones avanzadas, un análisis de una configuración ejemplo, sugerencias de optimización, casos de uso prácticos (blog académico, documentación técnica, portfolio), mejores prácticas, troubleshooting y comandos útiles. El objetivo es capacitar a investigadores y desarrolladores para crear sitios web académicos y profesionales eficientes con Quarto, priorizando accesibilidad, SEO y privacidad (RGPD). Se enfatiza la validación YAML, testing y recursos adicionales para una implementación óptima.},
  pdfkeywords={Quarto, _metadata.yml, Metadatos compartidos},
  unicode=true,
  colorlinks=true,
  linkcolor=blue,
  citecolor=blue,
  urlcolor=blue,
  pdfcreator={Quarto},
  pdflang={es}
}
\makeatletter
\@ifpackageloaded{caption}{}{\usepackage{caption}}
\AtBeginDocument{%
\ifdefined\contentsname
  \renewcommand*\contentsname{Tabla de contenidos}
\else
  \newcommand\contentsname{Tabla de contenidos}
\fi
\ifdefined\listfigurename
  \renewcommand*\listfigurename{Lista de Figuras}
\else
  \newcommand\listfigurename{Lista de Figuras}
\fi
\ifdefined\listtablename
  \renewcommand*\listtablename{Lista de Tablas}
\else
  \newcommand\listtablename{Lista de Tablas}
\fi
\ifdefined\figurename
  \renewcommand*\figurename{Figura}
\else
  \newcommand\figurename{Figura}
\fi
\ifdefined\tablename
  \renewcommand*\tablename{Tabla}
\else
  \newcommand\tablename{Tabla}
\fi
}
\@ifpackageloaded{float}{}{\usepackage{float}}
\floatstyle{ruled}
\@ifundefined{c@chapter}{\newfloat{codelisting}{h}{lop}}{\newfloat{codelisting}{h}{lop}[chapter]}
\floatname{codelisting}{Listado}
\newcommand*\listoflistings{\listof{codelisting}{Lista de Listados}}
\makeatother
\makeatletter
\makeatother
\makeatletter
\@ifpackageloaded{caption}{}{\usepackage{caption}}
\@ifpackageloaded{subcaption}{}{\usepackage{subcaption}}
\makeatother
\makeatletter
\@ifpackageloaded{fontawesome5}{}{\usepackage{fontawesome5}}
\makeatother

% From https://tex.stackexchange.com/a/645996/211326
%%% apa7 doesn't want to add appendix section titles in the toc
%%% let's make it do it
\makeatletter
\xpatchcmd{\appendix}
  {\par}
  {\addcontentsline{toc}{section}{\@currentlabelname}\par}
  {}{}
\makeatother

%% Disable longtable counter
%% https://tex.stackexchange.com/a/248395/211326

\usepackage{etoolbox}

\makeatletter
\patchcmd{\LT@caption}
  {\bgroup}
  {\bgroup\global\LTpatch@captiontrue}
  {}{}
\patchcmd{\longtable}
  {\par}
  {\par\global\LTpatch@captionfalse}
  {}{}
\apptocmd{\endlongtable}
  {\ifLTpatch@caption\else\addtocounter{table}{-1}\fi}
  {}{}
\newif\ifLTpatch@caption
\makeatother

\begin{document}

\maketitle


\hypertarget{toc}{}
\tableofcontents
\newpage
\section[Introduction]{Configuración de \texttt{\_quarto.yml} para
Proyectos Quarto}

\setcounter{secnumdepth}{3}

\setlength\LTleft{0pt}




\section{\texorpdfstring{¿Qué es
\texttt{\_quarto.yml}?}{¿Qué es \_quarto.yml?}}\label{quuxe9-es-_quarto.yml}

\texttt{\_quarto.yml} es el archivo de configuración principal de un
proyecto Quarto. Define las opciones globales que se aplicarán a todo el
sitio web, libro o proyecto.

\textbf{Ubicación:}

\begin{verbatim}
proyecto/
├── _quarto.yml          <- Archivo principal (raíz del proyecto)
├── index.qmd
├── posts/
│   ├── _metadata.yml    <- Opciones específicas para posts
│   └── post1/
│       └── index.qmd
└── assets/
\end{verbatim}

\subsection{\texorpdfstring{Diferencia con
\texttt{\_metadata.yml}}{Diferencia con \_metadata.yml}}\label{diferencia-con-_metadata.yml}

\begin{longtable}[]{@{}
  >{\raggedright\arraybackslash}p{(\linewidth - 6\tabcolsep) * \real{0.2045}}
  >{\raggedright\arraybackslash}p{(\linewidth - 6\tabcolsep) * \real{0.2045}}
  >{\raggedright\arraybackslash}p{(\linewidth - 6\tabcolsep) * \real{0.3409}}
  >{\raggedright\arraybackslash}p{(\linewidth - 6\tabcolsep) * \real{0.2500}}@{}}
\toprule\noalign{}
\begin{minipage}[b]{\linewidth}\raggedright
Archivo
\end{minipage} & \begin{minipage}[b]{\linewidth}\raggedright
Alcance
\end{minipage} & \begin{minipage}[b]{\linewidth}\raggedright
Uso Principal
\end{minipage} & \begin{minipage}[b]{\linewidth}\raggedright
Ubicación
\end{minipage} \\
\midrule\noalign{}
\endhead
\bottomrule\noalign{}
\endlastfoot
\texttt{\_quarto.yml} & Todo el proyecto & Configuración del sitio,
navbar, footer, tema & Raíz del proyecto \\
\texttt{\_metadata.yml} & Directorio específico & Opciones por defecto
para posts & Dentro de directorios \\
\end{longtable}

\textbf{Ejemplo de relación:}

\begin{Shaded}
\begin{Highlighting}[]
\CommentTok{\# \_quarto.yml (global para }\AlertTok{TODO}\CommentTok{ el sitio)}
\FunctionTok{website}\KeywordTok{:}
\AttributeTok{  }\FunctionTok{title}\KeywordTok{:}\AttributeTok{ }\StringTok{"Mi Sitio"}
\AttributeTok{  }\FunctionTok{navbar}\KeywordTok{:}
\AttributeTok{    }\FunctionTok{left}\KeywordTok{:}
\AttributeTok{      }\KeywordTok{{-}}\AttributeTok{ }\FunctionTok{text}\KeywordTok{:}\AttributeTok{ }\StringTok{"Blog"}
\AttributeTok{        }\FunctionTok{href}\KeywordTok{:}\AttributeTok{ blog/}

\CommentTok{\# blog/\_metadata.yml (solo para posts del blog)}
\FunctionTok{author}\KeywordTok{:}
\AttributeTok{  }\KeywordTok{{-}}\AttributeTok{ }\FunctionTok{name}\KeywordTok{:}\AttributeTok{ Edison Achalma}
\FunctionTok{date{-}modified}\KeywordTok{:}\AttributeTok{ }\StringTok{"today"}
\end{Highlighting}
\end{Shaded}

\subsection{Propósito y Funciones
Principales}\label{propuxf3sito-y-funciones-principales}

\textbf{\texttt{\_quarto.yml} controla:}

\begin{enumerate}
\def\labelenumi{\arabic{enumi}.}
\tightlist
\item
  \textbf{Tipo de proyecto} (website, book, default)
\item
  \textbf{Estructura del sitio} (navegación, footer, sidebar)
\item
  \textbf{Apariencia global} (tema, CSS, fuentes)
\item
  \textbf{Metadatos del sitio} (título, descripción, SEO)
\item
  \textbf{Herramientas externas} (Google Analytics, comentarios)
\item
  \textbf{Formatos de salida} (HTML, PDF, DOCX)
\item
  \textbf{Recursos compartidos} (imágenes, archivos estáticos)
\end{enumerate}

\section{Jerarquía de
Configuración}\label{jerarquuxeda-de-configuraciuxf3n}

\subsection{Orden de Prioridad}\label{orden-de-prioridad}

\begin{verbatim}
1. Front-matter del documento (index.qmd)
   └── Mayor prioridad
   
2. _metadata.yml (directorio del documento)
   
3. _metadata.yml (directorio padre)
   
4. _quarto.yml (raíz del proyecto)
   
5. Valores por defecto de Quarto
   └── Menor prioridad
\end{verbatim}

\textbf{Ejemplo práctico:}

\begin{Shaded}
\begin{Highlighting}[]
\CommentTok{\# \_quarto.yml (global)}
\FunctionTok{format}\KeywordTok{:}
\AttributeTok{  }\FunctionTok{html}\KeywordTok{:}
\AttributeTok{    }\FunctionTok{theme}\KeywordTok{:}\AttributeTok{ cosmo}
\AttributeTok{    }\FunctionTok{toc}\KeywordTok{:}\AttributeTok{ }\CharTok{true}

\CommentTok{\# posts/\_metadata.yml (categoría)}
\FunctionTok{format}\KeywordTok{:}
\AttributeTok{  }\FunctionTok{html}\KeywordTok{:}
\AttributeTok{    }\FunctionTok{toc{-}depth}\KeywordTok{:}\AttributeTok{ }\DecValTok{3}

\CommentTok{\# posts/mi{-}post/index.qmd (documento)}
\PreprocessorTok{{-}{-}{-}}
\FunctionTok{format}\KeywordTok{:}
\AttributeTok{  }\FunctionTok{html}\KeywordTok{:}
\AttributeTok{    }\FunctionTok{toc}\KeywordTok{:}\AttributeTok{ }\CharTok{false}
\PreprocessorTok{{-}{-}{-}}
\end{Highlighting}
\end{Shaded}

\textbf{Resultado final para \texttt{mi-post/index.qmd}:}

\begin{itemize}
\tightlist
\item
  \texttt{toc:\ false} (del documento - mayor prioridad)
\item
  \texttt{theme:\ cosmo} (de \_quarto.yml - heredado)
\item
  \texttt{toc-depth:\ 3} (de \_metadata.yml - no se usa porque toc está
  desactivado)
\end{itemize}

\subsection{Herencia y Combinación}\label{herencia-y-combinaciuxf3n}

\textbf{Regla:} Las configuraciones se \textbf{combinan} cuando son
diferentes claves, pero se \textbf{sobrescriben} cuando son la misma
clave.

\begin{Shaded}
\begin{Highlighting}[]
\CommentTok{\# \_quarto.yml}
\FunctionTok{format}\KeywordTok{:}
\AttributeTok{  }\FunctionTok{html}\KeywordTok{:}
\AttributeTok{    }\FunctionTok{theme}\KeywordTok{:}\AttributeTok{ cosmo}
\AttributeTok{    }\FunctionTok{toc}\KeywordTok{:}\AttributeTok{ }\CharTok{true}
\AttributeTok{    }\FunctionTok{code{-}fold}\KeywordTok{:}\AttributeTok{ }\CharTok{true}

\CommentTok{\# \_metadata.yml}
\FunctionTok{format}\KeywordTok{:}
\AttributeTok{  }\FunctionTok{html}\KeywordTok{:}
\AttributeTok{    }\FunctionTok{toc}\KeywordTok{:}\AttributeTok{ }\CharTok{false}\CommentTok{        \# Sobrescribe toc}
\AttributeTok{    }\FunctionTok{code{-}copy}\KeywordTok{:}\AttributeTok{ }\CharTok{true}\CommentTok{   \# Se añade (nueva clave)}
\CommentTok{    \# theme y code{-}fold se heredan}
\end{Highlighting}
\end{Shaded}

\textbf{Resultado combinado:}

\begin{Shaded}
\begin{Highlighting}[]
\FunctionTok{format}\KeywordTok{:}
\AttributeTok{  }\FunctionTok{html}\KeywordTok{:}
\AttributeTok{    }\FunctionTok{theme}\KeywordTok{:}\AttributeTok{ cosmo}\CommentTok{          \# Heredado de \_quarto.yml}
\AttributeTok{    }\FunctionTok{toc}\KeywordTok{:}\AttributeTok{ }\CharTok{false}\CommentTok{            \# Sobrescrito por \_metadata.yml}
\AttributeTok{    }\FunctionTok{code{-}fold}\KeywordTok{:}\AttributeTok{ }\CharTok{true}\CommentTok{       \# Heredado de \_quarto.yml}
\AttributeTok{    }\FunctionTok{code{-}copy}\KeywordTok{:}\AttributeTok{ }\CharTok{true}\CommentTok{       \# Añadido por \_metadata.yml}
\end{Highlighting}
\end{Shaded}

\section{Estructura del Archivo}\label{estructura-del-archivo}

\subsection{\texorpdfstring{Anatomía de
\texttt{\_quarto.yml}}{Anatomía de \_quarto.yml}}\label{anatomuxeda-de-_quarto.yml}

\begin{Shaded}
\begin{Highlighting}[]
\CommentTok{\# ========================================================================}
\CommentTok{\# CONFIGURACIÓN DE PROYECTO}
\CommentTok{\# ========================================================================}
\FunctionTok{project}\KeywordTok{:}
\AttributeTok{  }\FunctionTok{type}\KeywordTok{:}\AttributeTok{ website}
\AttributeTok{  }\FunctionTok{output{-}dir}\KeywordTok{:}\AttributeTok{ \_site}
\AttributeTok{  }\FunctionTok{resources}\KeywordTok{:}\AttributeTok{ }\KeywordTok{[]}

\CommentTok{\# ========================================================================}
\CommentTok{\# CONFIGURACIÓN DEL SITIO WEB}
\CommentTok{\# ========================================================================}
\FunctionTok{website}\KeywordTok{:}
\CommentTok{  \# Información básica}
\AttributeTok{  }\FunctionTok{title}\KeywordTok{:}\AttributeTok{ }\StringTok{""}
\AttributeTok{  }\FunctionTok{description}\KeywordTok{:}\AttributeTok{ }\StringTok{""}
\AttributeTok{  }
\CommentTok{  \# Navegación}
\AttributeTok{  }\FunctionTok{navbar}\KeywordTok{:}\AttributeTok{ }\KeywordTok{\{\}}
\AttributeTok{  }\FunctionTok{sidebar}\KeywordTok{:}\AttributeTok{ }\KeywordTok{[]}
\AttributeTok{  }
\CommentTok{  \# Footer}
\AttributeTok{  }\FunctionTok{page{-}footer}\KeywordTok{:}\AttributeTok{ }\KeywordTok{\{\}}
\AttributeTok{  }
\CommentTok{  \# SEO y metadatos}
\AttributeTok{  }\FunctionTok{site{-}url}\KeywordTok{:}\AttributeTok{ }\StringTok{""}
\AttributeTok{  }\FunctionTok{favicon}\KeywordTok{:}\AttributeTok{ }\StringTok{""}
\AttributeTok{  }\FunctionTok{open{-}graph}\KeywordTok{:}\AttributeTok{ }\KeywordTok{\{\}}
\AttributeTok{  }\FunctionTok{twitter{-}card}\KeywordTok{:}\AttributeTok{ }\KeywordTok{\{\}}
\AttributeTok{  }
\CommentTok{  \# Herramientas}
\AttributeTok{  }\FunctionTok{google{-}analytics}\KeywordTok{:}\AttributeTok{ }\KeywordTok{\{\}}
\AttributeTok{  }\FunctionTok{cookie{-}consent}\KeywordTok{:}\AttributeTok{ }\KeywordTok{\{\}}
\AttributeTok{  }\FunctionTok{comments}\KeywordTok{:}\AttributeTok{ }\KeywordTok{\{\}}

\CommentTok{\# ========================================================================}
\CommentTok{\# FORMATOS DE SALIDA}
\CommentTok{\# ========================================================================}
\FunctionTok{format}\KeywordTok{:}
\AttributeTok{  }\FunctionTok{html}\KeywordTok{:}\AttributeTok{ }\KeywordTok{\{\}}
\AttributeTok{  }\FunctionTok{pdf}\KeywordTok{:}\AttributeTok{ }\KeywordTok{\{\}}
\AttributeTok{  }\FunctionTok{docx}\KeywordTok{:}\AttributeTok{ }\KeywordTok{\{\}}

\CommentTok{\# ========================================================================}
\CommentTok{\# OPCIONES AVANZADAS}
\CommentTok{\# ========================================================================}
\CommentTok{\# Bibliografía global, filtros, extensiones, etc.}
\end{Highlighting}
\end{Shaded}

\subsection{Secciones Principales}\label{secciones-principales}

\begin{longtable}[]{@{}
  >{\raggedright\arraybackslash}p{(\linewidth - 4\tabcolsep) * \real{0.2727}}
  >{\raggedright\arraybackslash}p{(\linewidth - 4\tabcolsep) * \real{0.3333}}
  >{\raggedright\arraybackslash}p{(\linewidth - 4\tabcolsep) * \real{0.3939}}@{}}
\toprule\noalign{}
\begin{minipage}[b]{\linewidth}\raggedright
Sección
\end{minipage} & \begin{minipage}[b]{\linewidth}\raggedright
Propósito
\end{minipage} & \begin{minipage}[b]{\linewidth}\raggedright
Obligatoria
\end{minipage} \\
\midrule\noalign{}
\endhead
\bottomrule\noalign{}
\endlastfoot
\texttt{project} & Define tipo y estructura del proyecto & Sí \\
\texttt{website} & Configuración específica de sitios web & Sí (para
websites) \\
\texttt{format} & Formatos de salida y sus opciones & Recomendada \\
\texttt{editor} & Configuración del editor & Opcional \\
\texttt{execute} & Opciones de ejecución de código & Opcional \\
\end{longtable}

\section{Sección: Project}\label{secciuxf3n-project}

Esta sección define las características fundamentales del proyecto.

\subsection{Opciones Básicas}\label{opciones-buxe1sicas}

\begin{Shaded}
\begin{Highlighting}[]
\FunctionTok{project}\KeywordTok{:}
\AttributeTok{  }\FunctionTok{type}\KeywordTok{:}\AttributeTok{ website}\CommentTok{              \# Tipo de proyecto}
\AttributeTok{  }\FunctionTok{output{-}dir}\KeywordTok{:}\AttributeTok{ \_site}\CommentTok{          \# Directorio de salida}
\end{Highlighting}
\end{Shaded}

\subsection{\texorpdfstring{\texttt{type}: Tipo de
Proyecto}{type: Tipo de Proyecto}}\label{type-tipo-de-proyecto}

\textbf{Opciones disponibles:}

\begin{longtable}[]{@{}
  >{\raggedright\arraybackslash}p{(\linewidth - 4\tabcolsep) * \real{0.2800}}
  >{\raggedright\arraybackslash}p{(\linewidth - 4\tabcolsep) * \real{0.5200}}
  >{\raggedright\arraybackslash}p{(\linewidth - 4\tabcolsep) * \real{0.2000}}@{}}
\toprule\noalign{}
\begin{minipage}[b]{\linewidth}\raggedright
Valor
\end{minipage} & \begin{minipage}[b]{\linewidth}\raggedright
Descripción
\end{minipage} & \begin{minipage}[b]{\linewidth}\raggedright
Uso
\end{minipage} \\
\midrule\noalign{}
\endhead
\bottomrule\noalign{}
\endlastfoot
\texttt{website} & Sitio web con múltiples páginas & Blogs, sitios
académicos, portfolios \\
\texttt{book} & Libro con capítulos & Documentación técnica, libros
académicos \\
\texttt{default} & Proyecto simple sin estructura especial & Documentos
individuales \\
\end{longtable}

\textbf{Ejemplo para cada tipo:}

\begin{Shaded}
\begin{Highlighting}[]
\CommentTok{\# Sitio web}
\FunctionTok{project}\KeywordTok{:}
\AttributeTok{  }\FunctionTok{type}\KeywordTok{:}\AttributeTok{ website}

\CommentTok{\# Libro}
\FunctionTok{project}\KeywordTok{:}
\AttributeTok{  }\FunctionTok{type}\KeywordTok{:}\AttributeTok{ book}
\AttributeTok{  }
\CommentTok{\# Proyecto simple}
\FunctionTok{project}\KeywordTok{:}
\AttributeTok{  }\FunctionTok{type}\KeywordTok{:}\AttributeTok{ default}
\end{Highlighting}
\end{Shaded}

\subsection{\texorpdfstring{\texttt{output-dir}: Directorio de
Salida}{output-dir: Directorio de Salida}}\label{output-dir-directorio-de-salida}

Define dónde se generan los archivos compilados.

\begin{Shaded}
\begin{Highlighting}[]
\FunctionTok{project}\KeywordTok{:}
\AttributeTok{  }\FunctionTok{output{-}dir}\KeywordTok{:}\AttributeTok{ \_site}\CommentTok{          \# Directorio por defecto}

\CommentTok{\# Alternativas comunes:}
\CommentTok{\# output{-}dir: docs           \# Para GitHub Pages con /docs}
\CommentTok{\# output{-}dir: public         \# Convención de algunos frameworks}
\CommentTok{\# output{-}dir: dist           \# Build/distribución}
\end{Highlighting}
\end{Shaded}

\textbf{Importante:} Si cambias \texttt{output-dir}, asegúrate de
actualizar: - \texttt{.gitignore} (agregar el nuevo directorio) -
Configuración de despliegue (Netlify, GitHub Pages, etc.)

\subsection{\texorpdfstring{\texttt{resources}: Archivos
Adicionales}{resources: Archivos Adicionales}}\label{resources-archivos-adicionales}

\textbf{Site Resources} permite asegurar que archivos adicionales (como
imágenes, documentos, datos, configuraciones especiales, etc.) se copien
automáticamente al directorio de salida del sitio (\texttt{\_site})
durante el renderizado. Quarto ya copia de forma automática los archivos
referenciados en tus documentos (imágenes, CSS, etc.), pero con esta
opción puedes forzar la inclusión de archivos no referenciados
directamente.

\begin{Shaded}
\begin{Highlighting}[]
\FunctionTok{project}\KeywordTok{:}
\AttributeTok{  }\FunctionTok{resources}\KeywordTok{:}
\AttributeTok{    }\KeywordTok{{-}}\AttributeTok{ }\StringTok{"assets/img/sidebar.jpg"}\CommentTok{      \# Archivo individual}
\AttributeTok{    }\KeywordTok{{-}}\AttributeTok{ }\StringTok{"data/"}\CommentTok{                       \# Directorio completo}
\AttributeTok{    }\KeywordTok{{-}}\AttributeTok{ }\StringTok{"*.pdf"}\CommentTok{                       \# Patrón glob}
\AttributeTok{    }\KeywordTok{{-}}\AttributeTok{ }\StringTok{"CNAME"}\CommentTok{                       \# Dominio personalizado (GitHub Pages)}
\AttributeTok{    }\KeywordTok{{-}}\AttributeTok{ }\StringTok{".nojekyll"}\CommentTok{                   \# Desactivar Jekyll (GitHub Pages)}
\end{Highlighting}
\end{Shaded}

\textbf{Cuándo usar \texttt{resources}:}

\begin{itemize}
\tightlist
\item
  Archivos no referenciados directamente en documentos
\item
  Archivos de configuración de hosting (CNAME, robots.txt)
\item
  Datasets descargables
\item
  Documentos PDF, Excel, etc.
\end{itemize}

\textbf{Archivos copiados automáticamente (sin necesidad de
declararlos):}

\begin{itemize}
\tightlist
\item
  \texttt{404.html}
\item
  \texttt{robots.txt}
\item
  \texttt{\_redirects} (Netlify)
\item
  \texttt{CNAME} (GitHub Pages)
\item
  \texttt{.nojekyll} (GitHub Pages)
\end{itemize}

\subsubsection{Ejemplos prácticos}\label{ejemplos-pruxe1cticos}

\paragraph{1. Sitio con datos
descargables.}\label{sitio-con-datos-descargables}

\begin{Shaded}
\begin{Highlighting}[]
\FunctionTok{project}\KeywordTok{:}
\AttributeTok{  }\FunctionTok{type}\KeywordTok{:}\AttributeTok{ website}
\AttributeTok{  }\FunctionTok{resources}\KeywordTok{:}
\AttributeTok{    }\KeywordTok{{-}}\AttributeTok{ }\StringTok{"data/"}\CommentTok{                  \# Todos los datasets en carpeta data}
\AttributeTok{    }\KeywordTok{{-}}\AttributeTok{ }\StringTok{"downloads/*.zip"}\CommentTok{        \# Archivos comprimidos descargables}
\end{Highlighting}
\end{Shaded}

Los usuarios podrán acceder a
\texttt{https://tusitio.com/data/mi-datos.csv}.

\paragraph{2. Publicar en GitHub Pages con dominio
personalizado.}\label{publicar-en-github-pages-con-dominio-personalizado}

\begin{Shaded}
\begin{Highlighting}[]
\FunctionTok{project}\KeywordTok{:}
\AttributeTok{  }\FunctionTok{type}\KeywordTok{:}\AttributeTok{ website}
\AttributeTok{  }\FunctionTok{resources}\KeywordTok{:}
\AttributeTok{    }\KeywordTok{{-}}\AttributeTok{ }\StringTok{"CNAME"}\CommentTok{                  \# Archivo con tu dominio (ej: misitio.com)}
\AttributeTok{    }\KeywordTok{{-}}\AttributeTok{ }\StringTok{".nojekyll"}\CommentTok{              \# Evita que GitHub procese con Jekyll}
\end{Highlighting}
\end{Shaded}

\paragraph{3. Incluir documentos PDF o Excel para
descarga.}\label{incluir-documentos-pdf-o-excel-para-descarga}

\begin{Shaded}
\begin{Highlighting}[]
\FunctionTok{project}\KeywordTok{:}
\AttributeTok{  }\FunctionTok{type}\KeywordTok{:}\AttributeTok{ website}
\AttributeTok{  }\FunctionTok{resources}\KeywordTok{:}
\AttributeTok{    }\KeywordTok{{-}}\AttributeTok{ }\StringTok{"reports/*.pdf"}
\AttributeTok{    }\KeywordTok{{-}}\AttributeTok{ }\StringTok{"datasets/*.xlsx"}
\end{Highlighting}
\end{Shaded}

Y en una página:

\begin{Shaded}
\begin{Highlighting}[]
\NormalTok{Descarga el }\CommentTok{[}\OtherTok{reporte completo}\CommentTok{](/reports/informe{-}2025.pdf)}\NormalTok{.}
\end{Highlighting}
\end{Shaded}

\paragraph{4. Recursos solo en una página
específica.}\label{recursos-solo-en-una-puxe1gina-especuxedfica}

\begin{Shaded}
\begin{Highlighting}[]
\CommentTok{\# informe.qmd}
\PreprocessorTok{{-}{-}{-}}
\FunctionTok{title}\KeywordTok{:}\AttributeTok{ }\StringTok{"Informe Anual 2025"}
\FunctionTok{resources}\KeywordTok{:}
\AttributeTok{  }\KeywordTok{{-}}\AttributeTok{ }\StringTok{"supplements/anexo{-}datos.xlsx"}
\AttributeTok{  }\KeywordTok{{-}}\AttributeTok{ }\StringTok{"supplements/graficos{-}extra.pdf"}
\PreprocessorTok{{-}{-}{-}}
\end{Highlighting}
\end{Shaded}

Solo estas páginas tendrán los archivos adicionales copiados.

\subsection{\texorpdfstring{\texttt{pre-render} y \texttt{post-render}:
Scripts
Personalizados}{pre-render y post-render: Scripts Personalizados}}\label{pre-render-y-post-render-scripts-personalizados}

\textbf{Ejecutar scripts antes o después del renderizado:}

\begin{Shaded}
\begin{Highlighting}[]
\FunctionTok{project}\KeywordTok{:}
\AttributeTok{  }\FunctionTok{pre{-}render}\KeywordTok{:}\AttributeTok{ scripts/setup.sh}\CommentTok{         \# Ejecutar antes de renderizar}
\AttributeTok{  }\FunctionTok{post{-}render}\KeywordTok{:}\AttributeTok{ scripts/cleanup.sh}\CommentTok{      \# Ejecutar después de renderizar}
\end{Highlighting}
\end{Shaded}

\textbf{Ejemplo de \texttt{pre-render} (preparar datos):}

\begin{Shaded}
\begin{Highlighting}[]
\CommentTok{\#!/bin/bash}
\CommentTok{\# scripts/setup.sh}

\BuiltInTok{echo} \StringTok{"Preparando datos..."}
\ExtensionTok{python}\NormalTok{ scripts/fetch\_data.py}
\BuiltInTok{echo} \StringTok{"Datos listos."}
\end{Highlighting}
\end{Shaded}

\textbf{Ejemplo de \texttt{post-render} (optimizar imágenes):}

\begin{Shaded}
\begin{Highlighting}[]
\CommentTok{\#!/bin/bash}
\CommentTok{\# scripts/cleanup.sh}

\BuiltInTok{echo} \StringTok{"Optimizando imágenes..."}
\FunctionTok{find}\NormalTok{ \_site }\AttributeTok{{-}name} \StringTok{"*.png"} \AttributeTok{{-}exec}\NormalTok{ optipng \{\} }\DataTypeTok{\textbackslash{};}
\BuiltInTok{echo} \StringTok{"Optimización completa."}
\end{Highlighting}
\end{Shaded}

\subsection{\texorpdfstring{\texttt{render}: Control de
Renderizado}{render: Control de Renderizado}}\label{render-control-de-renderizado}

\textbf{Especificar qué archivos renderizar:}

\begin{Shaded}
\begin{Highlighting}[]
\FunctionTok{project}\KeywordTok{:}
\AttributeTok{  }\FunctionTok{render}\KeywordTok{:}
\AttributeTok{    }\KeywordTok{{-}}\AttributeTok{ }\StringTok{"*.qmd"}\CommentTok{                    \# Todos los .qmd}
\AttributeTok{    }\KeywordTok{{-}}\AttributeTok{ }\StringTok{"!drafts/"}\CommentTok{                 \# Excepto directorio drafts/}
\AttributeTok{    }\KeywordTok{{-}}\AttributeTok{ }\StringTok{"!templates/"}\CommentTok{              \# Excepto templates/}
\end{Highlighting}
\end{Shaded}

\textbf{Casos de uso:}

\begin{itemize}
\tightlist
\item
  Excluir directorios de trabajo en progreso
\item
  Renderizar solo archivos específicos durante desarrollo
\item
  Ignorar templates o plantillas
\end{itemize}

\section{Sección: Website}\label{secciuxf3n-website}

Esta sección contiene toda la configuración específica de sitios web.

\subsection{Información Básica}\label{informaciuxf3n-buxe1sica}

\begin{Shaded}
\begin{Highlighting}[]
\FunctionTok{website}\KeywordTok{:}
\AttributeTok{  }\FunctionTok{title}\KeywordTok{:}\AttributeTok{ }\StringTok{"Edison Achalma B.Sc. Econ."}
\AttributeTok{  }\FunctionTok{description}\KeywordTok{:}\AttributeTok{ }\StringTok{"Investigador y educador que aplica la ciencia de datos }\SpecialCharTok{\textbackslash{} }
\StringTok{  de forma que se dé prioridad a la equidad social."}
\AttributeTok{  }\FunctionTok{site{-}url}\KeywordTok{:}\AttributeTok{ https://methodica.netlify.app/}
\AttributeTok{  }\FunctionTok{repo{-}url}\KeywordTok{:}\AttributeTok{ https://github.com/achalmed/methodica}
\end{Highlighting}
\end{Shaded}

\subsubsection{\texorpdfstring{\texttt{title}: Título del
Sitio}{title: Título del Sitio}}\label{title-tuxedtulo-del-sitio}

\begin{Shaded}
\begin{Highlighting}[]
\FunctionTok{website}\KeywordTok{:}
\AttributeTok{  }\FunctionTok{title}\KeywordTok{:}\AttributeTok{ }\StringTok{"Mi Sitio Web"}
\CommentTok{  \# O con subtítulo:}
\AttributeTok{  }\FunctionTok{title}\KeywordTok{:}\AttributeTok{ }\StringTok{"Mi Sitio Web"}
\AttributeTok{  }\FunctionTok{subtitle}\KeywordTok{:}\AttributeTok{ }\StringTok{"Un blog sobre econometría"}
\end{Highlighting}
\end{Shaded}

\begin{itemize}
\tightlist
\item
  Aparece en la navbar (si no se especifica logo)
\item
  Se usa como título por defecto en Open Graph
\item
  Máximo recomendado: 60 caracteres
\end{itemize}

\subsubsection{\texorpdfstring{\texttt{description}: Descripción del
Sitio}{description: Descripción del Sitio}}\label{description-descripciuxf3n-del-sitio}

\begin{Shaded}
\begin{Highlighting}[]
\FunctionTok{website}\KeywordTok{:}
\AttributeTok{  }\FunctionTok{description}\KeywordTok{:}\AttributeTok{ }\StringTok{"Sitio web académico sobre econometría y estadística"}
\end{Highlighting}
\end{Shaded}

\begin{itemize}
\tightlist
\item
  Usada en metadatos SEO
\item
  Aparece en resultados de búsqueda
\item
  Recomendado: 150-160 caracteres
\end{itemize}

\subsubsection{\texorpdfstring{\texttt{site-url}: URL del
Sitio}{site-url: URL del Sitio}}\label{site-url-url-del-sitio}

\begin{Shaded}
\begin{Highlighting}[]
\FunctionTok{website}\KeywordTok{:}
\AttributeTok{  }\FunctionTok{site{-}url}\KeywordTok{:}\AttributeTok{ https://misitio.com}
\CommentTok{  \# NO incluir trailing slash al final}
\end{Highlighting}
\end{Shaded}

\textbf{Obligatorio para:}

\begin{itemize}
\tightlist
\item
  Open Graph y Twitter Cards (imágenes de vista previa)
\item
  RSS feeds
\item
  Sitemap.xml
\item
  Canonical URLs
\end{itemize}

\textbf{Sin \texttt{site-url}:} Las rutas relativas en Open Graph no
funcionarán correctamente.

\subsubsection{\texorpdfstring{\texttt{repo-url}: Repositorio de
Código}{repo-url: Repositorio de Código}}\label{repo-url-repositorio-de-cuxf3digo}

\begin{Shaded}
\begin{Highlighting}[]
\FunctionTok{website}\KeywordTok{:}
\AttributeTok{  }\FunctionTok{repo{-}url}\KeywordTok{:}\AttributeTok{ https://github.com/usuario/repositorio}
\end{Highlighting}
\end{Shaded}

\textbf{Beneficios:}

\begin{itemize}
\tightlist
\item
  Botón automático ``Edit this page'' en cada página
\item
  Enlace al repositorio en la navbar (con \texttt{repo-actions})
\end{itemize}

\begin{Shaded}
\begin{Highlighting}[]
\FunctionTok{website}\KeywordTok{:}
\AttributeTok{  }\FunctionTok{repo{-}url}\KeywordTok{:}\AttributeTok{ https://github.com/usuario/repo}
\AttributeTok{  }\FunctionTok{repo{-}actions}\KeywordTok{:}\AttributeTok{ }\KeywordTok{[}\AttributeTok{edit}\KeywordTok{,}\AttributeTok{ issue}\KeywordTok{]}\CommentTok{    \# Botones de editar y reportar issue}
\end{Highlighting}
\end{Shaded}

\subsection{Favicon e Imágenes}\label{favicon-e-imuxe1genes}

El \textbf{Favicon} es el pequeño ícono que aparece en la pestaña del
navegador, en los marcadores y en otros lugares donde se representa tu
sitio web. Configurarlo mejora la identidad visual y el reconocimiento
de tu sitio.

\subsubsection{\texorpdfstring{\texttt{favicon}: Ícono del
Sitio}{favicon: Ícono del Sitio}}\label{favicon-uxedcono-del-sitio}

\begin{Shaded}
\begin{Highlighting}[]
\FunctionTok{website}\KeywordTok{:}
\AttributeTok{  }\FunctionTok{favicon}\KeywordTok{:}\AttributeTok{ assets/img/favicon.png}
\CommentTok{  \# O:}
\CommentTok{  \# favicon: favicon.ico}
\end{Highlighting}
\end{Shaded}

\textbf{Formatos recomendados:}

\begin{itemize}
\tightlist
\item
  \texttt{.ico} (máxima compatibilidad)
\item
  \texttt{.png} (calidad, transparencia)
\item
  \texttt{.svg} (escalable)
\end{itemize}

\textbf{Tamaño recomendado:} 32x32 px mínimo, 64x64 px ideal.

\subsubsection{\texorpdfstring{\texttt{image}: Imagen de Vista Previa
por
Defecto}{image: Imagen de Vista Previa por Defecto}}\label{image-imagen-de-vista-previa-por-defecto}

\begin{Shaded}
\begin{Highlighting}[]
\FunctionTok{website}\KeywordTok{:}
\AttributeTok{  }\FunctionTok{image}\KeywordTok{:}\AttributeTok{ /assets/img/default{-}preview.jpg}
\end{Highlighting}
\end{Shaded}

\textbf{Usada cuando:}

\begin{itemize}
\tightlist
\item
  Una página no tiene su propia imagen especificada
\item
  Se comparte un enlace en redes sociales
\item
  Open Graph/Twitter Cards necesitan una imagen
\end{itemize}

\textbf{Especificaciones:}

\begin{itemize}
\tightlist
\item
  Tamaño ideal: 1200x630 px (ratio 1.91:1)
\item
  Formato: JPG o PNG
\item
  Máximo: 8 MB
\end{itemize}

\subsection{Navbar: Barra de
Navegación}\label{navbar-barra-de-navegaciuxf3n}

La navbar es la barra superior del sitio.

\subsubsection{Estructura Básica}\label{estructura-buxe1sica}

\begin{Shaded}
\begin{Highlighting}[]
\FunctionTok{website}\KeywordTok{:}
\AttributeTok{  }\FunctionTok{navbar}\KeywordTok{:}
\AttributeTok{    }\FunctionTok{title}\KeywordTok{:}\AttributeTok{ }\StringTok{"Mi Sitio"}\CommentTok{           \# Título en la navbar (opcional)}
\AttributeTok{    }\FunctionTok{logo}\KeywordTok{:}\AttributeTok{ assets/img/logo.png}\CommentTok{   \# Logo (opcional)}
\AttributeTok{    }\FunctionTok{background}\KeywordTok{:}\AttributeTok{ primary}\CommentTok{         \# Color de fondo}
\AttributeTok{    }\FunctionTok{foreground}\KeywordTok{:}\AttributeTok{ light}\CommentTok{           \# Color del texto}
\AttributeTok{    }\FunctionTok{left}\KeywordTok{:}\CommentTok{                       \# Items del lado izquierdo}
\AttributeTok{      }\KeywordTok{{-}}\AttributeTok{ }\FunctionTok{text}\KeywordTok{:}\AttributeTok{ }\StringTok{"Inicio"}
\AttributeTok{        }\FunctionTok{href}\KeywordTok{:}\AttributeTok{ index.qmd}
\AttributeTok{    }\FunctionTok{right}\KeywordTok{:}\CommentTok{                      \# Items del lado derecho}
\AttributeTok{      }\KeywordTok{{-}}\AttributeTok{ }\FunctionTok{text}\KeywordTok{:}\AttributeTok{ }\StringTok{"About"}
\AttributeTok{        }\FunctionTok{href}\KeywordTok{:}\AttributeTok{ about.qmd}
\end{Highlighting}
\end{Shaded}

\subsubsection{Items de Navegación}\label{items-de-navegaciuxf3n}

\textbf{Enlace simple:}

\begin{Shaded}
\begin{Highlighting}[]
\FunctionTok{navbar}\KeywordTok{:}
\AttributeTok{  }\FunctionTok{left}\KeywordTok{:}
\AttributeTok{    }\KeywordTok{{-}}\AttributeTok{ }\FunctionTok{text}\KeywordTok{:}\AttributeTok{ }\StringTok{"Blog"}
\AttributeTok{      }\FunctionTok{href}\KeywordTok{:}\AttributeTok{ blog/}
\end{Highlighting}
\end{Shaded}

\textbf{Enlace con ícono:}

\begin{Shaded}
\begin{Highlighting}[]
\FunctionTok{navbar}\KeywordTok{:}
\AttributeTok{  }\FunctionTok{right}\KeywordTok{:}
\AttributeTok{    }\KeywordTok{{-}}\AttributeTok{ }\FunctionTok{icon}\KeywordTok{:}\AttributeTok{ github}
\AttributeTok{      }\FunctionTok{href}\KeywordTok{:}\AttributeTok{ https://github.com/usuario}
\AttributeTok{      }\FunctionTok{aria{-}label}\KeywordTok{:}\AttributeTok{ }\StringTok{"GitHub"}
\end{Highlighting}
\end{Shaded}

\textbf{Menú desplegable:}

\begin{Shaded}
\begin{Highlighting}[]
\FunctionTok{navbar}\KeywordTok{:}
\AttributeTok{  }\FunctionTok{right}\KeywordTok{:}
\AttributeTok{    }\KeywordTok{{-}}\AttributeTok{ }\FunctionTok{text}\KeywordTok{:}\AttributeTok{ }\StringTok{"Recursos"}
\AttributeTok{      }\FunctionTok{menu}\KeywordTok{:}
\AttributeTok{        }\KeywordTok{{-}}\AttributeTok{ }\FunctionTok{text}\KeywordTok{:}\AttributeTok{ }\StringTok{"Tutoriales"}
\AttributeTok{          }\FunctionTok{href}\KeywordTok{:}\AttributeTok{ tutoriales/}
\AttributeTok{        }\KeywordTok{{-}}\AttributeTok{ }\FunctionTok{text}\KeywordTok{:}\AttributeTok{ }\StringTok{"Datasets"}
\AttributeTok{          }\FunctionTok{href}\KeywordTok{:}\AttributeTok{ datasets/}
\AttributeTok{        }\KeywordTok{{-}}\AttributeTok{ }\FunctionTok{text}\KeywordTok{:}\AttributeTok{ }\StringTok{"{-}{-}{-}"}\CommentTok{              \# Separador}
\AttributeTok{        }\KeywordTok{{-}}\AttributeTok{ }\FunctionTok{text}\KeywordTok{:}\AttributeTok{ }\StringTok{"Documentación"}
\AttributeTok{          }\FunctionTok{href}\KeywordTok{:}\AttributeTok{ docs/}
\end{Highlighting}
\end{Shaded}

\textbf{Menú con íconos:}

\begin{Shaded}
\begin{Highlighting}[]
\FunctionTok{navbar}\KeywordTok{:}
\AttributeTok{  }\FunctionTok{right}\KeywordTok{:}
\AttributeTok{    }\KeywordTok{{-}}\AttributeTok{ }\FunctionTok{text}\KeywordTok{:}\AttributeTok{ }\StringTok{"More"}
\AttributeTok{      }\FunctionTok{icon}\KeywordTok{:}\AttributeTok{ ellipsis{-}h}
\AttributeTok{      }\FunctionTok{menu}\KeywordTok{:}
\AttributeTok{        }\KeywordTok{{-}}\AttributeTok{ }\FunctionTok{text}\KeywordTok{:}\AttributeTok{ }\StringTok{"Econometría"}
\AttributeTok{          }\FunctionTok{href}\KeywordTok{:}\AttributeTok{ https://epsilon{-}y{-}beta.netlify.app/}
\AttributeTok{        }\KeywordTok{{-}}\AttributeTok{ }\FunctionTok{text}\KeywordTok{:}\AttributeTok{ }\StringTok{"Filosofía"}
\AttributeTok{          }\FunctionTok{href}\KeywordTok{:}\AttributeTok{ https://dialectica{-}y{-}mercado.netlify.app/}
\end{Highlighting}
\end{Shaded}

\subsubsection{Logo}\label{logo}

\begin{Shaded}
\begin{Highlighting}[]
\FunctionTok{navbar}\KeywordTok{:}
\AttributeTok{  }\FunctionTok{logo}\KeywordTok{:}\AttributeTok{ assets/img/logo.png}
\AttributeTok{  }\FunctionTok{logo{-}alt}\KeywordTok{:}\AttributeTok{ }\StringTok{"Logo del sitio"}\CommentTok{        \# Texto alternativo}
\AttributeTok{  }\FunctionTok{logo{-}href}\KeywordTok{:}\AttributeTok{ index.html}\CommentTok{              \# Enlace del logo (por defecto: home)}
\end{Highlighting}
\end{Shaded}

\textbf{Recomendaciones:}

\begin{itemize}
\tightlist
\item
  Tamaño: 40-50 px de alto
\item
  Formato: PNG con transparencia o SVG
\item
  Si usas logo, el título se oculta automáticamente (a menos que uses
  \texttt{title:\ false})
\end{itemize}

\subsubsection{Colores y Apariencia}\label{colores-y-apariencia}

\begin{Shaded}
\begin{Highlighting}[]
\FunctionTok{navbar}\KeywordTok{:}
\AttributeTok{  }\FunctionTok{background}\KeywordTok{:}\AttributeTok{ primary}\CommentTok{                \# Color de fondo}
\AttributeTok{  }\FunctionTok{foreground}\KeywordTok{:}\AttributeTok{ light}\CommentTok{                  \# Color del texto}
\AttributeTok{  }
\CommentTok{  \# Opciones de background:}
\CommentTok{  \# primary, secondary, success, danger, warning, info, light, dark}
\CommentTok{  \# O código hexadecimal: "\#1a1a1a"}
\AttributeTok{  }
\CommentTok{  \# Opciones de foreground:}
\CommentTok{  \# light, dark}
\end{Highlighting}
\end{Shaded}

\subsubsection{Búsqueda}\label{buxfasqueda}

\begin{Shaded}
\begin{Highlighting}[]
\FunctionTok{navbar}\KeywordTok{:}
\AttributeTok{  }\FunctionTok{search}\KeywordTok{:}\AttributeTok{ }\CharTok{true}\CommentTok{                       \# Habilitar búsqueda}
\CommentTok{  \# O con opciones avanzadas:}
\AttributeTok{  }\FunctionTok{search}\KeywordTok{:}
\AttributeTok{    }\FunctionTok{location}\KeywordTok{:}\AttributeTok{ navbar}\CommentTok{                 \# navbar (por defecto) o sidebar}
\AttributeTok{    }\FunctionTok{type}\KeywordTok{:}\AttributeTok{ overlay}\CommentTok{                    \# overlay (por defecto) o textbox}
\end{Highlighting}
\end{Shaded}

\subsubsection{Collapse (Menú Móvil)}\label{collapse-menuxfa-muxf3vil}

\begin{Shaded}
\begin{Highlighting}[]
\FunctionTok{navbar}\KeywordTok{:}
\AttributeTok{  }\FunctionTok{collapse{-}below}\KeywordTok{:}\AttributeTok{ lg}\CommentTok{                 \# Colapsar en pantallas menores a lg}
\CommentTok{  \# Opciones: sm, md, lg, xl}
\end{Highlighting}
\end{Shaded}

\subsubsection{Ejemplo Completo de
Navbar}\label{ejemplo-completo-de-navbar}

\begin{Shaded}
\begin{Highlighting}[]
\FunctionTok{website}\KeywordTok{:}
\AttributeTok{  }\FunctionTok{navbar}\KeywordTok{:}
\AttributeTok{    }\FunctionTok{title}\KeywordTok{:}\AttributeTok{ }\StringTok{"Mi Sitio Académico"}
\AttributeTok{    }\FunctionTok{logo}\KeywordTok{:}\AttributeTok{ assets/img/logo.png}
\AttributeTok{    }\FunctionTok{background}\KeywordTok{:}\AttributeTok{ primary}
\AttributeTok{    }\FunctionTok{foreground}\KeywordTok{:}\AttributeTok{ light}
\AttributeTok{    }\FunctionTok{search}\KeywordTok{:}\AttributeTok{ }\CharTok{true}
\AttributeTok{    }\FunctionTok{collapse{-}below}\KeywordTok{:}\AttributeTok{ lg}
\AttributeTok{    }
\AttributeTok{    }\FunctionTok{left}\KeywordTok{:}
\AttributeTok{      }\KeywordTok{{-}}\AttributeTok{ }\FunctionTok{text}\KeywordTok{:}\AttributeTok{ }\StringTok{"Inicio"}
\AttributeTok{        }\FunctionTok{href}\KeywordTok{:}\AttributeTok{ index.qmd}
\AttributeTok{      }\KeywordTok{{-}}\AttributeTok{ }\FunctionTok{text}\KeywordTok{:}\AttributeTok{ }\StringTok{"Blog"}
\AttributeTok{        }\FunctionTok{href}\KeywordTok{:}\AttributeTok{ blog/}
\AttributeTok{      }\KeywordTok{{-}}\AttributeTok{ }\FunctionTok{text}\KeywordTok{:}\AttributeTok{ }\StringTok{"Publicaciones"}
\AttributeTok{        }\FunctionTok{href}\KeywordTok{:}\AttributeTok{ publications/}
\AttributeTok{        }
\AttributeTok{    }\FunctionTok{right}\KeywordTok{:}
\AttributeTok{      }\KeywordTok{{-}}\AttributeTok{ }\FunctionTok{text}\KeywordTok{:}\AttributeTok{ }\StringTok{"Recursos"}
\AttributeTok{        }\FunctionTok{menu}\KeywordTok{:}
\AttributeTok{          }\KeywordTok{{-}}\AttributeTok{ }\FunctionTok{text}\KeywordTok{:}\AttributeTok{ }\StringTok{"Tutoriales"}
\AttributeTok{            }\FunctionTok{href}\KeywordTok{:}\AttributeTok{ tutoriales/}
\AttributeTok{          }\KeywordTok{{-}}\AttributeTok{ }\FunctionTok{text}\KeywordTok{:}\AttributeTok{ }\StringTok{"Datasets"}
\AttributeTok{            }\FunctionTok{href}\KeywordTok{:}\AttributeTok{ datasets/}
\AttributeTok{          }\KeywordTok{{-}}\AttributeTok{ }\FunctionTok{text}\KeywordTok{:}\AttributeTok{ }\StringTok{"{-}{-}{-}"}
\AttributeTok{          }\KeywordTok{{-}}\AttributeTok{ }\FunctionTok{text}\KeywordTok{:}\AttributeTok{ }\StringTok{"Código"}
\AttributeTok{            }\FunctionTok{icon}\KeywordTok{:}\AttributeTok{ github}
\AttributeTok{            }\FunctionTok{href}\KeywordTok{:}\AttributeTok{ https://github.com/usuario}
\AttributeTok{      }\KeywordTok{{-}}\AttributeTok{ }\FunctionTok{icon}\KeywordTok{:}\AttributeTok{ github}
\AttributeTok{        }\FunctionTok{href}\KeywordTok{:}\AttributeTok{ https://github.com/usuario}
\AttributeTok{        }\FunctionTok{aria{-}label}\KeywordTok{:}\AttributeTok{ }\StringTok{"GitHub"}
\AttributeTok{      }\KeywordTok{{-}}\AttributeTok{ }\FunctionTok{icon}\KeywordTok{:}\AttributeTok{ twitter}
\AttributeTok{        }\FunctionTok{href}\KeywordTok{:}\AttributeTok{ https://twitter.com/usuario}
\AttributeTok{        }\FunctionTok{aria{-}label}\KeywordTok{:}\AttributeTok{ }\StringTok{"Twitter"}
\end{Highlighting}
\end{Shaded}

\subsection{Sidebar: Barra Lateral}\label{sidebar-barra-lateral}

La sidebar aparece en el lado izquierdo o derecho del contenido.

\subsubsection{Estructura Básica}\label{estructura-buxe1sica-1}

\begin{Shaded}
\begin{Highlighting}[]
\FunctionTok{website}\KeywordTok{:}
\AttributeTok{  }\FunctionTok{sidebar}\KeywordTok{:}
\AttributeTok{    }\FunctionTok{style}\KeywordTok{:}\AttributeTok{ }\StringTok{"docked"}\CommentTok{                  \# docked (fija) o floating (flotante)}
\AttributeTok{    }\FunctionTok{background}\KeywordTok{:}\AttributeTok{ light}\CommentTok{                \# Color de fondo}
\AttributeTok{    }\FunctionTok{collapse{-}level}\KeywordTok{:}\AttributeTok{ }\DecValTok{2}\CommentTok{                \# Nivel de colapso automático}
\AttributeTok{    }\FunctionTok{contents}\KeywordTok{:}
\AttributeTok{      }\KeywordTok{{-}}\AttributeTok{ }\FunctionTok{section}\KeywordTok{:}\AttributeTok{ }\StringTok{"Introducción"}
\AttributeTok{        }\FunctionTok{contents}\KeywordTok{:}
\AttributeTok{          }\KeywordTok{{-}}\AttributeTok{ intro.qmd}
\AttributeTok{          }\KeywordTok{{-}}\AttributeTok{ setup.qmd}
\AttributeTok{      }\KeywordTok{{-}}\AttributeTok{ }\FunctionTok{section}\KeywordTok{:}\AttributeTok{ }\StringTok{"Análisis"}
\AttributeTok{        }\FunctionTok{contents}\KeywordTok{:}
\AttributeTok{          }\KeywordTok{{-}}\AttributeTok{ analysis1.qmd}
\AttributeTok{          }\KeywordTok{{-}}\AttributeTok{ analysis2.qmd}
\end{Highlighting}
\end{Shaded}

\subsubsection{Múltiples Sidebars}\label{muxfaltiples-sidebars}

\begin{Shaded}
\begin{Highlighting}[]
\FunctionTok{website}\KeywordTok{:}
\AttributeTok{  }\FunctionTok{sidebar}\KeywordTok{:}
\AttributeTok{    }\KeywordTok{{-}}\AttributeTok{ }\FunctionTok{title}\KeywordTok{:}\AttributeTok{ }\StringTok{"Blog"}
\AttributeTok{      }\FunctionTok{contents}\KeywordTok{:}
\AttributeTok{        }\KeywordTok{{-}}\AttributeTok{ blog/index.qmd}
\AttributeTok{        }\KeywordTok{{-}}\AttributeTok{ blog/post1.qmd}
\AttributeTok{        }\KeywordTok{{-}}\AttributeTok{ blog/post2.qmd}
\AttributeTok{        }
\AttributeTok{    }\KeywordTok{{-}}\AttributeTok{ }\FunctionTok{title}\KeywordTok{:}\AttributeTok{ }\StringTok{"Tutoriales"}
\AttributeTok{      }\FunctionTok{contents}\KeywordTok{:}
\AttributeTok{        }\KeywordTok{{-}}\AttributeTok{ tutoriales/index.qmd}
\AttributeTok{        }\KeywordTok{{-}}\AttributeTok{ tutoriales/tutorial1.qmd}
\end{Highlighting}
\end{Shaded}

\subsubsection{Estilos de Sidebar}\label{estilos-de-sidebar}

\begin{Shaded}
\begin{Highlighting}[]
\FunctionTok{sidebar}\KeywordTok{:}
\AttributeTok{  }\FunctionTok{style}\KeywordTok{:}\AttributeTok{ docked}\CommentTok{                      \# Fija al contenido}
\CommentTok{  \# style: floating                  \# Flotante sobre el contenido}
\end{Highlighting}
\end{Shaded}

\subsubsection{Herramientas de Sidebar}\label{herramientas-de-sidebar}

\begin{Shaded}
\begin{Highlighting}[]
\FunctionTok{sidebar}\KeywordTok{:}
\AttributeTok{  }\FunctionTok{tools}\KeywordTok{:}
\AttributeTok{    }\KeywordTok{{-}}\AttributeTok{ }\FunctionTok{icon}\KeywordTok{:}\AttributeTok{ github}
\AttributeTok{      }\FunctionTok{href}\KeywordTok{:}\AttributeTok{ https://github.com/usuario}
\AttributeTok{      }\FunctionTok{text}\KeywordTok{:}\AttributeTok{ }\StringTok{"GitHub"}
\AttributeTok{    }\KeywordTok{{-}}\AttributeTok{ }\FunctionTok{icon}\KeywordTok{:}\AttributeTok{ twitter}
\AttributeTok{      }\FunctionTok{href}\KeywordTok{:}\AttributeTok{ https://twitter.com/usuario}
\end{Highlighting}
\end{Shaded}

\subsection{Page Footer}\label{page-footer}

El footer aparece al final de todas las páginas.

\subsubsection{Estructura Básica}\label{estructura-buxe1sica-2}

\begin{Shaded}
\begin{Highlighting}[]
\FunctionTok{website}\KeywordTok{:}
\AttributeTok{  }\FunctionTok{page{-}footer}\KeywordTok{:}
\AttributeTok{    }\FunctionTok{left}\KeywordTok{:}\AttributeTok{ }\StringTok{"© 2026 Mi Nombre"}
\FunctionTok{    center}\KeywordTok{: }\CharTok{|}
\NormalTok{      \textless{}a href="privacy.html"\textgreater{}Privacy\textless{}/a\textgreater{}}
\AttributeTok{    }\FunctionTok{right}\KeywordTok{:}
\AttributeTok{      }\KeywordTok{{-}}\AttributeTok{ }\FunctionTok{text}\KeywordTok{:}\AttributeTok{ }\StringTok{"About"}
\AttributeTok{        }\FunctionTok{href}\KeywordTok{:}\AttributeTok{ about.html}
\AttributeTok{      }\KeywordTok{{-}}\AttributeTok{ }\FunctionTok{text}\KeywordTok{:}\AttributeTok{ }\StringTok{"Contact"}
\AttributeTok{        }\FunctionTok{href}\KeywordTok{:}\AttributeTok{ contact.html}
\end{Highlighting}
\end{Shaded}

\subsubsection{Con HTML y Markdown}\label{con-html-y-markdown}

\begin{Shaded}
\begin{Highlighting}[]
\FunctionTok{page{-}footer}\KeywordTok{:}
\FunctionTok{  left}\KeywordTok{: }\CharTok{\textgreater{}{-}}
\NormalTok{    © 2026 Edison Achalma · Made with [Quarto](https://quarto.org)}
    
\FunctionTok{  center}\KeywordTok{: }\CharTok{|}
\NormalTok{    \textless{}a class="link{-}dark" href="https://github.com/usuario"\textgreater{}}
\NormalTok{      \textless{}i class="bi bi{-}github"\textgreater{}\textless{}/i\textgreater{}}
\NormalTok{    \textless{}/a\textgreater{}}
    
\AttributeTok{  }\FunctionTok{right}\KeywordTok{:}
\AttributeTok{    }\KeywordTok{{-}}\AttributeTok{ }\FunctionTok{text}\KeywordTok{:}\AttributeTok{ }\StringTok{"License"}
\AttributeTok{      }\FunctionTok{href}\KeywordTok{:}\AttributeTok{ license.html}
\AttributeTok{    }\KeywordTok{{-}}\AttributeTok{ }\FunctionTok{icon}\KeywordTok{:}\AttributeTok{ rss}
\AttributeTok{      }\FunctionTok{href}\KeywordTok{:}\AttributeTok{ index.xml}
\end{Highlighting}
\end{Shaded}

\subsubsection{Footer Completo}\label{footer-completo}

\begin{Shaded}
\begin{Highlighting}[]
\FunctionTok{page{-}footer}\KeywordTok{:}
\FunctionTok{  left}\KeywordTok{: }\CharTok{\textgreater{}{-}}
\NormalTok{    © 2026 Edison Achalma · Made with [Quarto](https://quarto.org)}
    
\FunctionTok{  center}\KeywordTok{: }\CharTok{|}
\NormalTok{    \textless{}a class="link{-}dark me{-}1" href="https://github.com/usuario"\textgreater{}}
\NormalTok{      \textless{}i class="bi bi{-}github"\textgreater{}\textless{}/i\textgreater{}}
\NormalTok{    \textless{}/a\textgreater{}}
\NormalTok{    \textless{}a class="link{-}dark me{-}1" href="https://twitter.com/usuario"\textgreater{}}
\NormalTok{      \textless{}i class="bi bi{-}twitter"\textgreater{}\textless{}/i\textgreater{}}
\NormalTok{    \textless{}/a\textgreater{}}
    
\AttributeTok{  }\FunctionTok{right}\KeywordTok{:}
\AttributeTok{    }\KeywordTok{{-}}\AttributeTok{ }\FunctionTok{text}\KeywordTok{:}\AttributeTok{ }\StringTok{"Accessibility"}
\AttributeTok{      }\FunctionTok{href}\KeywordTok{:}\AttributeTok{ accessibility.html}
\AttributeTok{    }\KeywordTok{{-}}\AttributeTok{ }\FunctionTok{text}\KeywordTok{:}\AttributeTok{ }\StringTok{"Contact"}
\AttributeTok{      }\FunctionTok{href}\KeywordTok{:}\AttributeTok{ contact.html}
\AttributeTok{    }\KeywordTok{{-}}\AttributeTok{ }\FunctionTok{text}\KeywordTok{:}\AttributeTok{ }\StringTok{"License"}
\AttributeTok{      }\FunctionTok{href}\KeywordTok{:}\AttributeTok{ license.html}
\AttributeTok{    }\KeywordTok{{-}}\AttributeTok{ }\FunctionTok{icon}\KeywordTok{:}\AttributeTok{ rss}
\AttributeTok{      }\FunctionTok{href}\KeywordTok{:}\AttributeTok{ index.xml}
\end{Highlighting}
\end{Shaded}

\subsection{Anuncios (Announcement
Bar)}\label{anuncios-announcement-bar}

La \textbf{Announcement Bar} (barra de anuncio) es una barra prominente
y personalizable que se muestra en la parte superior del sitio web. Es
ideal para resaltar información importante como alertas,
actualizaciones, promociones, eventos o mensajes temporales.

\subsubsection{Configuración Completa}\label{configuraciuxf3n-completa}

\begin{Shaded}
\begin{Highlighting}[]
\FunctionTok{website}\KeywordTok{:}
\AttributeTok{  }\FunctionTok{announcement}\KeywordTok{:}
\AttributeTok{    }\FunctionTok{icon}\KeywordTok{:}\AttributeTok{ info{-}circle}\CommentTok{               \# Ícono de Bootstrap}
\AttributeTok{    }\FunctionTok{dismissable}\KeywordTok{:}\AttributeTok{ }\CharTok{true}\CommentTok{               \# Puede cerrarse}
\AttributeTok{    }\FunctionTok{content}\KeywordTok{:}\AttributeTok{ }\StringTok{"**Importante:** Nueva versión disponible"}
\AttributeTok{    }\FunctionTok{type}\KeywordTok{:}\AttributeTok{ primary}\CommentTok{                   \# Color}
\AttributeTok{    }\FunctionTok{position}\KeywordTok{:}\AttributeTok{ below{-}navbar}\CommentTok{          \# Ubicación}
\end{Highlighting}
\end{Shaded}

\subsubsection{\texorpdfstring{Opciones de
\texttt{type}}{Opciones de type}}\label{opciones-de-type}

\begin{Shaded}
\begin{Highlighting}[]
\FunctionTok{type}\KeywordTok{:}\AttributeTok{ primary}\CommentTok{      \# Azul}
\FunctionTok{type}\KeywordTok{:}\AttributeTok{ secondary}\CommentTok{    \# Gris}
\FunctionTok{type}\KeywordTok{:}\AttributeTok{ success}\CommentTok{      \# Verde}
\FunctionTok{type}\KeywordTok{:}\AttributeTok{ danger}\CommentTok{       \# Rojo}
\FunctionTok{type}\KeywordTok{:}\AttributeTok{ warning}\CommentTok{      \# Amarillo}
\FunctionTok{type}\KeywordTok{:}\AttributeTok{ info}\CommentTok{         \# Azul claro}
\FunctionTok{type}\KeywordTok{:}\AttributeTok{ light}\CommentTok{        \# Blanco}
\FunctionTok{type}\KeywordTok{:}\AttributeTok{ dark}\CommentTok{         \# Negro}
\end{Highlighting}
\end{Shaded}

\subsubsection{\texorpdfstring{Opciones de
\texttt{position}}{Opciones de position}}\label{opciones-de-position}

\begin{Shaded}
\begin{Highlighting}[]
\FunctionTok{position}\KeywordTok{:}\AttributeTok{ below{-}navbar}\CommentTok{    \# Debajo de la navbar (por defecto)}
\FunctionTok{position}\KeywordTok{:}\AttributeTok{ above{-}navbar}\CommentTok{    \# Encima de la navbar}
\end{Highlighting}
\end{Shaded}

\subsubsection{Íconos Disponibles}\label{uxedconos-disponibles}

Cualquier ícono de \href{https://icons.getbootstrap.com/}{Bootstrap
Icons}:

\begin{itemize}
\tightlist
\item
  \texttt{info-circle}
\item
  \texttt{bell}
\item
  \texttt{megaphone}
\item
  \texttt{gift}
\item
  \texttt{exclamation-triangle}
\item
  \texttt{calendar-check}
\item
  \texttt{heart}
\end{itemize}

\subsubsection{Ejemplos por Tipo de
Anuncio}\label{ejemplos-por-tipo-de-anuncio}

\textbf{Alerta importante:}

\begin{Shaded}
\begin{Highlighting}[]
\FunctionTok{announcement}\KeywordTok{:}
\AttributeTok{  }\FunctionTok{icon}\KeywordTok{:}\AttributeTok{ exclamation{-}triangle}
\AttributeTok{  }\FunctionTok{dismissable}\KeywordTok{:}\AttributeTok{ }\CharTok{false}
\AttributeTok{  }\FunctionTok{content}\KeywordTok{:}\AttributeTok{ }\StringTok{"**Atención:** Mantenimiento programado el 15 de enero"}
\AttributeTok{  }\FunctionTok{type}\KeywordTok{:}\AttributeTok{ danger}
\AttributeTok{  }\FunctionTok{position}\KeywordTok{:}\AttributeTok{ above{-}navbar}
\end{Highlighting}
\end{Shaded}

\textbf{Evento o promoción:}

\begin{Shaded}
\begin{Highlighting}[]
\FunctionTok{announcement}\KeywordTok{:}
\AttributeTok{  }\FunctionTok{icon}\KeywordTok{:}\AttributeTok{ calendar{-}check}
\AttributeTok{  }\FunctionTok{dismissable}\KeywordTok{:}\AttributeTok{ }\CharTok{true}
\AttributeTok{  }\FunctionTok{content}\KeywordTok{:}\AttributeTok{ }\StringTok{"Webinar gratuito el 20 de enero. \textbackslash{}}
\StringTok{  [Regístrate aquí](https://example.com)"}
\AttributeTok{  }\FunctionTok{type}\KeywordTok{:}\AttributeTok{ success}
\end{Highlighting}
\end{Shaded}

\textbf{Actualización:}

\begin{Shaded}
\begin{Highlighting}[]
\FunctionTok{announcement}\KeywordTok{:}
\AttributeTok{  }\FunctionTok{icon}\KeywordTok{:}\AttributeTok{ info{-}circle}
\AttributeTok{  }\FunctionTok{dismissable}\KeywordTok{:}\AttributeTok{ }\CharTok{true}
\AttributeTok{  }\FunctionTok{content}\KeywordTok{:}\AttributeTok{ }\StringTok{"Nueva versión disponible. \textbackslash{}}
\StringTok{  [Ver cambios](https://example.com/changelog)"}
\AttributeTok{  }\FunctionTok{type}\KeywordTok{:}\AttributeTok{ info}
\end{Highlighting}
\end{Shaded}

\subsection{Open Graph}\label{open-graph}

\textbf{Open Graph} mejora la presentación de tus páginas cuando se
comparten en plataformas como Facebook, LinkedIn, Slack, Discord,
WhatsApp, iMessage y muchas más, mostrando título, descripción, imagen
de vista previa y otros metadatos enriquecidos.

\subsubsection{Configuración
Completa}\label{configuraciuxf3n-completa-1}

\begin{Shaded}
\begin{Highlighting}[]
\FunctionTok{website}\KeywordTok{:}
\AttributeTok{  }\FunctionTok{open{-}graph}\KeywordTok{:}
\AttributeTok{    }\FunctionTok{title}\KeywordTok{:}\AttributeTok{ }\StringTok{"Mi Sitio Web"}
\AttributeTok{    }\FunctionTok{description}\KeywordTok{:}\AttributeTok{ }\StringTok{"Descripción del sitio"}
\AttributeTok{    }\FunctionTok{image}\KeywordTok{:}\AttributeTok{ /assets/img/og{-}image.jpg}
\AttributeTok{    }\FunctionTok{image{-}width}\KeywordTok{:}\AttributeTok{ }\DecValTok{1200}
\AttributeTok{    }\FunctionTok{image{-}height}\KeywordTok{:}\AttributeTok{ }\DecValTok{630}
\AttributeTok{    }\FunctionTok{locale}\KeywordTok{:}\AttributeTok{ es\_ES}
\AttributeTok{    }\FunctionTok{site{-}name}\KeywordTok{:}\AttributeTok{ }\StringTok{"Mi Sitio"}
\end{Highlighting}
\end{Shaded}

\subsubsection{Opciones}\label{opciones}

\begin{longtable}[]{@{}lll@{}}
\toprule\noalign{}
Opción & Descripción & Recomendación \\
\midrule\noalign{}
\endhead
\bottomrule\noalign{}
\endlastfoot
\texttt{title} & Título del sitio & Máx. 60 caracteres \\
\texttt{description} & Descripción & 150-160 caracteres \\
\texttt{image} & Imagen de vista previa & 1200x630 px, ratio 1.91:1 \\
\texttt{image-width} & Ancho de la imagen & 1200 \\
\texttt{image-height} & Alto de la imagen & 630 \\
\texttt{locale} & Idioma y región & es\_ES, en\_US, etc. \\
\texttt{site-name} & Nombre del sitio & Nombre corto \\
\end{longtable}

\subsubsection{Imágenes de vista previa (orden de
prioridad)}\label{imuxe1genes-de-vista-previa-orden-de-prioridad}

Exactamente igual que en Twitter Cards:

\begin{enumerate}
\def\labelenumi{\arabic{enumi}.}
\item
  Metadato \texttt{image} en la página (URL completa o ruta relativa).
\item
  Imagen con clase \texttt{.preview-image} en el contenido:

\begin{Shaded}
\begin{Highlighting}[]
\AlertTok{![Portada](/images/cover.jpg)}\NormalTok{\{.preview{-}image\}}
\end{Highlighting}
\end{Shaded}
\item
  Archivos con nombres: \texttt{preview.png}, \texttt{feature.png},
  \texttt{cover.png}, \texttt{thumbnail.png}.
\item
  Imagen por defecto del sitio:

\begin{Shaded}
\begin{Highlighting}[]
\FunctionTok{website}\KeywordTok{:}
\AttributeTok{  }\FunctionTok{image}\KeywordTok{:}\AttributeTok{ images/default{-}og{-}preview.jpg}
\end{Highlighting}
\end{Shaded}
\end{enumerate}

Para desactivar la imagen en una página específica:

\begin{Shaded}
\begin{Highlighting}[]
\PreprocessorTok{{-}{-}{-}}
\FunctionTok{image}\KeywordTok{:}\AttributeTok{ }\CharTok{false}
\PreprocessorTok{{-}{-}{-}}
\end{Highlighting}
\end{Shaded}

\subsubsection{Ejemplos prácticos}\label{ejemplos-pruxe1cticos-1}

\paragraph{1. Configuración mínima
recomendada.}\label{configuraciuxf3n-muxednima-recomendada}

\begin{Shaded}
\begin{Highlighting}[]
\FunctionTok{website}\KeywordTok{:}
\AttributeTok{  }\FunctionTok{site{-}url}\KeywordTok{:}\AttributeTok{ https://misitio.com}
\AttributeTok{  }\FunctionTok{open{-}graph}\KeywordTok{:}\AttributeTok{ }\CharTok{true}
\end{Highlighting}
\end{Shaded}

\paragraph{2. Configuración profesional
(global).}\label{configuraciuxf3n-profesional-global}

\begin{Shaded}
\begin{Highlighting}[]
\FunctionTok{website}\KeywordTok{:}
\AttributeTok{  }\FunctionTok{site{-}url}\KeywordTok{:}\AttributeTok{ https://misitio.com}
\AttributeTok{  }\FunctionTok{open{-}graph}\KeywordTok{:}
\AttributeTok{    }\FunctionTok{locale}\KeywordTok{:}\AttributeTok{ es\_ES}
\AttributeTok{    }\FunctionTok{site{-}name}\KeywordTok{:}\AttributeTok{ Mi Sitio con Quarto}
\AttributeTok{  }\FunctionTok{image}\KeywordTok{:}\AttributeTok{ images/default{-}og.jpg}\CommentTok{ \# Img por defecto (1200x630 px recomendado)}
\end{Highlighting}
\end{Shaded}

\paragraph{3. Personalización por página (en el front-matter de
page.qmd).}\label{personalizaciuxf3n-por-puxe1gina-en-el-front-matter-de-page.qmd}

\begin{Shaded}
\begin{Highlighting}[]
\PreprocessorTok{{-}{-}{-}}
\FunctionTok{title}\KeywordTok{:}\AttributeTok{ }\StringTok{"Guía Avanzada de Quarto"}
\FunctionTok{description}\KeywordTok{:}\AttributeTok{ }\StringTok{"Domina todas las funciones de Quarto en 2025"}
\FunctionTok{image}\KeywordTok{:}\AttributeTok{ /images/guia{-}avanzada{-}og.png}
\FunctionTok{open{-}graph}\KeywordTok{:}
\AttributeTok{  }\FunctionTok{title}\KeywordTok{:}\AttributeTok{ }\StringTok{"¡La Guía Definitiva de Quarto 2025!"}
\AttributeTok{  }\FunctionTok{description}\KeywordTok{:}\AttributeTok{ }\StringTok{"Todo lo nuevo y avanzado en una sola guía"}
\AttributeTok{  }\FunctionTok{image{-}width}\KeywordTok{:}\AttributeTok{ }\DecValTok{1200}
\AttributeTok{  }\FunctionTok{image{-}height}\KeywordTok{:}\AttributeTok{ }\DecValTok{630}
\PreprocessorTok{{-}{-}{-}}
\end{Highlighting}
\end{Shaded}

\textbf{Importante:} Requiere \texttt{site-url} definido para rutas
relativas.

\subsection{Preview Images}\label{preview-images}

Las \textbf{Preview Images} (imágenes de vista previa) son las imágenes
que se muestran cuando una página de tu sitio se comparte en redes
sociales (Twitter/X Cards, Open Graph) o en mensajeros como Slack,
WhatsApp, etc. Quarto busca automáticamente estas imágenes siguiendo un
orden de prioridad específico, lo que facilita su configuración.

Estas imágenes se usan tanto para \textbf{Twitter Cards} como para
\textbf{Open Graph}.

\subsubsection{Configuración básica}\label{configuraciuxf3n-buxe1sica}

La forma más simple es definir una imagen por defecto a nivel de sitio
en \texttt{\_quarto.yml}:

\begin{Shaded}
\begin{Highlighting}[]
\FunctionTok{website}\KeywordTok{:}
\AttributeTok{  }\FunctionTok{site{-}url}\KeywordTok{:}\AttributeTok{ https://misitio.com}\CommentTok{      \# Obligatorio para rutas relativas}
\AttributeTok{  }\FunctionTok{image}\KeywordTok{:}\AttributeTok{ images/default{-}preview.jpg}\CommentTok{  \# Imagen por defecto para todo el sitio}
\end{Highlighting}
\end{Shaded}

Con esto, todas las páginas que no especifiquen su propia imagen usarán
esta.

\subsubsection{Configuración completa (todas las opciones y
métodos)}\label{configuraciuxf3n-completa-todas-las-opciones-y-muxe9todos}

Quarto busca la imagen de vista previa en el siguiente orden de
prioridad:

\paragraph{1. Imagen explícita en el front-matter de la página (método
recomendado).}\label{imagen-expluxedcita-en-el-front-matter-de-la-puxe1gina-muxe9todo-recomendado}

\begin{Shaded}
\begin{Highlighting}[]
\CommentTok{\# En page.qmd}
\PreprocessorTok{{-}{-}{-}}
\FunctionTok{title}\KeywordTok{:}\AttributeTok{ }\StringTok{"Mi Artículo Genial"}
\FunctionTok{description}\KeywordTok{:}\AttributeTok{ }\StringTok{"Una guía completa sobre Quarto"}
\FunctionTok{image}\KeywordTok{:}\AttributeTok{ /images/mi{-}articulo{-}preview.jpg}\CommentTok{     \# Ruta relativa al proyecto}
\CommentTok{\# o URL completa:}
\CommentTok{\# image: https://misitio.com/images/mi{-}articulo{-}preview.jpg}
\FunctionTok{image{-}width}\KeywordTok{:}\AttributeTok{ }\DecValTok{1200}\CommentTok{                          \# Opcional: ancho en píxeles}
\FunctionTok{image{-}height}\KeywordTok{:}\AttributeTok{ }\DecValTok{630}\CommentTok{                          \# Opcional: alto en píxeles}
\PreprocessorTok{{-}{-}{-}}
\end{Highlighting}
\end{Shaded}

\paragraph{2. Imagen con clase especial en el contenido
Markdown.}\label{imagen-con-clase-especial-en-el-contenido-markdown}

\begin{Shaded}
\begin{Highlighting}[]
\AlertTok{![Portada del artículo](/images/portada.jpg)}\NormalTok{\{.preview{-}image width="1200" height="630"\}}
\end{Highlighting}
\end{Shaded}

Quarto selecciona la \textbf{primera} imagen con la clase
\texttt{.preview-image}.

\paragraph{3. Imagen por nombre
automático.}\label{imagen-por-nombre-automuxe1tico}

Si no hay ninguna de las anteriores, Quarto busca en la página
renderizada un archivo con uno de estos nombres: - \texttt{preview.png}
- \texttt{feature.png} - \texttt{cover.png} - \texttt{thumbnail.png}

\paragraph{\texorpdfstring{4. Imagen por defecto del sitio (en
\texttt{\_quarto.yml}).}{4. Imagen por defecto del sitio (en \_quarto.yml).}}\label{imagen-por-defecto-del-sitio-en-_quarto.yml}

\begin{Shaded}
\begin{Highlighting}[]
\FunctionTok{website}\KeywordTok{:}
\AttributeTok{  }\FunctionTok{image}\KeywordTok{:}\AttributeTok{ images/default{-}preview.jpg}\CommentTok{      \# Usada cuando ninguna página tiene su propia imagen}
\end{Highlighting}
\end{Shaded}

\paragraph{5. Desactivar imagen en una página
específica.}\label{desactivar-imagen-en-una-puxe1gina-especuxedfica}

\begin{Shaded}
\begin{Highlighting}[]
\PreprocessorTok{{-}{-}{-}}
\FunctionTok{title}\KeywordTok{:}\AttributeTok{ }\StringTok{"Página sin vista previa"}
\FunctionTok{image}\KeywordTok{:}\AttributeTok{ }\CharTok{false}\CommentTok{                           \# Evita que se use cualquier imagen}
\PreprocessorTok{{-}{-}{-}}
\end{Highlighting}
\end{Shaded}

\subsubsection{Ejemplos prácticos}\label{ejemplos-pruxe1cticos-2}

\paragraph{1. Configuración global básica + página
individual.}\label{configuraciuxf3n-global-buxe1sica-puxe1gina-individual}

\begin{Shaded}
\begin{Highlighting}[]
\CommentTok{\# \_quarto.yml}
\FunctionTok{website}\KeywordTok{:}
\AttributeTok{  }\FunctionTok{site{-}url}\KeywordTok{:}\AttributeTok{ https://misitio.com}
\AttributeTok{  }\FunctionTok{image}\KeywordTok{:}\AttributeTok{ images/default{-}og{-}preview.jpg}\CommentTok{  \# Imagen por defecto (1200x630 px)}
\end{Highlighting}
\end{Shaded}

\begin{Shaded}
\begin{Highlighting}[]
\CommentTok{\# blog/nuevo{-}articulo.qmd}
\PreprocessorTok{{-}{-}{-}}
\FunctionTok{title}\KeywordTok{:}\AttributeTok{ }\StringTok{"Lanzamiento de Quarto 1.5"}
\FunctionTok{description}\KeywordTok{:}\AttributeTok{ }\StringTok{"Nuevas funciones increíbles"}
\FunctionTok{image}\KeywordTok{:}\AttributeTok{ /images/quarto{-}1.5{-}preview.png}\CommentTok{  \# Imagen específica para este post}
\PreprocessorTok{{-}{-}{-}}
\end{Highlighting}
\end{Shaded}

\paragraph{2. Usando clase .preview-image en el
contenido.}\label{usando-clase-.preview-image-en-el-contenido}

\begin{Shaded}
\begin{Highlighting}[]
\FunctionTok{\# Mi Guía Definitiva}

\AlertTok{![Esta será la imagen de vista previa](/images/guia{-}completa.jpg)}\NormalTok{\{.preview{-}image\}}

\NormalTok{¡Bienvenido a la guía más completa de Quarto!}
\end{Highlighting}
\end{Shaded}

\paragraph{3. Página con dimensiones específicas (mejora
presentación).}\label{puxe1gina-con-dimensiones-especuxedficas-mejora-presentaciuxf3n}

\begin{Shaded}
\begin{Highlighting}[]
\PreprocessorTok{{-}{-}{-}}
\FunctionTok{title}\KeywordTok{:}\AttributeTok{ }\StringTok{"Análisis de Datos con Quarto"}
\FunctionTok{image}\KeywordTok{:}\AttributeTok{ /images/analisis{-}datos{-}preview.jpg}
\FunctionTok{image{-}width}\KeywordTok{:}\AttributeTok{ }\DecValTok{1200}
\FunctionTok{image{-}height}\KeywordTok{:}\AttributeTok{ }\DecValTok{630}
\PreprocessorTok{{-}{-}{-}}
\end{Highlighting}
\end{Shaded}

\paragraph{4. Sitio con imagen por defecto y algunas páginas
personalizadas.}\label{sitio-con-imagen-por-defecto-y-algunas-puxe1ginas-personalizadas}

\begin{Shaded}
\begin{Highlighting}[]
\CommentTok{\# \_quarto.yml}
\FunctionTok{website}\KeywordTok{:}
\AttributeTok{  }\FunctionTok{site{-}url}\KeywordTok{:}\AttributeTok{ https://misitio.com}
\AttributeTok{  }\FunctionTok{image}\KeywordTok{:}\AttributeTok{ images/site{-}default{-}preview.jpg}\CommentTok{  \# Usada en el 90\% de las páginas}
\AttributeTok{  }\FunctionTok{twitter{-}card}\KeywordTok{:}\AttributeTok{ }\CharTok{true}
\AttributeTok{  }\FunctionTok{open{-}graph}\KeywordTok{:}\AttributeTok{ }\CharTok{true}
\end{Highlighting}
\end{Shaded}

Páginas normales: usan la imagen por defecto.\\
Páginas importantes (blog posts, anuncios): definen su propia
\texttt{image:}.

\subsection{Twitter Cards}\label{twitter-cards}

\textbf{Twitter Cards} (ahora también conocidas como X Cards) permiten
que los enlaces a tu sitio web se muestren de forma enriquecida cuando
se comparten en X/Twitter, incluyendo título, descripción, imagen de
vista previa y estilo de tarjeta.

\subsubsection{Configuración
Completa}\label{configuraciuxf3n-completa-2}

\begin{Shaded}
\begin{Highlighting}[]
\FunctionTok{website}\KeywordTok{:}
\AttributeTok{  }\FunctionTok{twitter{-}card}\KeywordTok{:}
\AttributeTok{    }\FunctionTok{creator}\KeywordTok{:}\AttributeTok{ }\StringTok{"@usuario"}
\AttributeTok{    }\FunctionTok{site}\KeywordTok{:}\AttributeTok{ }\StringTok{"@sitio\_oficial"}
\AttributeTok{    }\FunctionTok{card{-}style}\KeywordTok{:}\AttributeTok{ summary\_large\_image}
\end{Highlighting}
\end{Shaded}

\subsubsection{Opciones}\label{opciones-1}

\begin{longtable}[]{@{}
  >{\raggedright\arraybackslash}p{(\linewidth - 6\tabcolsep) * \real{0.1173}}
  >{\raggedright\arraybackslash}p{(\linewidth - 6\tabcolsep) * \real{0.1564}}
  >{\raggedright\arraybackslash}p{(\linewidth - 6\tabcolsep) * \real{0.4302}}
  >{\raggedright\arraybackslash}p{(\linewidth - 6\tabcolsep) * \real{0.2961}}@{}}
\toprule\noalign{}
\begin{minipage}[b]{\linewidth}\raggedright
Opción
\end{minipage} & \begin{minipage}[b]{\linewidth}\raggedright
Ubicación
\end{minipage} & \begin{minipage}[b]{\linewidth}\raggedright
Descripción
\end{minipage} & \begin{minipage}[b]{\linewidth}\raggedright
Ejemplo / Valores posibles
\end{minipage} \\
\midrule\noalign{}
\endhead
\bottomrule\noalign{}
\endlastfoot
\texttt{twitter-card} & \texttt{website:} & Activa la generación
automática & \texttt{true} \\
\texttt{creator} & Dentro de \texttt{twitter-card:} & \texttt{@username}
del autor del contenido & \texttt{"@juanperez"} (comillas si incluye
\texttt{@}) \\
\texttt{site} & Dentro de \texttt{twitter-card:} & \texttt{@username} de
la cuenta oficial del sitio & \texttt{"@quarto\_pub"} \\
\texttt{card-style} & Dentro de \texttt{twitter-card:} & Estilo de la
tarjeta & \texttt{summary} o \texttt{summary\_large\_image} \\
\texttt{title} & Global o por página & Sobrescribe el título solo para
la tarjeta & ``¡Nueva guía de Quarto!'' \\
\texttt{description} & Global o por página & Sobrescribe la descripción
& ``Aprende todo sobre sitios web con Quarto'' \\
\texttt{image} & Global o por página & Ruta/URL de la imagen de vista
previa & ``/images/preview.png'' o URL completa \\
\texttt{image-width} & Global o por página & Ancho recomendado de la
imagen & 1200 \\
\texttt{image-height} & Global o por página & Alto recomendado de la
imagen & \\
\end{longtable}

\textbf{Estilos de tarjeta:}

\begin{itemize}
\tightlist
\item
  \texttt{summary}: Imagen pequeña al lado
\item
  \texttt{summary\_large\_image}: Imagen grande arriba (recomendado)
\end{itemize}

\subsubsection{Imágenes de vista previa (orden de
prioridad)}\label{imuxe1genes-de-vista-previa-orden-de-prioridad-1}

Quarto busca la imagen en este orden:

\begin{enumerate}
\def\labelenumi{\arabic{enumi}.}
\item
  Metadato \texttt{image} en la página (URL completa o ruta relativa).
\item
  Imagen con clase \texttt{.preview-image} en el contenido:

\begin{Shaded}
\begin{Highlighting}[]
\AlertTok{![Mi imagen](/images/cover.jpg)}\NormalTok{\{.preview{-}image\}}
\end{Highlighting}
\end{Shaded}
\item
  Archivos con nombres: \texttt{preview.png}, \texttt{feature.png},
  \texttt{cover.png}, \texttt{thumbnail.png}.
\item
  Imagen por defecto del sitio (configurada globalmente):

\begin{Shaded}
\begin{Highlighting}[]
\FunctionTok{website}\KeywordTok{:}
\AttributeTok{  }\FunctionTok{image}\KeywordTok{:}\AttributeTok{ images/default{-}preview.jpg}
\end{Highlighting}
\end{Shaded}
\end{enumerate}

Para desactivar la imagen en una página específica:

\begin{Shaded}
\begin{Highlighting}[]
\PreprocessorTok{{-}{-}{-}}
\FunctionTok{image}\KeywordTok{:}\AttributeTok{ }\CharTok{false}
\PreprocessorTok{{-}{-}{-}}
\end{Highlighting}
\end{Shaded}

\subsubsection{Ejemplos prácticos}\label{ejemplos-pruxe1cticos-3}

\paragraph{1. Configuración mínima
recomendada.}\label{configuraciuxf3n-muxednima-recomendada-1}

\begin{Shaded}
\begin{Highlighting}[]
\FunctionTok{website}\KeywordTok{:}
\AttributeTok{  }\FunctionTok{site{-}url}\KeywordTok{:}\AttributeTok{ https://misitio.com}
\AttributeTok{  }\FunctionTok{twitter{-}card}\KeywordTok{:}\AttributeTok{ }\CharTok{true}
\end{Highlighting}
\end{Shaded}

\paragraph{2. Configuración profesional
(global).}\label{configuraciuxf3n-profesional-global-1}

\begin{Shaded}
\begin{Highlighting}[]
\FunctionTok{website}\KeywordTok{:}
\AttributeTok{  }\FunctionTok{site{-}url}\KeywordTok{:}\AttributeTok{ https://misitio.com}
\AttributeTok{  }\FunctionTok{twitter{-}card}\KeywordTok{:}
\AttributeTok{    }\FunctionTok{creator}\KeywordTok{:}\AttributeTok{ }\StringTok{"@autor\_principal"}
\AttributeTok{    }\FunctionTok{site}\KeywordTok{:}\AttributeTok{ }\StringTok{"@mi\_sitio\_quarto"}
\AttributeTok{    }\FunctionTok{card{-}style}\KeywordTok{:}\AttributeTok{ summary\_large\_image}
\AttributeTok{  }\FunctionTok{image}\KeywordTok{:}\AttributeTok{ images/default{-}card.jpg}\CommentTok{   \# Imagen por defecto}
\end{Highlighting}
\end{Shaded}

\paragraph{3. Personalización por página (en el front-matter de
page.qmd).}\label{personalizaciuxf3n-por-puxe1gina-en-el-front-matter-de-page.qmd-1}

\begin{Shaded}
\begin{Highlighting}[]
\PreprocessorTok{{-}{-}{-}}
\FunctionTok{title}\KeywordTok{:}\AttributeTok{ }\StringTok{"Mi Guía Definitiva de Quarto"}
\FunctionTok{description}\KeywordTok{:}\AttributeTok{ }\StringTok{"Aprende a crear sitios increíbles"}
\FunctionTok{image}\KeywordTok{:}\AttributeTok{ /images/guia{-}quarto{-}preview.png}
\FunctionTok{twitter{-}card}\KeywordTok{:}
\AttributeTok{  }\FunctionTok{title}\KeywordTok{:}\AttributeTok{ }\StringTok{"¡Nueva Guía de Quarto 2025!"}
\AttributeTok{  }\FunctionTok{description}\KeywordTok{:}\AttributeTok{ }\StringTok{"Todo lo que necesitas para dominar Quarto"}
\AttributeTok{  }\FunctionTok{image}\KeywordTok{:}\AttributeTok{ /images/guia{-}quarto{-}large.jpg}
\PreprocessorTok{{-}{-}{-}}
\end{Highlighting}
\end{Shaded}

\textbf{Nota:} Twitter Cards también usa la configuración de Open Graph
para título, descripción e imagen.

\subsection{Google Analytics}\label{google-analytics}

Integración con Google Analytics.

\subsubsection{Configuración Básica}\label{configuraciuxf3n-buxe1sica-1}

\begin{Shaded}
\begin{Highlighting}[]
\FunctionTok{website}\KeywordTok{:}
\AttributeTok{  }\FunctionTok{google{-}analytics}\KeywordTok{:}\AttributeTok{ }\StringTok{"G{-}XXXXXXXXXX"}
\end{Highlighting}
\end{Shaded}

\subsubsection{Configuración
Completa}\label{configuraciuxf3n-completa-3}

\begin{Shaded}
\begin{Highlighting}[]
\FunctionTok{website}\KeywordTok{:}
\AttributeTok{  }\FunctionTok{google{-}analytics}\KeywordTok{:}
\AttributeTok{    }\FunctionTok{tracking{-}id}\KeywordTok{:}\AttributeTok{ }\StringTok{"G{-}XXXXXXXXXX"}
\AttributeTok{    }\FunctionTok{storage}\KeywordTok{:}\AttributeTok{ cookies}\CommentTok{                 \# cookies o none}
\AttributeTok{    }\FunctionTok{anonymize{-}ip}\KeywordTok{:}\AttributeTok{ }\CharTok{true}\CommentTok{               \# Anonimizar IPs (RGPD)}
\AttributeTok{    }\FunctionTok{version}\KeywordTok{:}\AttributeTok{ }\DecValTok{4}\CommentTok{                       \# Versión (3 o 4)}
\end{Highlighting}
\end{Shaded}

\subsubsection{Opciones}\label{opciones-2}

\begin{longtable}[]{@{}
  >{\raggedright\arraybackslash}p{(\linewidth - 6\tabcolsep) * \real{0.1778}}
  >{\raggedright\arraybackslash}p{(\linewidth - 6\tabcolsep) * \real{0.2889}}
  >{\raggedright\arraybackslash}p{(\linewidth - 6\tabcolsep) * \real{0.2000}}
  >{\raggedright\arraybackslash}p{(\linewidth - 6\tabcolsep) * \real{0.3333}}@{}}
\toprule\noalign{}
\begin{minipage}[b]{\linewidth}\raggedright
Opción
\end{minipage} & \begin{minipage}[b]{\linewidth}\raggedright
Descripción
\end{minipage} & \begin{minipage}[b]{\linewidth}\raggedright
Valores
\end{minipage} & \begin{minipage}[b]{\linewidth}\raggedright
Recomendación
\end{minipage} \\
\midrule\noalign{}
\endhead
\bottomrule\noalign{}
\endlastfoot
\texttt{tracking-id} & ID de seguimiento & G-XXXXXXX (GA4) o UA-XXXXX
(antiguo) & Obligatorio \\
\texttt{storage} & Almacenamiento & \texttt{cookies} (por defecto) o
\texttt{none} & \texttt{cookies} normal, \texttt{none} para
privacidad \\
\texttt{anonymize-ip} & Anonimizar IPs & \texttt{true} o \texttt{false}
& \texttt{true} para RGPD/privacidad \\
\texttt{version} & Versión de GA & \texttt{3} o \texttt{4} & Se detecta
automáticamente \\
\end{longtable}

\textbf{Recomendación para Europa:}

\begin{Shaded}
\begin{Highlighting}[]
\FunctionTok{google{-}analytics}\KeywordTok{:}
\AttributeTok{  }\FunctionTok{tracking{-}id}\KeywordTok{:}\AttributeTok{ }\StringTok{"G{-}XXXXXXXXXX"}
\AttributeTok{  }\FunctionTok{anonymize{-}ip}\KeywordTok{:}\AttributeTok{ }\CharTok{true}
\end{Highlighting}
\end{Shaded}

\subsubsection{Ejemplos prácticos}\label{ejemplos-pruxe1cticos-4}

\paragraph{1. Configuración mínima (recomendada para la
mayoría).}\label{configuraciuxf3n-muxednima-recomendada-para-la-mayoruxeda}

\begin{Shaded}
\begin{Highlighting}[]
\FunctionTok{website}\KeywordTok{:}
\AttributeTok{  }\FunctionTok{google{-}analytics}\KeywordTok{:}\AttributeTok{ }\StringTok{"G{-}XXXXXXXXXX"}\CommentTok{   \# Solo el ID, todo lo demás por defecto}
\end{Highlighting}
\end{Shaded}

\paragraph{2. Configuración respetuosa con la privacidad (recomendada en
Europa).}\label{configuraciuxf3n-respetuosa-con-la-privacidad-recomendada-en-europa}

\begin{Shaded}
\begin{Highlighting}[]
\FunctionTok{website}\KeywordTok{:}
\AttributeTok{  }\FunctionTok{google{-}analytics}\KeywordTok{:}
\AttributeTok{    }\FunctionTok{tracking{-}id}\KeywordTok{:}\AttributeTok{ }\StringTok{"G{-}XXXXXXXXXX"}
\AttributeTok{    }\FunctionTok{anonymize{-}ip}\KeywordTok{:}\AttributeTok{ }\CharTok{true}\CommentTok{                \# Cumple mejor con RGPD}
\end{Highlighting}
\end{Shaded}

\paragraph{3. Sin cookies propias (máxima
privacidad).}\label{sin-cookies-propias-muxe1xima-privacidad}

\begin{Shaded}
\begin{Highlighting}[]
\FunctionTok{website}\KeywordTok{:}
\AttributeTok{  }\FunctionTok{google{-}analytics}\KeywordTok{:}
\AttributeTok{    }\FunctionTok{tracking{-}id}\KeywordTok{:}\AttributeTok{ }\StringTok{"G{-}XXXXXXXXXX"}
\AttributeTok{    }\FunctionTok{anonymize{-}ip}\KeywordTok{:}\AttributeTok{ }\CharTok{true}
\AttributeTok{    }\FunctionTok{storage}\KeywordTok{:}\AttributeTok{ none}\CommentTok{                     \# No usa cookies de identificación}
\end{Highlighting}
\end{Shaded}

\paragraph{4. Combinación con Cookie Consent (ideal para cumplimiento
legal).}\label{combinaciuxf3n-con-cookie-consent-ideal-para-cumplimiento-legal}

\begin{Shaded}
\begin{Highlighting}[]
\FunctionTok{website}\KeywordTok{:}
\AttributeTok{  }\FunctionTok{google{-}analytics}\KeywordTok{:}
\AttributeTok{    }\FunctionTok{tracking{-}id}\KeywordTok{:}\AttributeTok{ }\StringTok{"G{-}XXXXXXXXXX"}
\AttributeTok{    }\FunctionTok{anonymize{-}ip}\KeywordTok{:}\AttributeTok{ }\CharTok{true}
\AttributeTok{  }
\AttributeTok{  }\FunctionTok{cookie{-}consent}\KeywordTok{:}
\AttributeTok{    }\FunctionTok{type}\KeywordTok{:}\AttributeTok{ express}\CommentTok{                     \# Bloquea cookies hasta que el usuario acepte}
\AttributeTok{    }\FunctionTok{style}\KeywordTok{:}\AttributeTok{ headline}
\AttributeTok{    }\FunctionTok{palette}\KeywordTok{:}\AttributeTok{ dark}
\end{Highlighting}
\end{Shaded}

Con \texttt{cookie-consent} activado, Google Analytics \textbf{solo se
cargará} si el usuario acepta las cookies de seguimiento.

\subsection{Cookie Consent}\label{cookie-consent}

Solicitar consentimiento para cookies (RGPD).

\subsubsection{Configuración
Completa}\label{configuraciuxf3n-completa-4}

\begin{Shaded}
\begin{Highlighting}[]
\FunctionTok{website}\KeywordTok{:}
\AttributeTok{  }\FunctionTok{cookie{-}consent}\KeywordTok{:}
\AttributeTok{    }\FunctionTok{type}\KeywordTok{:}\AttributeTok{ express}\CommentTok{                    \# implied o express}
\AttributeTok{    }\FunctionTok{style}\KeywordTok{:}\AttributeTok{ headline}\CommentTok{                  \# simple, interstitial, standalone}
\AttributeTok{    }\FunctionTok{palette}\KeywordTok{:}\AttributeTok{ dark}\CommentTok{                    \# light o dark}
\AttributeTok{    }\FunctionTok{language}\KeywordTok{:}\AttributeTok{ es}\CommentTok{                     \# Código de idioma}
\AttributeTok{    }\FunctionTok{policy{-}url}\KeywordTok{:}\AttributeTok{ /politica{-}cookies}\CommentTok{    \# URL de política}
\AttributeTok{    }\FunctionTok{prefs{-}text}\KeywordTok{:}\AttributeTok{ }\StringTok{"Preferencias de Cookies"}
\end{Highlighting}
\end{Shaded}

\subsubsection{\texorpdfstring{Opciones de
\texttt{type}}{Opciones de type}}\label{opciones-de-type-1}

\begin{longtable}[]{@{}
  >{\raggedright\arraybackslash}p{(\linewidth - 2\tabcolsep) * \real{0.3043}}
  >{\raggedright\arraybackslash}p{(\linewidth - 2\tabcolsep) * \real{0.6957}}@{}}
\toprule\noalign{}
\begin{minipage}[b]{\linewidth}\raggedright
Valor
\end{minipage} & \begin{minipage}[b]{\linewidth}\raggedright
Comportamiento
\end{minipage} \\
\midrule\noalign{}
\endhead
\bottomrule\noalign{}
\endlastfoot
\texttt{implied} & Informa pero no bloquea cookies (notificación) \\
\texttt{express} & Bloquea cookies hasta que el usuario acepte
explícitamente \\
\end{longtable}

\textbf{Recomendación:} \texttt{express} para cumplir con RGPD.

\subsubsection{\texorpdfstring{Opciones de
\texttt{style}}{Opciones de style}}\label{opciones-de-style}

\begin{longtable}[]{@{}ll@{}}
\toprule\noalign{}
Valor & Apariencia \\
\midrule\noalign{}
\endhead
\bottomrule\noalign{}
\endlastfoot
\texttt{simple} & Diálogo pequeño en esquina inferior derecha \\
\texttt{headline} & Barra completa en la parte superior \\
\texttt{interstitial} & Overlay semitransparente \\
\texttt{standalone} & Overlay opaco \\
\end{longtable}

\subsubsection{Integración con Google
Analytics}\label{integraciuxf3n-con-google-analytics}

\begin{Shaded}
\begin{Highlighting}[]
\FunctionTok{website}\KeywordTok{:}
\AttributeTok{  }\FunctionTok{cookie{-}consent}\KeywordTok{:}
\AttributeTok{    }\FunctionTok{type}\KeywordTok{:}\AttributeTok{ express}
\AttributeTok{    }\FunctionTok{style}\KeywordTok{:}\AttributeTok{ headline}
\AttributeTok{    }
\AttributeTok{  }\FunctionTok{google{-}analytics}\KeywordTok{:}
\AttributeTok{    }\FunctionTok{tracking{-}id}\KeywordTok{:}\AttributeTok{ }\StringTok{"G{-}XXXXXXXXXX"}
\AttributeTok{    }\FunctionTok{anonymize{-}ip}\KeywordTok{:}\AttributeTok{ }\CharTok{true}
\end{Highlighting}
\end{Shaded}

\textbf{Resultado:} Google Analytics solo se cargará si el usuario
acepta las cookies de tracking.

\subsection{Comentarios}\label{comentarios}

Integración de sistemas de comentarios.

\subsubsection{Utterances (GitHub
Issues)}\label{utterances-github-issues}

\begin{Shaded}
\begin{Highlighting}[]
\FunctionTok{website}\KeywordTok{:}
\AttributeTok{  }\FunctionTok{comments}\KeywordTok{:}
\AttributeTok{    }\FunctionTok{utterances}\KeywordTok{:}
\AttributeTok{      }\FunctionTok{repo}\KeywordTok{:}\AttributeTok{ usuario/repositorio}
\AttributeTok{      }\FunctionTok{issue{-}term}\KeywordTok{:}\AttributeTok{ title}\CommentTok{              \# title, pathname, url, og:title}
\AttributeTok{      }\FunctionTok{theme}\KeywordTok{:}\AttributeTok{ github{-}light}\CommentTok{            \# Tema}
\AttributeTok{      }\FunctionTok{label}\KeywordTok{:}\AttributeTok{ }\StringTok{"comments"}\CommentTok{              \# Etiqueta en issues}
\end{Highlighting}
\end{Shaded}

\textbf{Opciones de \texttt{issue-term}:}

\begin{itemize}
\tightlist
\item
  \texttt{title}: Usa el título del post como título del issue
\item
  \texttt{pathname}: Usa la ruta del post
\item
  \texttt{url}: Usa la URL completa
\item
  \texttt{og:title}: Usa el título de Open Graph
\end{itemize}

\textbf{Temas disponibles:}

\begin{itemize}
\tightlist
\item
  \texttt{github-light}, \texttt{github-dark}
\item
  \texttt{preferred-color-scheme} (automático)
\item
  \texttt{github-dark-orange}, \texttt{icy-dark}, \texttt{dark-blue},
  \texttt{photon-dark}
\item
  \texttt{boxy-light}, \texttt{gruvbox-dark}
\end{itemize}

\subsubsection{Giscus (GitHub
Discussions)}\label{giscus-github-discussions}

\begin{Shaded}
\begin{Highlighting}[]
\FunctionTok{website}\KeywordTok{:}
\AttributeTok{  }\FunctionTok{comments}\KeywordTok{:}
\AttributeTok{    }\FunctionTok{giscus}\KeywordTok{:}
\AttributeTok{      }\FunctionTok{repo}\KeywordTok{:}\AttributeTok{ usuario/repositorio}
\AttributeTok{      }\FunctionTok{repo{-}id}\KeywordTok{:}\AttributeTok{ }\StringTok{"R\_xxxxx"}
\AttributeTok{      }\FunctionTok{category}\KeywordTok{:}\AttributeTok{ }\StringTok{"Announcements"}
\AttributeTok{      }\FunctionTok{category{-}id}\KeywordTok{:}\AttributeTok{ }\StringTok{"DIC\_xxxxx"}
\AttributeTok{      }\FunctionTok{mapping}\KeywordTok{:}\AttributeTok{ pathname}
\AttributeTok{      }\FunctionTok{reactions{-}enabled}\KeywordTok{:}\AttributeTok{ }\CharTok{true}
\AttributeTok{      }\FunctionTok{theme}\KeywordTok{:}\AttributeTok{ light}
\end{Highlighting}
\end{Shaded}

\subsubsection{Hypothesis}\label{hypothesis}

\begin{Shaded}
\begin{Highlighting}[]
\FunctionTok{website}\KeywordTok{:}
\AttributeTok{  }\FunctionTok{comments}\KeywordTok{:}
\AttributeTok{    }\FunctionTok{hypothesis}\KeywordTok{:}\AttributeTok{ }\CharTok{true}
\end{Highlighting}
\end{Shaded}

\subsection{Headers y Footers
Personalizados}\label{headers-y-footers-personalizados}

Agregar contenido Markdown en posiciones específicas.

\begin{Shaded}
\begin{Highlighting}[]
\FunctionTok{website}\KeywordTok{:}
\FunctionTok{  body{-}header}\KeywordTok{: }\CharTok{|}
\NormalTok{    Esta página es parte de [Mi Proyecto](https://example.com)}
    
\FunctionTok{  body{-}footer}\KeywordTok{: }\CharTok{|}
\NormalTok{    © 2026 Mi Nombre}
    
\FunctionTok{  margin{-}header}\KeywordTok{: }\CharTok{|}
\NormalTok{    ![Logo](/assets/img/logo.png)}
    
\FunctionTok{  margin{-}footer}\KeywordTok{: }\CharTok{|}
\NormalTok{    *Última actualización: 2026{-}01{-}22*}
\end{Highlighting}
\end{Shaded}

\textbf{Ubicaciones:}

\begin{itemize}
\tightlist
\item
  \texttt{body-header}: Inicio del cuerpo (debajo del título)
\item
  \texttt{body-footer}: Final del cuerpo
\item
  \texttt{margin-header}: Arriba del margen derecho (TOC)
\item
  \texttt{margin-footer}: Debajo del margen derecho
\end{itemize}

\section{Sección: Format}\label{secciuxf3n-format}

Define los formatos de salida y sus opciones.

\subsection{Formato HTML}\label{formato-html}

\subsubsection{Configuración Básica}\label{configuraciuxf3n-buxe1sica-2}

\begin{Shaded}
\begin{Highlighting}[]
\FunctionTok{format}\KeywordTok{:}
\AttributeTok{  }\FunctionTok{html}\KeywordTok{:}
\AttributeTok{    }\FunctionTok{theme}\KeywordTok{:}\AttributeTok{ cosmo}
\AttributeTok{    }\FunctionTok{toc}\KeywordTok{:}\AttributeTok{ }\CharTok{true}
\AttributeTok{    }\FunctionTok{code{-}fold}\KeywordTok{:}\AttributeTok{ }\CharTok{true}
\end{Highlighting}
\end{Shaded}

\subsubsection{Tema}\label{tema}

\textbf{Temas predefinidos de Bootswatch:}

\begin{Shaded}
\begin{Highlighting}[]
\FunctionTok{format}\KeywordTok{:}
\AttributeTok{  }\FunctionTok{html}\KeywordTok{:}
\AttributeTok{    }\FunctionTok{theme}\KeywordTok{:}\AttributeTok{ cosmo}
\end{Highlighting}
\end{Shaded}

\textbf{Temas disponibles:}

\begin{itemize}
\tightlist
\item
  \texttt{default}, \texttt{cerulean}, \texttt{cosmo}, \texttt{cyborg},
  \texttt{darkly}, \texttt{flatly}
\item
  \texttt{journal}, \texttt{litera}, \texttt{lumen}, \texttt{lux},
  \texttt{materia}, \texttt{minty}
\item
  \texttt{morph}, \texttt{pulse}, \texttt{quartz}, \texttt{sandstone},
  \texttt{simplex}, \texttt{sketchy}
\item
  \texttt{slate}, \texttt{solar}, \texttt{spacelab}, \texttt{superhero},
  \texttt{united}, \texttt{vapor}
\item
  \texttt{yeti}, \texttt{zephyr}
\end{itemize}

\textbf{Tema claro y oscuro:}

\begin{Shaded}
\begin{Highlighting}[]
\FunctionTok{format}\KeywordTok{:}
\AttributeTok{  }\FunctionTok{html}\KeywordTok{:}
\AttributeTok{    }\FunctionTok{theme}\KeywordTok{:}
\AttributeTok{      }\FunctionTok{light}\KeywordTok{:}\AttributeTok{ flatly}
\AttributeTok{      }\FunctionTok{dark}\KeywordTok{:}\AttributeTok{ darkly}
\end{Highlighting}
\end{Shaded}

\textbf{Tema personalizado:}

\begin{Shaded}
\begin{Highlighting}[]
\FunctionTok{format}\KeywordTok{:}
\AttributeTok{  }\FunctionTok{html}\KeywordTok{:}
\AttributeTok{    }\FunctionTok{theme}\KeywordTok{:}
\AttributeTok{      }\KeywordTok{{-}}\AttributeTok{ cosmo}
\AttributeTok{      }\KeywordTok{{-}}\AttributeTok{ assets/custom.scss}
\end{Highlighting}
\end{Shaded}

\subsubsection{CSS Adicional}\label{css-adicional}

\begin{Shaded}
\begin{Highlighting}[]
\FunctionTok{format}\KeywordTok{:}
\AttributeTok{  }\FunctionTok{html}\KeywordTok{:}
\AttributeTok{    }\FunctionTok{css}\KeywordTok{:}\AttributeTok{ assets/styles.css}
\CommentTok{    \# O múltiples archivos:}
\AttributeTok{    }\FunctionTok{css}\KeywordTok{:}
\AttributeTok{      }\KeywordTok{{-}}\AttributeTok{ assets/styles.css}
\AttributeTok{      }\KeywordTok{{-}}\AttributeTok{ assets/colors.css}
\end{Highlighting}
\end{Shaded}

\subsubsection{Tabla de Contenidos}\label{tabla-de-contenidos}

\begin{Shaded}
\begin{Highlighting}[]
\FunctionTok{format}\KeywordTok{:}
\AttributeTok{  }\FunctionTok{html}\KeywordTok{:}
\AttributeTok{    }\FunctionTok{toc}\KeywordTok{:}\AttributeTok{ }\CharTok{true}
\AttributeTok{    }\FunctionTok{toc{-}depth}\KeywordTok{:}\AttributeTok{ }\DecValTok{3}
\AttributeTok{    }\FunctionTok{toc{-}expand}\KeywordTok{:}\AttributeTok{ }\DecValTok{2}
\AttributeTok{    }\FunctionTok{toc{-}title}\KeywordTok{:}\AttributeTok{ }\StringTok{"Contenidos"}
\AttributeTok{    }\FunctionTok{toc{-}location}\KeywordTok{:}\AttributeTok{ left}\CommentTok{               \# left, right, body}
\end{Highlighting}
\end{Shaded}

\subsubsection{Código}\label{cuxf3digo}

\begin{Shaded}
\begin{Highlighting}[]
\FunctionTok{format}\KeywordTok{:}
\AttributeTok{  }\FunctionTok{html}\KeywordTok{:}
\AttributeTok{    }\FunctionTok{code{-}fold}\KeywordTok{:}\AttributeTok{ }\CharTok{true}
\AttributeTok{    }\FunctionTok{code{-}summary}\KeywordTok{:}\AttributeTok{ }\StringTok{"Mostrar código"}
\AttributeTok{    }\FunctionTok{code{-}copy}\KeywordTok{:}\AttributeTok{ }\CharTok{true}
\AttributeTok{    }\FunctionTok{code{-}overflow}\KeywordTok{:}\AttributeTok{ scroll}\CommentTok{            \# scroll, wrap}
\AttributeTok{    }\FunctionTok{code{-}line{-}numbers}\KeywordTok{:}\AttributeTok{ }\CharTok{true}
\AttributeTok{    }\FunctionTok{highlight{-}style}\KeywordTok{:}\AttributeTok{ github}
\end{Highlighting}
\end{Shaded}

\subsubsection{Navegación}\label{navegaciuxf3n}

\begin{Shaded}
\begin{Highlighting}[]
\FunctionTok{format}\KeywordTok{:}
\AttributeTok{  }\FunctionTok{html}\KeywordTok{:}
\AttributeTok{    }\FunctionTok{page{-}navigation}\KeywordTok{:}\AttributeTok{ }\CharTok{true}\CommentTok{            \# Botones Siguiente/Anterior}
\AttributeTok{    }\FunctionTok{smooth{-}scroll}\KeywordTok{:}\AttributeTok{ }\CharTok{true}
\AttributeTok{    }\FunctionTok{link{-}external{-}newwindow}\KeywordTok{:}\AttributeTok{ }\CharTok{true}
\AttributeTok{    }\FunctionTok{link{-}external{-}icon}\KeywordTok{:}\AttributeTok{ }\CharTok{true}
\end{Highlighting}
\end{Shaded}

\subsubsection{Includes (Archivos
Adicionales)}\label{includes-archivos-adicionales}

\begin{Shaded}
\begin{Highlighting}[]
\FunctionTok{format}\KeywordTok{:}
\AttributeTok{  }\FunctionTok{html}\KeywordTok{:}
\AttributeTok{    }\FunctionTok{include{-}in{-}header}\KeywordTok{:}\AttributeTok{ assets/gtm{-}head.html}
\AttributeTok{    }\FunctionTok{include{-}before{-}body}\KeywordTok{:}\AttributeTok{ assets/banner.html}
\AttributeTok{    }\FunctionTok{include{-}after{-}body}\KeywordTok{:}\AttributeTok{ assets/gtm{-}body.html}
\end{Highlighting}
\end{Shaded}

\textbf{Casos de uso:}

\begin{itemize}
\tightlist
\item
  Google Tag Manager
\item
  Scripts personalizados
\item
  Analytics adicionales
\item
  Banners o avisos
\end{itemize}

\subsubsection{Configuración Completa de
HTML}\label{configuraciuxf3n-completa-de-html}

\begin{Shaded}
\begin{Highlighting}[]
\FunctionTok{format}\KeywordTok{:}
\AttributeTok{  }\FunctionTok{html}\KeywordTok{:}
\CommentTok{    \# Tema}
\AttributeTok{    }\FunctionTok{theme}\KeywordTok{:}
\AttributeTok{      }\FunctionTok{light}\KeywordTok{:}\AttributeTok{ cosmo}
\AttributeTok{      }\FunctionTok{dark}\KeywordTok{:}\AttributeTok{ darkly}
\AttributeTok{    }\FunctionTok{css}\KeywordTok{:}\AttributeTok{ assets/styles.css}
\AttributeTok{    }
\CommentTok{    \# Tabla de contenidos}
\AttributeTok{    }\FunctionTok{toc}\KeywordTok{:}\AttributeTok{ }\CharTok{true}
\AttributeTok{    }\FunctionTok{toc{-}depth}\KeywordTok{:}\AttributeTok{ }\DecValTok{3}
\AttributeTok{    }\FunctionTok{toc{-}expand}\KeywordTok{:}\AttributeTok{ }\DecValTok{2}
\AttributeTok{    }\FunctionTok{toc{-}location}\KeywordTok{:}\AttributeTok{ left}
\AttributeTok{    }
\CommentTok{    \# Código}
\AttributeTok{    }\FunctionTok{code{-}fold}\KeywordTok{:}\AttributeTok{ }\CharTok{true}
\AttributeTok{    }\FunctionTok{code{-}copy}\KeywordTok{:}\AttributeTok{ }\CharTok{true}
\AttributeTok{    }\FunctionTok{code{-}line{-}numbers}\KeywordTok{:}\AttributeTok{ }\CharTok{true}
\AttributeTok{    }\FunctionTok{code{-}overflow}\KeywordTok{:}\AttributeTok{ scroll}
\AttributeTok{    }\FunctionTok{highlight{-}style}\KeywordTok{:}\AttributeTok{ github}
\AttributeTok{    }
\CommentTok{    \# Navegación}
\AttributeTok{    }\FunctionTok{page{-}navigation}\KeywordTok{:}\AttributeTok{ }\CharTok{true}
\AttributeTok{    }\FunctionTok{smooth{-}scroll}\KeywordTok{:}\AttributeTok{ }\CharTok{true}
\AttributeTok{    }\FunctionTok{link{-}external{-}newwindow}\KeywordTok{:}\AttributeTok{ }\CharTok{true}
\AttributeTok{    }
\CommentTok{    \# Citas}
\AttributeTok{    }\FunctionTok{citations{-}hover}\KeywordTok{:}\AttributeTok{ }\CharTok{true}
\AttributeTok{    }\FunctionTok{footnotes{-}hover}\KeywordTok{:}\AttributeTok{ }\CharTok{true}
\AttributeTok{    }
\CommentTok{    \# Includes}
\AttributeTok{    }\FunctionTok{include{-}in{-}header}\KeywordTok{:}\AttributeTok{ assets/gtm{-}head.html}
\AttributeTok{    }\FunctionTok{include{-}after{-}body}\KeywordTok{:}\AttributeTok{ assets/gtm{-}body.html}
\AttributeTok{    }
\CommentTok{    \# Grid (ancho de columnas)}
\AttributeTok{    }\FunctionTok{grid}\KeywordTok{:}
\AttributeTok{      }\FunctionTok{body{-}width}\KeywordTok{:}\AttributeTok{ 1000px}
\AttributeTok{      }\FunctionTok{sidebar{-}width}\KeywordTok{:}\AttributeTok{ 300px}
\AttributeTok{      }\FunctionTok{margin{-}width}\KeywordTok{:}\AttributeTok{ 200px}
\end{Highlighting}
\end{Shaded}

\subsection{Formato PDF}\label{formato-pdf}

\begin{Shaded}
\begin{Highlighting}[]
\FunctionTok{format}\KeywordTok{:}
\AttributeTok{  }\FunctionTok{pdf}\KeywordTok{:}
\AttributeTok{    }\FunctionTok{documentclass}\KeywordTok{:}\AttributeTok{ article}
\AttributeTok{    }\FunctionTok{papersize}\KeywordTok{:}\AttributeTok{ a4}
\AttributeTok{    }\FunctionTok{toc}\KeywordTok{:}\AttributeTok{ }\CharTok{true}
\AttributeTok{    }\FunctionTok{number{-}sections}\KeywordTok{:}\AttributeTok{ }\CharTok{true}
\AttributeTok{    }\FunctionTok{colorlinks}\KeywordTok{:}\AttributeTok{ }\CharTok{true}
\AttributeTok{    }\FunctionTok{pdf{-}engine}\KeywordTok{:}\AttributeTok{ xelatex}
\end{Highlighting}
\end{Shaded}

\subsection{Formato DOCX}\label{formato-docx}

\begin{Shaded}
\begin{Highlighting}[]
\FunctionTok{format}\KeywordTok{:}
\AttributeTok{  }\FunctionTok{docx}\KeywordTok{:}
\AttributeTok{    }\FunctionTok{toc}\KeywordTok{:}\AttributeTok{ }\CharTok{true}
\AttributeTok{    }\FunctionTok{number{-}sections}\KeywordTok{:}\AttributeTok{ }\CharTok{true}
\AttributeTok{    }\FunctionTok{reference{-}doc}\KeywordTok{:}\AttributeTok{ template.docx}
\end{Highlighting}
\end{Shaded}

\subsection{Múltiples Formatos}\label{muxfaltiples-formatos}

\begin{Shaded}
\begin{Highlighting}[]
\FunctionTok{format}\KeywordTok{:}
\AttributeTok{  }\FunctionTok{html}\KeywordTok{:}
\AttributeTok{    }\FunctionTok{theme}\KeywordTok{:}\AttributeTok{ cosmo}
\AttributeTok{    }\FunctionTok{toc}\KeywordTok{:}\AttributeTok{ }\CharTok{true}
\AttributeTok{    }
\AttributeTok{  }\FunctionTok{pdf}\KeywordTok{:}
\AttributeTok{    }\FunctionTok{documentclass}\KeywordTok{:}\AttributeTok{ article}
\AttributeTok{    }\FunctionTok{toc}\KeywordTok{:}\AttributeTok{ }\CharTok{true}
\AttributeTok{    }
\AttributeTok{  }\FunctionTok{docx}\KeywordTok{:}
\AttributeTok{    }\FunctionTok{toc}\KeywordTok{:}\AttributeTok{ }\CharTok{true}
\end{Highlighting}
\end{Shaded}

\section{Configuraciones Avanzadas}\label{configuraciones-avanzadas}

\subsection{Filtros}\label{filtros}

\begin{Shaded}
\begin{Highlighting}[]
\FunctionTok{filters}\KeywordTok{:}
\AttributeTok{  }\KeywordTok{{-}}\AttributeTok{ quarto}
\end{Highlighting}
\end{Shaded}

\subsection{Metadata Compartidos}\label{metadata-compartidos}

\begin{Shaded}
\begin{Highlighting}[]
\FunctionTok{author}\KeywordTok{:}\AttributeTok{ }\StringTok{"Edison Achalma"}
\FunctionTok{date}\KeywordTok{:}\AttributeTok{ }\StringTok{"2026{-}01{-}02"}
\FunctionTok{lang}\KeywordTok{:}\AttributeTok{ es}
\end{Highlighting}
\end{Shaded}

\textbf{Estos metadatos se aplican a todos los documentos.}

\subsection{Bibliografía Global}\label{bibliografuxeda-global}

\begin{Shaded}
\begin{Highlighting}[]
\FunctionTok{bibliography}\KeywordTok{:}\AttributeTok{ references.bib}
\FunctionTok{csl}\KeywordTok{:}\AttributeTok{ apa{-}6th{-}edition.csl}
\end{Highlighting}
\end{Shaded}

\subsection{Variables de Entorno}\label{variables-de-entorno}

\begin{Shaded}
\begin{Highlighting}[]
\FunctionTok{environment}\KeywordTok{:}
\AttributeTok{  }\FunctionTok{QUARTO\_PYTHON}\KeywordTok{:}\AttributeTok{ /usr/bin/python3}
\end{Highlighting}
\end{Shaded}

\section{Análisis de Tu
Configuración}\label{anuxe1lisis-de-tu-configuraciuxf3n}

Voy a analizar tu \texttt{\_quarto.yml} sección por sección.

\subsection{Tu Configuración Actual}\label{tu-configuraciuxf3n-actual}

\begin{Shaded}
\begin{Highlighting}[]
\FunctionTok{project}\KeywordTok{:}
\AttributeTok{  }\FunctionTok{type}\KeywordTok{:}\AttributeTok{ website}
\AttributeTok{  }\FunctionTok{output{-}dir}\KeywordTok{:}\AttributeTok{ \_site}
\AttributeTok{  }\FunctionTok{resources}\KeywordTok{:}
\AttributeTok{    }\KeywordTok{{-}}\AttributeTok{ assets/img/sidebar.jpg}

\FunctionTok{website}\KeywordTok{:}
\AttributeTok{  }\FunctionTok{title}\KeywordTok{:}\AttributeTok{ }\StringTok{"Edison Achalma B.Sc. Econ."}
\AttributeTok{  }\FunctionTok{description}\KeywordTok{:}\AttributeTok{ }\StringTok{"Investigador y educador que aplica la ciencia de datos }\SpecialCharTok{\textbackslash{} }
\StringTok{  de forma que se dé prioridad a la equidad social."}
\AttributeTok{  }\FunctionTok{favicon}\KeywordTok{:}\AttributeTok{ assets/img/favicon.png }
\AttributeTok{  }\FunctionTok{image}\KeywordTok{:}\AttributeTok{ /assets/img/default{-}preview.jpg}
\AttributeTok{  }
\AttributeTok{  }\FunctionTok{announcement}\KeywordTok{:}
\AttributeTok{    }\FunctionTok{content}\KeywordTok{:}\AttributeTok{ }\StringTok{"¡Bienvenidos a mi mundo! ¡Felices fiestas! Gracias por  }\SpecialCharTok{\textbackslash{} }
\StringTok{    visitar 🎄"}
\AttributeTok{    }\FunctionTok{icon}\KeywordTok{:}\AttributeTok{ gift}
\AttributeTok{    }\FunctionTok{type}\KeywordTok{:}\AttributeTok{ primary}
\AttributeTok{    }\FunctionTok{dismissable}\KeywordTok{:}\AttributeTok{ }\CharTok{true}
\AttributeTok{    }\FunctionTok{position}\KeywordTok{:}\AttributeTok{ below{-}navbar}
\AttributeTok{  }
\AttributeTok{  }\FunctionTok{open{-}graph}\KeywordTok{:}
\AttributeTok{    }\FunctionTok{title}\KeywordTok{:}\AttributeTok{ }\StringTok{"Edison Achalma B.Sc. Econ."}
\AttributeTok{    }\FunctionTok{description}\KeywordTok{:}\AttributeTok{ }\StringTok{"Investigador y educador que aplica la ciencia de    }\SpecialCharTok{\textbackslash{} }
\StringTok{    datos de forma que se dé prioridad a la equidad social."}
\AttributeTok{    }\FunctionTok{image}\KeywordTok{:}\AttributeTok{ /assets/img/sidebar.jpg}
\AttributeTok{    }\FunctionTok{image{-}width}\KeywordTok{:}\AttributeTok{ }\DecValTok{1200}
\AttributeTok{    }\FunctionTok{image{-}height}\KeywordTok{:}\AttributeTok{ }\DecValTok{630}
\AttributeTok{    }\FunctionTok{locale}\KeywordTok{:}\AttributeTok{ es\_ES}
\AttributeTok{    }\FunctionTok{site{-}name}\KeywordTok{:}\AttributeTok{ }\StringTok{"Actus mercator"}
\AttributeTok{    }
\AttributeTok{  }\FunctionTok{twitter{-}card}\KeywordTok{:}
\AttributeTok{    }\FunctionTok{creator}\KeywordTok{:}\AttributeTok{ }\StringTok{"@achalmaedison"}
\AttributeTok{    }\FunctionTok{card{-}style}\KeywordTok{:}\AttributeTok{ summary\_large\_image}
\AttributeTok{    }
\AttributeTok{  }\FunctionTok{comments}\KeywordTok{:}
\AttributeTok{    }\FunctionTok{utterances}\KeywordTok{:}
\AttributeTok{      }\FunctionTok{repo}\KeywordTok{:}\AttributeTok{ achalmed/methodica}
\AttributeTok{      }\FunctionTok{label}\KeywordTok{:}\AttributeTok{ utterances}
\AttributeTok{      }\FunctionTok{theme}\KeywordTok{:}\AttributeTok{ body{-}light}
\AttributeTok{      }\FunctionTok{issue{-}term}\KeywordTok{:}\AttributeTok{ title}
\AttributeTok{      }
\AttributeTok{  }\FunctionTok{site{-}url}\KeywordTok{:}\AttributeTok{ https://methodica.netlify.app/}
\AttributeTok{  }\FunctionTok{repo{-}url}\KeywordTok{:}\AttributeTok{ https://github.com/achalmed/methodica}
\AttributeTok{  }
\AttributeTok{  }\FunctionTok{navbar}\KeywordTok{:}
\AttributeTok{    }\FunctionTok{right}\KeywordTok{:}
\AttributeTok{      }\KeywordTok{{-}}\AttributeTok{ }\FunctionTok{text}\KeywordTok{:}\AttributeTok{ About}
\AttributeTok{        }\FunctionTok{aria{-}label}\KeywordTok{:}\AttributeTok{ }\StringTok{"About Me"}
\AttributeTok{        }\FunctionTok{href}\KeywordTok{:}\AttributeTok{ https://achalmaedison.netlify.app/about/}
\AttributeTok{      }\KeywordTok{{-}}\AttributeTok{ }\FunctionTok{text}\KeywordTok{:}\AttributeTok{ }\StringTok{"More"}
\AttributeTok{        }\FunctionTok{aria{-}label}\KeywordTok{:}\AttributeTok{ }\StringTok{"More"}
\AttributeTok{        }\FunctionTok{icon}\KeywordTok{:}\AttributeTok{ ellipsis{-}h}
\AttributeTok{        }\FunctionTok{menu}\KeywordTok{:}
\AttributeTok{          }\KeywordTok{{-}}\AttributeTok{ }\FunctionTok{text}\KeywordTok{:}\AttributeTok{ }\StringTok{"Econometria"}
\AttributeTok{            }\FunctionTok{href}\KeywordTok{:}\AttributeTok{ https://epsilon{-}y{-}beta.netlify.app/}
\CommentTok{          \# ... más items}
\AttributeTok{      }\KeywordTok{{-}}\AttributeTok{ }\FunctionTok{icon}\KeywordTok{:}\AttributeTok{ github}
\AttributeTok{        }\FunctionTok{href}\KeywordTok{:}\AttributeTok{ https://github.com/achalmed}
\AttributeTok{      }\KeywordTok{{-}}\AttributeTok{ }\FunctionTok{icon}\KeywordTok{:}\AttributeTok{ twitter}
\AttributeTok{        }\FunctionTok{href}\KeywordTok{:}\AttributeTok{ https://x.com/achalmaedison}
\AttributeTok{      }\KeywordTok{{-}}\AttributeTok{ }\FunctionTok{icon}\KeywordTok{:}\AttributeTok{ rss}
\AttributeTok{        }\FunctionTok{href}\KeywordTok{:}\AttributeTok{ index.xml}
\AttributeTok{        }
\AttributeTok{  }\FunctionTok{page{-}footer}\KeywordTok{:}
\FunctionTok{    left}\KeywordTok{: }\CharTok{\textgreater{}{-}}
\NormalTok{      © 2026 Edison Achalma · Made with [Quarto](https://quarto.org)}
\FunctionTok{    center}\KeywordTok{: }\CharTok{|}
\NormalTok{      \textless{}a class="link{-}dark me{-}1" href="/accessibility.html"\textgreater{}\textless{}/a\textgreater{}}
\NormalTok{      \# ... más enlaces}
\AttributeTok{    }\FunctionTok{right}\KeywordTok{:}
\AttributeTok{      }\KeywordTok{{-}}\AttributeTok{ }\FunctionTok{text}\KeywordTok{:}\AttributeTok{ }\StringTok{"Accessibility"}
\AttributeTok{        }\FunctionTok{href}\KeywordTok{:}\AttributeTok{ https://achalmaedison.netlify.app/accessibility}
\CommentTok{      \# ... más items}

\FunctionTok{format}\KeywordTok{:}
\AttributeTok{  }\FunctionTok{html}\KeywordTok{:}
\AttributeTok{    }\FunctionTok{theme}\KeywordTok{:}
\AttributeTok{      }\FunctionTok{light}\KeywordTok{:}
\AttributeTok{        }\KeywordTok{{-}}\AttributeTok{ cosmo}
\AttributeTok{        }\KeywordTok{{-}}\AttributeTok{ assets/theme\_light.scss}
\AttributeTok{        }\KeywordTok{{-}}\AttributeTok{ assets/colors.scss}
\AttributeTok{        }\KeywordTok{{-}}\AttributeTok{ assets/fonts.scss}
\AttributeTok{    }\FunctionTok{css}\KeywordTok{:}\AttributeTok{ assets/styles.css}
\AttributeTok{    }\FunctionTok{include{-}in{-}header}\KeywordTok{:}\AttributeTok{ assets/gtm{-}head.html}
\AttributeTok{    }\FunctionTok{include{-}after{-}body}\KeywordTok{:}\AttributeTok{ assets/gtm{-}body.html}
\end{Highlighting}
\end{Shaded}

\subsection{Análisis Detallado}\label{anuxe1lisis-detallado}

\subsubsection{Sección: Project}\label{secciuxf3n-project-1}

\textbf{Estado:} Correcto y bien configurado.

\begin{Shaded}
\begin{Highlighting}[]
\FunctionTok{project}\KeywordTok{:}
\AttributeTok{  }\FunctionTok{type}\KeywordTok{:}\AttributeTok{ website}\CommentTok{                      \# Correcto}
\AttributeTok{  }\FunctionTok{output{-}dir}\KeywordTok{:}\AttributeTok{ \_site}\CommentTok{                  \# Estándar}
\AttributeTok{  }\FunctionTok{resources}\KeywordTok{:}
\AttributeTok{    }\KeywordTok{{-}}\AttributeTok{ assets/img/sidebar.jpg}\CommentTok{         \# Solo un archivo}
\end{Highlighting}
\end{Shaded}

\textbf{Sugerencias:}

\begin{itemize}
\tightlist
\item
  Considera agregar otros recursos si los necesitas:
\end{itemize}

\begin{Shaded}
\begin{Highlighting}[]
\FunctionTok{resources}\KeywordTok{:}
\AttributeTok{  }\KeywordTok{{-}}\AttributeTok{ assets/img/sidebar.jpg}
\AttributeTok{  }\KeywordTok{{-}}\AttributeTok{ }\StringTok{"data/"}\CommentTok{                          \# Si tienes datasets}
\AttributeTok{  }\KeywordTok{{-}}\AttributeTok{ }\StringTok{"CNAME"}\CommentTok{                          \# Si usas dominio personalizado}
\end{Highlighting}
\end{Shaded}

\subsubsection{Sección: Website - Información
Básica}\label{secciuxf3n-website---informaciuxf3n-buxe1sica}

\textbf{Estado:} Excelente.

\begin{Shaded}
\begin{Highlighting}[]
\FunctionTok{title}\KeywordTok{:}\AttributeTok{ }\StringTok{"Edison Achalma B.Sc. Econ."}\CommentTok{              \# Bien}
\FunctionTok{description}\KeywordTok{:}\AttributeTok{ }\StringTok{"..."}\CommentTok{                                \# Bien (descriptiva)}
\FunctionTok{site{-}url}\KeywordTok{:}\AttributeTok{ https://methodica.netlify.app/}\CommentTok{         \# Correcto}
\FunctionTok{repo{-}url}\KeywordTok{:}\AttributeTok{ https://github.com/achalmed/methodica}\CommentTok{  \# Bien}
\FunctionTok{favicon}\KeywordTok{:}\AttributeTok{ assets/img/favicon.png}\CommentTok{                  \# Correcto}
\FunctionTok{image}\KeywordTok{:}\AttributeTok{ /assets/img/default{-}preview.jpg}\CommentTok{           \# Bien}
\end{Highlighting}
\end{Shaded}

\textbf{Nota importante:} La ruta de \texttt{image} comienza con
\texttt{/} (absoluta desde la raíz), lo cual es correcto.

\subsubsection{Anuncio}\label{anuncio}

\textbf{Estado:} Bien configurado.

\begin{Shaded}
\begin{Highlighting}[]
\FunctionTok{announcement}\KeywordTok{:}
\AttributeTok{  }\FunctionTok{content}\KeywordTok{:}\AttributeTok{ }\StringTok{"¡Bienvenidos a mi mundo! Gracias por visitar 🎄"}
\AttributeTok{  }\FunctionTok{icon}\KeywordTok{:}\AttributeTok{ gift}
\AttributeTok{  }\FunctionTok{type}\KeywordTok{:}\AttributeTok{ primary}
\AttributeTok{  }\FunctionTok{dismissable}\KeywordTok{:}\AttributeTok{ }\CharTok{true}
\AttributeTok{  }\FunctionTok{position}\KeywordTok{:}\AttributeTok{ below{-}navbar}
\end{Highlighting}
\end{Shaded}

\textbf{Sugerencia:} El mensaje parece temporal (fiestas). Considera
actualizarlo o quitarlo después de las fiestas.

\textbf{Alternativa neutra:}

\begin{Shaded}
\begin{Highlighting}[]
\FunctionTok{announcement}\KeywordTok{:}
\AttributeTok{  }\FunctionTok{icon}\KeywordTok{:}\AttributeTok{ info{-}circle}
\AttributeTok{  }\FunctionTok{content}\KeywordTok{:}\AttributeTok{ }\StringTok{"**Nuevo:** Explora mi colección de tutoriales de econometría"}
\AttributeTok{  }\FunctionTok{type}\KeywordTok{:}\AttributeTok{ info}
\AttributeTok{  }\FunctionTok{dismissable}\KeywordTok{:}\AttributeTok{ }\CharTok{true}
\end{Highlighting}
\end{Shaded}

\subsubsection{Open Graph}\label{open-graph-1}

\textbf{Estado:} Excelente configuración.

\begin{Shaded}
\begin{Highlighting}[]
\FunctionTok{open{-}graph}\KeywordTok{:}
\AttributeTok{  }\FunctionTok{title}\KeywordTok{:}\AttributeTok{ }\StringTok{"Edison Achalma B.Sc. Econ."}
\AttributeTok{  }\FunctionTok{description}\KeywordTok{:}\AttributeTok{ }\StringTok{"..."}
\AttributeTok{  }\FunctionTok{image}\KeywordTok{:}\AttributeTok{ /assets/img/sidebar.jpg}\CommentTok{                 \# Bien}
\AttributeTok{  }\FunctionTok{image{-}width}\KeywordTok{:}\AttributeTok{ }\DecValTok{1200}\CommentTok{                              \# Perfecto}
\AttributeTok{  }\FunctionTok{image{-}height}\KeywordTok{:}\AttributeTok{ }\DecValTok{630}\CommentTok{                              \# Perfecto (ratio 1.91:1)}
\AttributeTok{  }\FunctionTok{locale}\KeywordTok{:}\AttributeTok{ es\_ES}\CommentTok{                                  \# Correcto}
\AttributeTok{  }\FunctionTok{site{-}name}\KeywordTok{:}\AttributeTok{ }\StringTok{"Actus mercator"}\CommentTok{                    \# Bien}
\end{Highlighting}
\end{Shaded}

\textbf{Nota:} \texttt{site-name} (``Actus mercator'') difiere del
\texttt{title} del sitio (``Edison Achalma\ldots{}''). Esto es
intencional. Está bien si quieres un nombre más corto para redes
sociales.

\subsubsection{Twitter Card}\label{twitter-card}

\textbf{Estado:} Correcto.

\begin{Shaded}
\begin{Highlighting}[]
\FunctionTok{twitter{-}card}\KeywordTok{:}
\AttributeTok{  }\FunctionTok{creator}\KeywordTok{:}\AttributeTok{ }\StringTok{"@achalmaedison"}
\AttributeTok{  }\FunctionTok{card{-}style}\KeywordTok{:}\AttributeTok{ summary\_large\_image}
\end{Highlighting}
\end{Shaded}

\textbf{Sugerencia opcional:} Agregar \texttt{site}:

\begin{Shaded}
\begin{Highlighting}[]
\FunctionTok{twitter{-}card}\KeywordTok{:}
\AttributeTok{  }\FunctionTok{creator}\KeywordTok{:}\AttributeTok{ }\StringTok{"@achalmaedison"}
\AttributeTok{  }\FunctionTok{site}\KeywordTok{:}\AttributeTok{ }\StringTok{"@achalmaedison"}\CommentTok{              \# Cuenta oficial del sitio}
\AttributeTok{  }\FunctionTok{card{-}style}\KeywordTok{:}\AttributeTok{ summary\_large\_image}
\end{Highlighting}
\end{Shaded}

\subsubsection{Comentarios}\label{comentarios-1}

\textbf{Estado:} Bien configurado.

\begin{Shaded}
\begin{Highlighting}[]
\FunctionTok{comments}\KeywordTok{:}
\AttributeTok{  }\FunctionTok{utterances}\KeywordTok{:}
\AttributeTok{    }\FunctionTok{repo}\KeywordTok{:}\AttributeTok{ achalmed/methodica}
\AttributeTok{    }\FunctionTok{label}\KeywordTok{:}\AttributeTok{ utterances}
\AttributeTok{    }\FunctionTok{theme}\KeywordTok{:}\AttributeTok{ body{-}light}\CommentTok{                            \# Tema claro}
\AttributeTok{    }\FunctionTok{issue{-}term}\KeywordTok{:}\AttributeTok{ title}
\end{Highlighting}
\end{Shaded}

\textbf{Sugerencias:} 1. Verifica que el repositorio
\texttt{achalmed/methodica} existe y tiene Utterances instalado. 2. Si
usas tema claro/oscuro en tu sitio, considera
\texttt{theme:\ preferred-color-scheme} para que se adapte
automáticamente.

\begin{Shaded}
\begin{Highlighting}[]
\FunctionTok{comments}\KeywordTok{:}
\AttributeTok{  }\FunctionTok{utterances}\KeywordTok{:}
\AttributeTok{    }\FunctionTok{repo}\KeywordTok{:}\AttributeTok{ achalmed/methodica}
\AttributeTok{    }\FunctionTok{label}\KeywordTok{:}\AttributeTok{ }\StringTok{"comments"}\CommentTok{                \# Más descriptivo}
\AttributeTok{    }\FunctionTok{theme}\KeywordTok{:}\AttributeTok{ preferred{-}color{-}scheme}\CommentTok{    \# Se adapta al tema del usuario}
\AttributeTok{    }\FunctionTok{issue{-}term}\KeywordTok{:}\AttributeTok{ title}
\end{Highlighting}
\end{Shaded}

\subsubsection{Google Analytics y Cookie
Consent}\label{google-analytics-y-cookie-consent}

\textbf{Estado:} Comentado (deshabilitado).

\begin{Shaded}
\begin{Highlighting}[]
\CommentTok{\#google{-}analytics:}
\CommentTok{ \# tracking{-}id: "G{-}XGH6TP6RB3"}
\CommentTok{ \# storage: cookies}
\CommentTok{ \# anonymize{-}ip: true}

\CommentTok{\#cookie{-}consent:}
\CommentTok{ \# type: express}
\CommentTok{ \# style: simple}
\CommentTok{ \# palette: light}
\CommentTok{ \# language: es}
\CommentTok{ \# policy{-}url: https://achalmaedison.netlify.app/}
\CommentTok{ \# prefs{-}text: "Preferencias de Cookies"}
\end{Highlighting}
\end{Shaded}

\textbf{Sugerencia:} Si deseas analytics, descomenta y configura:

\begin{Shaded}
\begin{Highlighting}[]
\FunctionTok{google{-}analytics}\KeywordTok{:}
\AttributeTok{  }\FunctionTok{tracking{-}id}\KeywordTok{:}\AttributeTok{ }\StringTok{"G{-}XGH6TP6RB3"}
\AttributeTok{  }\FunctionTok{anonymize{-}ip}\KeywordTok{:}\AttributeTok{ }\CharTok{true}\CommentTok{                 \# Recomendado para privacidad}
\AttributeTok{  }
\FunctionTok{cookie{-}consent}\KeywordTok{:}
\AttributeTok{  }\FunctionTok{type}\KeywordTok{:}\AttributeTok{ express}\CommentTok{                      \# Cumplir con RGPD}
\AttributeTok{  }\FunctionTok{style}\KeywordTok{:}\AttributeTok{ headline}\CommentTok{                    \# Más visible}
\AttributeTok{  }\FunctionTok{language}\KeywordTok{:}\AttributeTok{ es}
\end{Highlighting}
\end{Shaded}

\subsubsection{Navbar}\label{navbar}

\textbf{Estado:} Bien estructurada.

\begin{Shaded}
\begin{Highlighting}[]
\FunctionTok{navbar}\KeywordTok{:}
\AttributeTok{  }\FunctionTok{right}\KeywordTok{:}
\AttributeTok{    }\KeywordTok{{-}}\AttributeTok{ }\FunctionTok{text}\KeywordTok{:}\AttributeTok{ About}
\AttributeTok{      }\FunctionTok{href}\KeywordTok{:}\AttributeTok{ https://achalmaedison.netlify.app/about/}
\AttributeTok{    }\KeywordTok{{-}}\AttributeTok{ }\FunctionTok{text}\KeywordTok{:}\AttributeTok{ }\StringTok{"More"}
\AttributeTok{      }\FunctionTok{icon}\KeywordTok{:}\AttributeTok{ ellipsis{-}h}
\AttributeTok{      }\FunctionTok{menu}\KeywordTok{:}
\AttributeTok{        }\KeywordTok{{-}}\AttributeTok{ }\FunctionTok{text}\KeywordTok{:}\AttributeTok{ }\StringTok{"Econometria"}
\AttributeTok{          }\FunctionTok{href}\KeywordTok{:}\AttributeTok{ https://epsilon{-}y{-}beta.netlify.app/}
\CommentTok{        \# ... más items}
\AttributeTok{    }\KeywordTok{{-}}\AttributeTok{ }\FunctionTok{icon}\KeywordTok{:}\AttributeTok{ github}
\AttributeTok{      }\FunctionTok{href}\KeywordTok{:}\AttributeTok{ https://github.com/achalmed}
\AttributeTok{    }\KeywordTok{{-}}\AttributeTok{ }\FunctionTok{icon}\KeywordTok{:}\AttributeTok{ twitter}
\AttributeTok{      }\FunctionTok{href}\KeywordTok{:}\AttributeTok{ https://x.com/achalmaedison}
\AttributeTok{    }\KeywordTok{{-}}\AttributeTok{ }\FunctionTok{icon}\KeywordTok{:}\AttributeTok{ rss}
\AttributeTok{      }\FunctionTok{href}\KeywordTok{:}\AttributeTok{ index.xml}
\end{Highlighting}
\end{Shaded}

\textbf{Sugerencias:} 1. Agregar \texttt{aria-label} a íconos sin texto:

\begin{Shaded}
\begin{Highlighting}[]
\KeywordTok{{-}}\AttributeTok{ }\FunctionTok{icon}\KeywordTok{:}\AttributeTok{ github}
\AttributeTok{  }\FunctionTok{href}\KeywordTok{:}\AttributeTok{ https://github.com/achalmed}
\AttributeTok{  }\FunctionTok{aria{-}label}\KeywordTok{:}\AttributeTok{ }\StringTok{"GitHub Profile"}
\end{Highlighting}
\end{Shaded}

\begin{enumerate}
\def\labelenumi{\arabic{enumi}.}
\setcounter{enumi}{1}
\tightlist
\item
  Considerar agregar items en \texttt{left} si tienes páginas
  principales del sitio:
\end{enumerate}

\begin{Shaded}
\begin{Highlighting}[]
\FunctionTok{navbar}\KeywordTok{:}
\AttributeTok{  }\FunctionTok{left}\KeywordTok{:}
\AttributeTok{    }\KeywordTok{{-}}\AttributeTok{ }\FunctionTok{text}\KeywordTok{:}\AttributeTok{ }\StringTok{"Blog"}
\AttributeTok{      }\FunctionTok{href}\KeywordTok{:}\AttributeTok{ blog/}
\AttributeTok{    }\KeywordTok{{-}}\AttributeTok{ }\FunctionTok{text}\KeywordTok{:}\AttributeTok{ }\StringTok{"Publicaciones"}
\AttributeTok{      }\FunctionTok{href}\KeywordTok{:}\AttributeTok{ publications/}
\AttributeTok{  }\FunctionTok{right}\KeywordTok{:}
\CommentTok{    \# ... (mantener lo actual)}
\end{Highlighting}
\end{Shaded}

\subsubsection{Page Footer}\label{page-footer-1}

\textbf{Estado:} Bien configurado.

\begin{Shaded}
\begin{Highlighting}[]
\FunctionTok{page{-}footer}\KeywordTok{:}
\FunctionTok{  left}\KeywordTok{: }\CharTok{\textgreater{}{-}}
\NormalTok{    © 2026 Edison Achalma · Made with [Quarto](https://quarto.org)}
\FunctionTok{  center}\KeywordTok{: }\CharTok{|}
\NormalTok{    \textless{}a class="link{-}dark me{-}1" href="/accessibility.html"\textgreater{}\textless{}/a\textgreater{}}
\NormalTok{    \# ... más íconos}
\AttributeTok{  }\FunctionTok{right}\KeywordTok{:}
\AttributeTok{    }\KeywordTok{{-}}\AttributeTok{ }\FunctionTok{text}\KeywordTok{:}\AttributeTok{ }\StringTok{"Accessibility"}
\AttributeTok{      }\FunctionTok{href}\KeywordTok{:}\AttributeTok{ https://achalmaedison.netlify.app/accessibility}
\CommentTok{    \# ... más enlaces}
\end{Highlighting}
\end{Shaded}

\textbf{Nota:} Los enlaces en \texttt{right} apuntan a
\texttt{https://achalmaedison.netlify.app/} (otro sitio). Si son páginas
de este sitio, usa rutas relativas:

\begin{Shaded}
\begin{Highlighting}[]
\FunctionTok{right}\KeywordTok{:}
\AttributeTok{  }\KeywordTok{{-}}\AttributeTok{ }\FunctionTok{text}\KeywordTok{:}\AttributeTok{ }\StringTok{"Accessibility"}
\AttributeTok{    }\FunctionTok{href}\KeywordTok{:}\AttributeTok{ accessibility.html}\CommentTok{         \# Ruta local}
\AttributeTok{  }\KeywordTok{{-}}\AttributeTok{ }\FunctionTok{text}\KeywordTok{:}\AttributeTok{ }\StringTok{"Contact"}
\AttributeTok{    }\FunctionTok{href}\KeywordTok{:}\AttributeTok{ contact.html}
\end{Highlighting}
\end{Shaded}

\subsubsection{Formato HTML}\label{formato-html-1}

\textbf{Estado:} Excelente configuración avanzada.

\begin{Shaded}
\begin{Highlighting}[]
\FunctionTok{format}\KeywordTok{:}
\AttributeTok{  }\FunctionTok{html}\KeywordTok{:}
\AttributeTok{    }\FunctionTok{theme}\KeywordTok{:}
\AttributeTok{      }\FunctionTok{light}\KeywordTok{:}
\AttributeTok{        }\KeywordTok{{-}}\AttributeTok{ cosmo}
\AttributeTok{        }\KeywordTok{{-}}\AttributeTok{ assets/theme\_light.scss}
\AttributeTok{        }\KeywordTok{{-}}\AttributeTok{ assets/colors.scss}
\AttributeTok{        }\KeywordTok{{-}}\AttributeTok{ assets/fonts.scss}
\AttributeTok{    }\FunctionTok{css}\KeywordTok{:}\AttributeTok{ assets/styles.css}
\AttributeTok{    }\FunctionTok{include{-}in{-}header}\KeywordTok{:}\AttributeTok{ assets/gtm{-}head.html}
\AttributeTok{    }\FunctionTok{include{-}after{-}body}\KeywordTok{:}\AttributeTok{ assets/gtm{-}body.html}
\end{Highlighting}
\end{Shaded}

\textbf{Observaciones:}

\begin{itemize}
\tightlist
\item
  Tema personalizado bien estructurado (base + SCSS personalizados)
\item
  Google Tag Manager integrado (gtm-head.html, gtm-body.html)
\item
  CSS adicional para estilos personalizados
\end{itemize}

\textbf{Sugerencia opcional:} Agregar tema oscuro:

\begin{Shaded}
\begin{Highlighting}[]
\FunctionTok{format}\KeywordTok{:}
\AttributeTok{  }\FunctionTok{html}\KeywordTok{:}
\AttributeTok{    }\FunctionTok{theme}\KeywordTok{:}
\AttributeTok{      }\FunctionTok{light}\KeywordTok{:}
\AttributeTok{        }\KeywordTok{{-}}\AttributeTok{ cosmo}
\AttributeTok{        }\KeywordTok{{-}}\AttributeTok{ assets/theme\_light.scss}
\AttributeTok{        }\KeywordTok{{-}}\AttributeTok{ assets/colors.scss}
\AttributeTok{        }\KeywordTok{{-}}\AttributeTok{ assets/fonts.scss}
\AttributeTok{      }\FunctionTok{dark}\KeywordTok{:}
\AttributeTok{        }\KeywordTok{{-}}\AttributeTok{ darkly}\CommentTok{                     \# Tema oscuro base}
\AttributeTok{        }\KeywordTok{{-}}\AttributeTok{ assets/theme\_dark.scss}\CommentTok{     \# Personalizaciones oscuras}
\AttributeTok{    }\FunctionTok{css}\KeywordTok{:}\AttributeTok{ assets/styles.css}
\AttributeTok{    }\FunctionTok{include{-}in{-}header}\KeywordTok{:}\AttributeTok{ assets/gtm{-}head.html}
\AttributeTok{    }\FunctionTok{include{-}after{-}body}\KeywordTok{:}\AttributeTok{ assets/gtm{-}body.html}
\end{Highlighting}
\end{Shaded}

\subsection{Configuración Optimizada}\label{configuraciuxf3n-optimizada}

\begin{Shaded}
\begin{Highlighting}[]
\CommentTok{\# ========================================================================}
\CommentTok{\# CONFIGURACIÓN DE PROYECTO}
\CommentTok{\# ========================================================================}
\FunctionTok{project}\KeywordTok{:}
\AttributeTok{  }\FunctionTok{type}\KeywordTok{:}\AttributeTok{ website}
\AttributeTok{  }\FunctionTok{output{-}dir}\KeywordTok{:}\AttributeTok{ \_site}
\AttributeTok{  }\FunctionTok{resources}\KeywordTok{:}
\AttributeTok{    }\KeywordTok{{-}}\AttributeTok{ assets/img/sidebar.jpg}
\AttributeTok{    }\KeywordTok{{-}}\AttributeTok{ }\StringTok{"CNAME"}\CommentTok{                        \# Si usas dominio personalizado}
\AttributeTok{    }\KeywordTok{{-}}\AttributeTok{ }\StringTok{".nojekyll"}\CommentTok{                    \# Para GitHub Pages}

\CommentTok{\# ========================================================================}
\CommentTok{\# CONFIGURACIÓN DEL SITIO WEB}
\CommentTok{\# ========================================================================}
\FunctionTok{website}\KeywordTok{:}
\CommentTok{  \# {-}{-}{-}{-}{-}{-}{-}{-}{-}{-}{-}{-}{-}{-}{-}{-}{-}{-}{-}{-}{-}{-}{-}{-}{-}{-}{-}{-}{-}{-}{-}{-}{-}{-}{-}{-}{-}{-}{-}{-}{-}{-}{-}{-}{-}{-}{-}{-}{-}{-}{-}{-}{-}{-}{-}{-}{-}{-}{-}{-}{-}{-}{-}{-}{-}{-}{-}{-}{-}{-}}
\CommentTok{  \# Información Básica}
\CommentTok{  \# {-}{-}{-}{-}{-}{-}{-}{-}{-}{-}{-}{-}{-}{-}{-}{-}{-}{-}{-}{-}{-}{-}{-}{-}{-}{-}{-}{-}{-}{-}{-}{-}{-}{-}{-}{-}{-}{-}{-}{-}{-}{-}{-}{-}{-}{-}{-}{-}{-}{-}{-}{-}{-}{-}{-}{-}{-}{-}{-}{-}{-}{-}{-}{-}{-}{-}{-}{-}{-}{-}}
\AttributeTok{  }\FunctionTok{title}\KeywordTok{:}\AttributeTok{ }\StringTok{"Edison Achalma B.Sc. Econ."}
\AttributeTok{  }\FunctionTok{description}\KeywordTok{:}\AttributeTok{ }\StringTok{"Investigador y educador que aplica la ciencia de datos }\SpecialCharTok{\textbackslash{} }
\StringTok{  de forma que se dé prioridad a la equidad social."}
\AttributeTok{  }\FunctionTok{site{-}url}\KeywordTok{:}\AttributeTok{ https://methodica.netlify.app/}
\AttributeTok{  }\FunctionTok{repo{-}url}\KeywordTok{:}\AttributeTok{ https://github.com/achalmed/methodica}
\AttributeTok{  }
\CommentTok{  \# {-}{-}{-}{-}{-}{-}{-}{-}{-}{-}{-}{-}{-}{-}{-}{-}{-}{-}{-}{-}{-}{-}{-}{-}{-}{-}{-}{-}{-}{-}{-}{-}{-}{-}{-}{-}{-}{-}{-}{-}{-}{-}{-}{-}{-}{-}{-}{-}{-}{-}{-}{-}{-}{-}{-}{-}{-}{-}{-}{-}{-}{-}{-}{-}{-}{-}{-}{-}{-}{-}}
\CommentTok{  \# Imágenes}
\CommentTok{  \# {-}{-}{-}{-}{-}{-}{-}{-}{-}{-}{-}{-}{-}{-}{-}{-}{-}{-}{-}{-}{-}{-}{-}{-}{-}{-}{-}{-}{-}{-}{-}{-}{-}{-}{-}{-}{-}{-}{-}{-}{-}{-}{-}{-}{-}{-}{-}{-}{-}{-}{-}{-}{-}{-}{-}{-}{-}{-}{-}{-}{-}{-}{-}{-}{-}{-}{-}{-}{-}{-}}
\AttributeTok{  }\FunctionTok{favicon}\KeywordTok{:}\AttributeTok{ assets/img/favicon.png}
\AttributeTok{  }\FunctionTok{image}\KeywordTok{:}\AttributeTok{ /assets/img/default{-}preview.jpg}
\AttributeTok{  }
\CommentTok{  \# {-}{-}{-}{-}{-}{-}{-}{-}{-}{-}{-}{-}{-}{-}{-}{-}{-}{-}{-}{-}{-}{-}{-}{-}{-}{-}{-}{-}{-}{-}{-}{-}{-}{-}{-}{-}{-}{-}{-}{-}{-}{-}{-}{-}{-}{-}{-}{-}{-}{-}{-}{-}{-}{-}{-}{-}{-}{-}{-}{-}{-}{-}{-}{-}{-}{-}{-}{-}{-}{-}}
\CommentTok{  \# Anuncio}
\CommentTok{  \# {-}{-}{-}{-}{-}{-}{-}{-}{-}{-}{-}{-}{-}{-}{-}{-}{-}{-}{-}{-}{-}{-}{-}{-}{-}{-}{-}{-}{-}{-}{-}{-}{-}{-}{-}{-}{-}{-}{-}{-}{-}{-}{-}{-}{-}{-}{-}{-}{-}{-}{-}{-}{-}{-}{-}{-}{-}{-}{-}{-}{-}{-}{-}{-}{-}{-}{-}{-}{-}{-}}
\AttributeTok{  }\FunctionTok{announcement}\KeywordTok{:}
\AttributeTok{    }\FunctionTok{icon}\KeywordTok{:}\AttributeTok{ info{-}circle}
\AttributeTok{    }\FunctionTok{content}\KeywordTok{:}\AttributeTok{ }\StringTok{"**Nuevo:** Explora la colección de recursos de econometría"}
\AttributeTok{    }\FunctionTok{type}\KeywordTok{:}\AttributeTok{ info}
\AttributeTok{    }\FunctionTok{dismissable}\KeywordTok{:}\AttributeTok{ }\CharTok{true}
\AttributeTok{    }\FunctionTok{position}\KeywordTok{:}\AttributeTok{ below{-}navbar}
\AttributeTok{  }
\CommentTok{  \# {-}{-}{-}{-}{-}{-}{-}{-}{-}{-}{-}{-}{-}{-}{-}{-}{-}{-}{-}{-}{-}{-}{-}{-}{-}{-}{-}{-}{-}{-}{-}{-}{-}{-}{-}{-}{-}{-}{-}{-}{-}{-}{-}{-}{-}{-}{-}{-}{-}{-}{-}{-}{-}{-}{-}{-}{-}{-}{-}{-}{-}{-}{-}{-}{-}{-}{-}{-}{-}{-}}
\CommentTok{  \# Open Graph (Redes Sociales)}
\CommentTok{  \# {-}{-}{-}{-}{-}{-}{-}{-}{-}{-}{-}{-}{-}{-}{-}{-}{-}{-}{-}{-}{-}{-}{-}{-}{-}{-}{-}{-}{-}{-}{-}{-}{-}{-}{-}{-}{-}{-}{-}{-}{-}{-}{-}{-}{-}{-}{-}{-}{-}{-}{-}{-}{-}{-}{-}{-}{-}{-}{-}{-}{-}{-}{-}{-}{-}{-}{-}{-}{-}{-}}
\AttributeTok{  }\FunctionTok{open{-}graph}\KeywordTok{:}
\AttributeTok{    }\FunctionTok{title}\KeywordTok{:}\AttributeTok{ }\StringTok{"Edison Achalma B.Sc. Econ."}
\AttributeTok{    }\FunctionTok{description}\KeywordTok{:}\AttributeTok{ }\StringTok{"Investigador y educador que aplica la ciencia de datos \textbackslash{}}
\StringTok{    de forma que se dé prioridad a la equidad social."}
\AttributeTok{    }\FunctionTok{image}\KeywordTok{:}\AttributeTok{ /assets/img/sidebar.jpg}
\AttributeTok{    }\FunctionTok{image{-}width}\KeywordTok{:}\AttributeTok{ }\DecValTok{1200}
\AttributeTok{    }\FunctionTok{image{-}height}\KeywordTok{:}\AttributeTok{ }\DecValTok{630}
\AttributeTok{    }\FunctionTok{locale}\KeywordTok{:}\AttributeTok{ es\_ES}
\AttributeTok{    }\FunctionTok{site{-}name}\KeywordTok{:}\AttributeTok{ }\StringTok{"Actus mercator"}
\AttributeTok{  }
\CommentTok{  \# {-}{-}{-}{-}{-}{-}{-}{-}{-}{-}{-}{-}{-}{-}{-}{-}{-}{-}{-}{-}{-}{-}{-}{-}{-}{-}{-}{-}{-}{-}{-}{-}{-}{-}{-}{-}{-}{-}{-}{-}{-}{-}{-}{-}{-}{-}{-}{-}{-}{-}{-}{-}{-}{-}{-}{-}{-}{-}{-}{-}{-}{-}{-}{-}{-}{-}{-}{-}{-}{-}}
\CommentTok{  \# Twitter Card}
\CommentTok{  \# {-}{-}{-}{-}{-}{-}{-}{-}{-}{-}{-}{-}{-}{-}{-}{-}{-}{-}{-}{-}{-}{-}{-}{-}{-}{-}{-}{-}{-}{-}{-}{-}{-}{-}{-}{-}{-}{-}{-}{-}{-}{-}{-}{-}{-}{-}{-}{-}{-}{-}{-}{-}{-}{-}{-}{-}{-}{-}{-}{-}{-}{-}{-}{-}{-}{-}{-}{-}{-}{-}}
\AttributeTok{  }\FunctionTok{twitter{-}card}\KeywordTok{:}
\AttributeTok{    }\FunctionTok{creator}\KeywordTok{:}\AttributeTok{ }\StringTok{"@achalmaedison"}
\AttributeTok{    }\FunctionTok{site}\KeywordTok{:}\AttributeTok{ }\StringTok{"@achalmaedison"}
\AttributeTok{    }\FunctionTok{card{-}style}\KeywordTok{:}\AttributeTok{ summary\_large\_image}
\AttributeTok{  }
\CommentTok{  \# {-}{-}{-}{-}{-}{-}{-}{-}{-}{-}{-}{-}{-}{-}{-}{-}{-}{-}{-}{-}{-}{-}{-}{-}{-}{-}{-}{-}{-}{-}{-}{-}{-}{-}{-}{-}{-}{-}{-}{-}{-}{-}{-}{-}{-}{-}{-}{-}{-}{-}{-}{-}{-}{-}{-}{-}{-}{-}{-}{-}{-}{-}{-}{-}{-}{-}{-}{-}{-}{-}}
\CommentTok{  \# Google Analytics}
\CommentTok{  \# {-}{-}{-}{-}{-}{-}{-}{-}{-}{-}{-}{-}{-}{-}{-}{-}{-}{-}{-}{-}{-}{-}{-}{-}{-}{-}{-}{-}{-}{-}{-}{-}{-}{-}{-}{-}{-}{-}{-}{-}{-}{-}{-}{-}{-}{-}{-}{-}{-}{-}{-}{-}{-}{-}{-}{-}{-}{-}{-}{-}{-}{-}{-}{-}{-}{-}{-}{-}{-}{-}}
\AttributeTok{  }\FunctionTok{google{-}analytics}\KeywordTok{:}
\AttributeTok{    }\FunctionTok{tracking{-}id}\KeywordTok{:}\AttributeTok{ }\StringTok{"G{-}XGH6TP6RB3"}
\AttributeTok{    }\FunctionTok{anonymize{-}ip}\KeywordTok{:}\AttributeTok{ }\CharTok{true}
\AttributeTok{  }
\CommentTok{  \# {-}{-}{-}{-}{-}{-}{-}{-}{-}{-}{-}{-}{-}{-}{-}{-}{-}{-}{-}{-}{-}{-}{-}{-}{-}{-}{-}{-}{-}{-}{-}{-}{-}{-}{-}{-}{-}{-}{-}{-}{-}{-}{-}{-}{-}{-}{-}{-}{-}{-}{-}{-}{-}{-}{-}{-}{-}{-}{-}{-}{-}{-}{-}{-}{-}{-}{-}{-}{-}{-}}
\CommentTok{  \# Cookie Consent}
\CommentTok{  \# {-}{-}{-}{-}{-}{-}{-}{-}{-}{-}{-}{-}{-}{-}{-}{-}{-}{-}{-}{-}{-}{-}{-}{-}{-}{-}{-}{-}{-}{-}{-}{-}{-}{-}{-}{-}{-}{-}{-}{-}{-}{-}{-}{-}{-}{-}{-}{-}{-}{-}{-}{-}{-}{-}{-}{-}{-}{-}{-}{-}{-}{-}{-}{-}{-}{-}{-}{-}{-}{-}}
\AttributeTok{  }\FunctionTok{cookie{-}consent}\KeywordTok{:}
\AttributeTok{    }\FunctionTok{type}\KeywordTok{:}\AttributeTok{ express}
\AttributeTok{    }\FunctionTok{style}\KeywordTok{:}\AttributeTok{ headline}
\AttributeTok{    }\FunctionTok{palette}\KeywordTok{:}\AttributeTok{ dark}
\AttributeTok{    }\FunctionTok{language}\KeywordTok{:}\AttributeTok{ es}
\AttributeTok{    }\FunctionTok{policy{-}url}\KeywordTok{:}\AttributeTok{ /privacy}
\AttributeTok{    }\FunctionTok{prefs{-}text}\KeywordTok{:}\AttributeTok{ }\StringTok{"Preferencias de Cookies"}
\AttributeTok{  }
\CommentTok{  \# {-}{-}{-}{-}{-}{-}{-}{-}{-}{-}{-}{-}{-}{-}{-}{-}{-}{-}{-}{-}{-}{-}{-}{-}{-}{-}{-}{-}{-}{-}{-}{-}{-}{-}{-}{-}{-}{-}{-}{-}{-}{-}{-}{-}{-}{-}{-}{-}{-}{-}{-}{-}{-}{-}{-}{-}{-}{-}{-}{-}{-}{-}{-}{-}{-}{-}{-}{-}{-}{-}}
\CommentTok{  \# Comentarios}
\CommentTok{  \# {-}{-}{-}{-}{-}{-}{-}{-}{-}{-}{-}{-}{-}{-}{-}{-}{-}{-}{-}{-}{-}{-}{-}{-}{-}{-}{-}{-}{-}{-}{-}{-}{-}{-}{-}{-}{-}{-}{-}{-}{-}{-}{-}{-}{-}{-}{-}{-}{-}{-}{-}{-}{-}{-}{-}{-}{-}{-}{-}{-}{-}{-}{-}{-}{-}{-}{-}{-}{-}{-}}
\AttributeTok{  }\FunctionTok{comments}\KeywordTok{:}
\AttributeTok{    }\FunctionTok{utterances}\KeywordTok{:}
\AttributeTok{      }\FunctionTok{repo}\KeywordTok{:}\AttributeTok{ achalmed/methodica}
\AttributeTok{      }\FunctionTok{label}\KeywordTok{:}\AttributeTok{ }\StringTok{"comments"}
\AttributeTok{      }\FunctionTok{theme}\KeywordTok{:}\AttributeTok{ preferred{-}color{-}scheme}
\AttributeTok{      }\FunctionTok{issue{-}term}\KeywordTok{:}\AttributeTok{ title}
\AttributeTok{  }
\CommentTok{  \# {-}{-}{-}{-}{-}{-}{-}{-}{-}{-}{-}{-}{-}{-}{-}{-}{-}{-}{-}{-}{-}{-}{-}{-}{-}{-}{-}{-}{-}{-}{-}{-}{-}{-}{-}{-}{-}{-}{-}{-}{-}{-}{-}{-}{-}{-}{-}{-}{-}{-}{-}{-}{-}{-}{-}{-}{-}{-}{-}{-}{-}{-}{-}{-}{-}{-}{-}{-}{-}{-}}
\CommentTok{  \# Navegación}
\CommentTok{  \# {-}{-}{-}{-}{-}{-}{-}{-}{-}{-}{-}{-}{-}{-}{-}{-}{-}{-}{-}{-}{-}{-}{-}{-}{-}{-}{-}{-}{-}{-}{-}{-}{-}{-}{-}{-}{-}{-}{-}{-}{-}{-}{-}{-}{-}{-}{-}{-}{-}{-}{-}{-}{-}{-}{-}{-}{-}{-}{-}{-}{-}{-}{-}{-}{-}{-}{-}{-}{-}{-}}
\AttributeTok{  }\FunctionTok{navbar}\KeywordTok{:}
\AttributeTok{    }\FunctionTok{left}\KeywordTok{:}
\AttributeTok{      }\KeywordTok{{-}}\AttributeTok{ }\FunctionTok{text}\KeywordTok{:}\AttributeTok{ }\StringTok{"Inicio"}
\AttributeTok{        }\FunctionTok{href}\KeywordTok{:}\AttributeTok{ index.qmd}
\AttributeTok{      }\KeywordTok{{-}}\AttributeTok{ }\FunctionTok{text}\KeywordTok{:}\AttributeTok{ }\StringTok{"Blog"}
\AttributeTok{        }\FunctionTok{href}\KeywordTok{:}\AttributeTok{ blog/}
\AttributeTok{        }
\AttributeTok{    }\FunctionTok{right}\KeywordTok{:}
\AttributeTok{      }\KeywordTok{{-}}\AttributeTok{ }\FunctionTok{text}\KeywordTok{:}\AttributeTok{ }\StringTok{"About"}
\AttributeTok{        }\FunctionTok{aria{-}label}\KeywordTok{:}\AttributeTok{ }\StringTok{"About Me"}
\AttributeTok{        }\FunctionTok{href}\KeywordTok{:}\AttributeTok{ https://achalmaedison.netlify.app/about/}
\AttributeTok{        }
\AttributeTok{      }\KeywordTok{{-}}\AttributeTok{ }\FunctionTok{text}\KeywordTok{:}\AttributeTok{ }\StringTok{"More"}
\AttributeTok{        }\FunctionTok{aria{-}label}\KeywordTok{:}\AttributeTok{ }\StringTok{"More Resources"}
\AttributeTok{        }\FunctionTok{icon}\KeywordTok{:}\AttributeTok{ ellipsis{-}h}
\AttributeTok{        }\FunctionTok{menu}\KeywordTok{:}
\AttributeTok{          }\KeywordTok{{-}}\AttributeTok{ }\FunctionTok{text}\KeywordTok{:}\AttributeTok{ }\StringTok{"Econometría"}
\AttributeTok{            }\FunctionTok{href}\KeywordTok{:}\AttributeTok{ https://epsilon{-}y{-}beta.netlify.app/}
\AttributeTok{          }\KeywordTok{{-}}\AttributeTok{ }\FunctionTok{text}\KeywordTok{:}\AttributeTok{ }\StringTok{"Filosofía"}
\AttributeTok{            }\FunctionTok{href}\KeywordTok{:}\AttributeTok{ https://dialectica{-}y{-}mercado.netlify.app/}
\AttributeTok{          }\KeywordTok{{-}}\AttributeTok{ }\FunctionTok{text}\KeywordTok{:}\AttributeTok{ }\StringTok{"Finanzas"}
\AttributeTok{            }\FunctionTok{href}\KeywordTok{:}\AttributeTok{ https://pecunia{-}fluxus.netlify.app/}
\AttributeTok{          }\KeywordTok{{-}}\AttributeTok{ }\FunctionTok{text}\KeywordTok{:}\AttributeTok{ }\StringTok{"Gestión empresarial"}
\AttributeTok{            }\FunctionTok{href}\KeywordTok{:}\AttributeTok{ https://actus{-}mercator.netlify.app/}
\AttributeTok{          }\KeywordTok{{-}}\AttributeTok{ }\FunctionTok{text}\KeywordTok{:}\AttributeTok{ }\StringTok{"Gestión pública"}
\AttributeTok{            }\FunctionTok{href}\KeywordTok{:}\AttributeTok{ https://res{-}publica.netlify.app/}
\AttributeTok{          }\KeywordTok{{-}}\AttributeTok{ }\FunctionTok{text}\KeywordTok{:}\AttributeTok{ }\StringTok{"Metodología de la investigación"}
\AttributeTok{            }\FunctionTok{href}\KeywordTok{:}\AttributeTok{ https://methodica.netlify.app/}
\AttributeTok{          }\KeywordTok{{-}}\AttributeTok{ }\FunctionTok{text}\KeywordTok{:}\AttributeTok{ }\StringTok{"Macroeconomía"}
\AttributeTok{            }\FunctionTok{href}\KeywordTok{:}\AttributeTok{ https://aequilibria.netlify.app/}
\AttributeTok{          }\KeywordTok{{-}}\AttributeTok{ }\FunctionTok{text}\KeywordTok{:}\AttributeTok{ }\StringTok{"Matemáticas"}
\AttributeTok{            }\FunctionTok{href}\KeywordTok{:}\AttributeTok{ https://axiomata.netlify.app/}
\AttributeTok{          }\KeywordTok{{-}}\AttributeTok{ }\FunctionTok{text}\KeywordTok{:}\AttributeTok{ }\StringTok{"Microeconomía"}
\AttributeTok{            }\FunctionTok{href}\KeywordTok{:}\AttributeTok{ https://optimums.netlify.app/}
\AttributeTok{          }\KeywordTok{{-}}\AttributeTok{ }\FunctionTok{text}\KeywordTok{:}\AttributeTok{ }\StringTok{"Programación y Software"}
\AttributeTok{            }\FunctionTok{href}\KeywordTok{:}\AttributeTok{ https://numerus{-}scriptum.netlify.app/}
\AttributeTok{          }\KeywordTok{{-}}\AttributeTok{ }\FunctionTok{text}\KeywordTok{:}\AttributeTok{ }\StringTok{"Seguridad informática"}
\AttributeTok{            }\FunctionTok{href}\KeywordTok{:}\AttributeTok{ https://chaska{-}x.netlify.app/}
\AttributeTok{            }
\AttributeTok{      }\KeywordTok{{-}}\AttributeTok{ }\FunctionTok{icon}\KeywordTok{:}\AttributeTok{ github}
\AttributeTok{        }\FunctionTok{href}\KeywordTok{:}\AttributeTok{ https://github.com/achalmed}
\AttributeTok{        }\FunctionTok{aria{-}label}\KeywordTok{:}\AttributeTok{ }\StringTok{"GitHub Profile"}
\AttributeTok{        }
\AttributeTok{      }\KeywordTok{{-}}\AttributeTok{ }\FunctionTok{icon}\KeywordTok{:}\AttributeTok{ twitter}
\AttributeTok{        }\FunctionTok{href}\KeywordTok{:}\AttributeTok{ https://x.com/achalmaedison}
\AttributeTok{        }\FunctionTok{aria{-}label}\KeywordTok{:}\AttributeTok{ }\StringTok{"Twitter/X Profile"}
\AttributeTok{        }
\AttributeTok{      }\KeywordTok{{-}}\AttributeTok{ }\FunctionTok{icon}\KeywordTok{:}\AttributeTok{ rss}
\AttributeTok{        }\FunctionTok{href}\KeywordTok{:}\AttributeTok{ index.xml}
\AttributeTok{        }\FunctionTok{aria{-}label}\KeywordTok{:}\AttributeTok{ }\StringTok{"RSS Feed"}
\AttributeTok{  }
\CommentTok{  \# {-}{-}{-}{-}{-}{-}{-}{-}{-}{-}{-}{-}{-}{-}{-}{-}{-}{-}{-}{-}{-}{-}{-}{-}{-}{-}{-}{-}{-}{-}{-}{-}{-}{-}{-}{-}{-}{-}{-}{-}{-}{-}{-}{-}{-}{-}{-}{-}{-}{-}{-}{-}{-}{-}{-}{-}{-}{-}{-}{-}{-}{-}{-}{-}{-}{-}{-}{-}{-}{-}}
\CommentTok{  \# Footer}
\CommentTok{  \# {-}{-}{-}{-}{-}{-}{-}{-}{-}{-}{-}{-}{-}{-}{-}{-}{-}{-}{-}{-}{-}{-}{-}{-}{-}{-}{-}{-}{-}{-}{-}{-}{-}{-}{-}{-}{-}{-}{-}{-}{-}{-}{-}{-}{-}{-}{-}{-}{-}{-}{-}{-}{-}{-}{-}{-}{-}{-}{-}{-}{-}{-}{-}{-}{-}{-}{-}{-}{-}{-}}
\AttributeTok{  }\FunctionTok{page{-}footer}\KeywordTok{:}
\FunctionTok{    left}\KeywordTok{: }\CharTok{\textgreater{}{-}}
\NormalTok{      © 2026 Edison Achalma · Made with [Quarto](https://quarto.org)}
      
\FunctionTok{    center}\KeywordTok{: }\CharTok{|}
\NormalTok{      \textless{}a class="link{-}dark me{-}1" href="/accessibility.html" title="Accessibility commitment" target="\_blank" rel="noopener"\textgreater{}\textless{}/a\textgreater{}}
\NormalTok{      \textless{}a class="link{-}dark me{-}1" href="https://fosstodon.org/@achalmaedison" title="Mastodon" target="\_blank" rel="noopener"\textgreater{}\textless{}/a\textgreater{}}
\NormalTok{      \textless{}a class="link{-}dark me{-}1" href="https://github.com/achalmed" title="GitHub" target="\_blank" rel="noopener"\textgreater{}\textless{}/a\textgreater{}}
\NormalTok{      \textless{}a class="link{-}dark me{-}1" href="https://linkedin.com/in/achalmaedison" title="LinkedIn" target="\_blank" rel="noopener"\textgreater{}\textless{}/a\textgreater{}}
      
\AttributeTok{    }\FunctionTok{right}\KeywordTok{:}
\AttributeTok{      }\KeywordTok{{-}}\AttributeTok{ }\FunctionTok{text}\KeywordTok{:}\AttributeTok{ }\StringTok{"Accessibility"}
\AttributeTok{        }\FunctionTok{aria{-}label}\KeywordTok{:}\AttributeTok{ }\StringTok{"Accessibility Commitment"}
\AttributeTok{        }\FunctionTok{href}\KeywordTok{:}\AttributeTok{ accessibility.html}
\AttributeTok{      }\KeywordTok{{-}}\AttributeTok{ }\FunctionTok{text}\KeywordTok{:}\AttributeTok{ }\StringTok{"Contact"}
\AttributeTok{        }\FunctionTok{aria{-}label}\KeywordTok{:}\AttributeTok{ }\StringTok{"Contact Form"}
\AttributeTok{        }\FunctionTok{href}\KeywordTok{:}\AttributeTok{ contact.html}
\AttributeTok{      }\KeywordTok{{-}}\AttributeTok{ }\FunctionTok{text}\KeywordTok{:}\AttributeTok{ }\StringTok{"Privacy"}
\AttributeTok{        }\FunctionTok{aria{-}label}\KeywordTok{:}\AttributeTok{ }\StringTok{"Privacy Policy"}
\AttributeTok{        }\FunctionTok{href}\KeywordTok{:}\AttributeTok{ privacy.html}
\AttributeTok{      }\KeywordTok{{-}}\AttributeTok{ }\FunctionTok{text}\KeywordTok{:}\AttributeTok{ }\StringTok{"License"}
\AttributeTok{        }\FunctionTok{aria{-}label}\KeywordTok{:}\AttributeTok{ }\StringTok{"License Details"}
\AttributeTok{        }\FunctionTok{href}\KeywordTok{:}\AttributeTok{ license.html}
\AttributeTok{      }\KeywordTok{{-}}\AttributeTok{ }\FunctionTok{icon}\KeywordTok{:}\AttributeTok{ rss}
\AttributeTok{        }\FunctionTok{href}\KeywordTok{:}\AttributeTok{ index.xml}
\AttributeTok{        }\FunctionTok{aria{-}label}\KeywordTok{:}\AttributeTok{ }\StringTok{"RSS Feed"}

\CommentTok{\# ========================================================================}
\CommentTok{\# FORMATOS DE SALIDA}
\CommentTok{\# ========================================================================}
\FunctionTok{format}\KeywordTok{:}
\AttributeTok{  }\FunctionTok{html}\KeywordTok{:}
\CommentTok{    \# Tema}
\AttributeTok{    }\FunctionTok{theme}\KeywordTok{:}
\AttributeTok{      }\FunctionTok{light}\KeywordTok{:}
\AttributeTok{        }\KeywordTok{{-}}\AttributeTok{ cosmo}
\AttributeTok{        }\KeywordTok{{-}}\AttributeTok{ assets/theme\_light.scss}
\AttributeTok{        }\KeywordTok{{-}}\AttributeTok{ assets/colors.scss}
\AttributeTok{        }\KeywordTok{{-}}\AttributeTok{ assets/fonts.scss}
\AttributeTok{      }\FunctionTok{dark}\KeywordTok{:}
\AttributeTok{        }\KeywordTok{{-}}\AttributeTok{ darkly}
\AttributeTok{        }\KeywordTok{{-}}\AttributeTok{ assets/theme\_dark.scss}
\AttributeTok{        }\KeywordTok{{-}}\AttributeTok{ assets/colors.scss}
\AttributeTok{        }\KeywordTok{{-}}\AttributeTok{ assets/fonts.scss}
\AttributeTok{        }
\CommentTok{    \# CSS Adicional}
\AttributeTok{    }\FunctionTok{css}\KeywordTok{:}\AttributeTok{ assets/styles.css}
\AttributeTok{    }
\CommentTok{    \# Includes (Google Tag Manager)}
\AttributeTok{    }\FunctionTok{include{-}in{-}header}\KeywordTok{:}\AttributeTok{ assets/gtm{-}head.html}
\AttributeTok{    }\FunctionTok{include{-}after{-}body}\KeywordTok{:}\AttributeTok{ assets/gtm{-}body.html}
\AttributeTok{    }
\CommentTok{    \# Navegación}
\AttributeTok{    }\FunctionTok{page{-}navigation}\KeywordTok{:}\AttributeTok{ }\CharTok{true}
\AttributeTok{    }\FunctionTok{smooth{-}scroll}\KeywordTok{:}\AttributeTok{ }\CharTok{true}
\AttributeTok{    }
\CommentTok{    \# Tabla de Contenidos}
\AttributeTok{    }\FunctionTok{toc}\KeywordTok{:}\AttributeTok{ }\CharTok{true}
\AttributeTok{    }\FunctionTok{toc{-}depth}\KeywordTok{:}\AttributeTok{ }\DecValTok{3}
\AttributeTok{    }\FunctionTok{toc{-}location}\KeywordTok{:}\AttributeTok{ right}
\AttributeTok{    }
\CommentTok{    \# Código}
\AttributeTok{    }\FunctionTok{code{-}fold}\KeywordTok{:}\AttributeTok{ }\CharTok{true}
\AttributeTok{    }\FunctionTok{code{-}copy}\KeywordTok{:}\AttributeTok{ }\CharTok{true}
\AttributeTok{    }\FunctionTok{code{-}line{-}numbers}\KeywordTok{:}\AttributeTok{ }\CharTok{true}
\end{Highlighting}
\end{Shaded}

\textbf{Cambios principales en la versión optimizada:} 1. Tema oscuro
añadido 2. Google Analytics y Cookie Consent activados 3.
\texttt{aria-label} en todos los íconos 4. Anuncio actualizado (no
temporal) 5. Enlaces del footer a rutas locales 6. \texttt{site} añadido
a Twitter Card 7. \texttt{page-navigation:\ true} para mejor UX 8.
Comentarios más descriptivos

\section{Casos de Uso Completos}\label{casos-de-uso-completos}

\subsection{Caso 1: Blog Académico
Personal}\label{caso-1-blog-acaduxe9mico-personal}

\textbf{Escenario:} Blog de investigación con posts, publicaciones,
about.

\begin{Shaded}
\begin{Highlighting}[]
\CommentTok{\# ========================================================================}
\CommentTok{\# BLOG ACADÉMICO PERSONAL}
\CommentTok{\# ========================================================================}
\FunctionTok{project}\KeywordTok{:}
\AttributeTok{  }\FunctionTok{type}\KeywordTok{:}\AttributeTok{ website}
\AttributeTok{  }\FunctionTok{output{-}dir}\KeywordTok{:}\AttributeTok{ \_site}

\FunctionTok{website}\KeywordTok{:}
\AttributeTok{  }\FunctionTok{title}\KeywordTok{:}\AttributeTok{ }\StringTok{"Dr. María López"}
\AttributeTok{  }\FunctionTok{description}\KeywordTok{:}\AttributeTok{ }\StringTok{"Investigación en neurociencia cognitiva"}
\AttributeTok{  }\FunctionTok{site{-}url}\KeywordTok{:}\AttributeTok{ https://marialopez.com}
\AttributeTok{  }\FunctionTok{favicon}\KeywordTok{:}\AttributeTok{ assets/favicon.png}
\AttributeTok{  }
\AttributeTok{  }\FunctionTok{navbar}\KeywordTok{:}
\AttributeTok{    }\FunctionTok{left}\KeywordTok{:}
\AttributeTok{      }\KeywordTok{{-}}\AttributeTok{ }\FunctionTok{text}\KeywordTok{:}\AttributeTok{ }\StringTok{"Inicio"}
\AttributeTok{        }\FunctionTok{href}\KeywordTok{:}\AttributeTok{ index.qmd}
\AttributeTok{      }\KeywordTok{{-}}\AttributeTok{ }\FunctionTok{text}\KeywordTok{:}\AttributeTok{ }\StringTok{"Blog"}
\AttributeTok{        }\FunctionTok{href}\KeywordTok{:}\AttributeTok{ blog/}
\AttributeTok{      }\KeywordTok{{-}}\AttributeTok{ }\FunctionTok{text}\KeywordTok{:}\AttributeTok{ }\StringTok{"Publicaciones"}
\AttributeTok{        }\FunctionTok{href}\KeywordTok{:}\AttributeTok{ publications.qmd}
\AttributeTok{      }\KeywordTok{{-}}\AttributeTok{ }\FunctionTok{text}\KeywordTok{:}\AttributeTok{ }\StringTok{"CV"}
\AttributeTok{        }\FunctionTok{href}\KeywordTok{:}\AttributeTok{ cv.qmd}
\AttributeTok{    }\FunctionTok{right}\KeywordTok{:}
\AttributeTok{      }\KeywordTok{{-}}\AttributeTok{ }\FunctionTok{icon}\KeywordTok{:}\AttributeTok{ github}
\AttributeTok{        }\FunctionTok{href}\KeywordTok{:}\AttributeTok{ https://github.com/mlopez}
\AttributeTok{        }\FunctionTok{aria{-}label}\KeywordTok{:}\AttributeTok{ }\StringTok{"GitHub"}
\AttributeTok{      }\KeywordTok{{-}}\AttributeTok{ }\FunctionTok{icon}\KeywordTok{:}\AttributeTok{ twitter}
\AttributeTok{        }\FunctionTok{href}\KeywordTok{:}\AttributeTok{ https://twitter.com/mlopez}
\AttributeTok{        }\FunctionTok{aria{-}label}\KeywordTok{:}\AttributeTok{ }\StringTok{"Twitter"}
\AttributeTok{  }
\AttributeTok{  }\FunctionTok{page{-}footer}\KeywordTok{:}
\AttributeTok{    }\FunctionTok{center}\KeywordTok{:}\AttributeTok{ }\StringTok{"© 2026 María López"}

\FunctionTok{format}\KeywordTok{:}
\AttributeTok{  }\FunctionTok{html}\KeywordTok{:}
\AttributeTok{    }\FunctionTok{theme}\KeywordTok{:}\AttributeTok{ minty}
\AttributeTok{    }\FunctionTok{toc}\KeywordTok{:}\AttributeTok{ }\CharTok{true}
\end{Highlighting}
\end{Shaded}

\subsection{Caso 2: Sitio de Documentación
Técnica}\label{caso-2-sitio-de-documentaciuxf3n-tuxe9cnica}

\textbf{Escenario:} Documentación de software con sidebar.

\begin{Shaded}
\begin{Highlighting}[]
\CommentTok{\# ========================================================================}
\CommentTok{\# DOCUMENTACIÓN TÉCNICA}
\CommentTok{\# ========================================================================}
\FunctionTok{project}\KeywordTok{:}
\AttributeTok{  }\FunctionTok{type}\KeywordTok{:}\AttributeTok{ website}
\AttributeTok{  }\FunctionTok{output{-}dir}\KeywordTok{:}\AttributeTok{ docs}

\FunctionTok{website}\KeywordTok{:}
\AttributeTok{  }\FunctionTok{title}\KeywordTok{:}\AttributeTok{ }\StringTok{"MiPaquete Docs"}
\AttributeTok{  }\FunctionTok{site{-}url}\KeywordTok{:}\AttributeTok{ https://docs.mipaquete.com}
\AttributeTok{  }\FunctionTok{repo{-}url}\KeywordTok{:}\AttributeTok{ https://github.com/org/mipaquete}
\AttributeTok{  }\FunctionTok{repo{-}actions}\KeywordTok{:}\AttributeTok{ }\KeywordTok{[}\AttributeTok{edit}\KeywordTok{,}\AttributeTok{ issue}\KeywordTok{]}
\AttributeTok{  }
\AttributeTok{  }\FunctionTok{sidebar}\KeywordTok{:}
\AttributeTok{    }\FunctionTok{style}\KeywordTok{:}\AttributeTok{ }\StringTok{"docked"}
\AttributeTok{    }\FunctionTok{background}\KeywordTok{:}\AttributeTok{ light}
\AttributeTok{    }\FunctionTok{contents}\KeywordTok{:}
\AttributeTok{      }\KeywordTok{{-}}\AttributeTok{ }\FunctionTok{section}\KeywordTok{:}\AttributeTok{ }\StringTok{"Getting Started"}
\AttributeTok{        }\FunctionTok{contents}\KeywordTok{:}
\AttributeTok{          }\KeywordTok{{-}}\AttributeTok{ docs/installation.qmd}
\AttributeTok{          }\KeywordTok{{-}}\AttributeTok{ docs/quickstart.qmd}
\AttributeTok{      }\KeywordTok{{-}}\AttributeTok{ }\FunctionTok{section}\KeywordTok{:}\AttributeTok{ }\StringTok{"User Guide"}
\AttributeTok{        }\FunctionTok{contents}\KeywordTok{:}
\AttributeTok{          }\KeywordTok{{-}}\AttributeTok{ docs/basics.qmd}
\AttributeTok{          }\KeywordTok{{-}}\AttributeTok{ docs/advanced.qmd}
\AttributeTok{      }\KeywordTok{{-}}\AttributeTok{ }\FunctionTok{section}\KeywordTok{:}\AttributeTok{ }\StringTok{"API Reference"}
\AttributeTok{        }\FunctionTok{contents}\KeywordTok{:}
\AttributeTok{          }\KeywordTok{{-}}\AttributeTok{ docs/api/index.qmd}
\AttributeTok{          }
\AttributeTok{  }\FunctionTok{navbar}\KeywordTok{:}
\AttributeTok{    }\FunctionTok{right}\KeywordTok{:}
\AttributeTok{      }\KeywordTok{{-}}\AttributeTok{ }\FunctionTok{icon}\KeywordTok{:}\AttributeTok{ github}
\AttributeTok{        }\FunctionTok{href}\KeywordTok{:}\AttributeTok{ https://github.com/org/mipaquete}
\AttributeTok{        }\FunctionTok{aria{-}label}\KeywordTok{:}\AttributeTok{ }\StringTok{"GitHub Repository"}
\AttributeTok{      }\KeywordTok{{-}}\AttributeTok{ }\FunctionTok{text}\KeywordTok{:}\AttributeTok{ }\StringTok{"Changelog"}
\AttributeTok{        }\FunctionTok{href}\KeywordTok{:}\AttributeTok{ changelog.qmd}

\FunctionTok{format}\KeywordTok{:}
\AttributeTok{  }\FunctionTok{html}\KeywordTok{:}
\AttributeTok{    }\FunctionTok{theme}\KeywordTok{:}\AttributeTok{ cosmo}
\AttributeTok{    }\FunctionTok{toc}\KeywordTok{:}\AttributeTok{ }\CharTok{true}
\AttributeTok{    }\FunctionTok{code{-}copy}\KeywordTok{:}\AttributeTok{ }\CharTok{true}
\AttributeTok{    }\FunctionTok{code{-}line{-}numbers}\KeywordTok{:}\AttributeTok{ }\CharTok{true}
\end{Highlighting}
\end{Shaded}

\subsection{Caso 3: Portfolio Personal}\label{caso-3-portfolio-personal}

\textbf{Escenario:} Portfolio de proyectos con galería.

\begin{Shaded}
\begin{Highlighting}[]
\CommentTok{\# ========================================================================}
\CommentTok{\# PORTFOLIO PERSONAL}
\CommentTok{\# ========================================================================}
\FunctionTok{project}\KeywordTok{:}
\AttributeTok{  }\FunctionTok{type}\KeywordTok{:}\AttributeTok{ website}
\AttributeTok{  }\FunctionTok{output{-}dir}\KeywordTok{:}\AttributeTok{ \_site}

\FunctionTok{website}\KeywordTok{:}
\AttributeTok{  }\FunctionTok{title}\KeywordTok{:}\AttributeTok{ }\StringTok{"Juan Pérez"}
\AttributeTok{  }\FunctionTok{description}\KeywordTok{:}\AttributeTok{ }\StringTok{"Data Scientist \& Visualización"}
\AttributeTok{  }\FunctionTok{site{-}url}\KeywordTok{:}\AttributeTok{ https://juanperez.com}
\AttributeTok{  }
\AttributeTok{  }\FunctionTok{navbar}\KeywordTok{:}
\AttributeTok{    }\FunctionTok{left}\KeywordTok{:}
\AttributeTok{      }\KeywordTok{{-}}\AttributeTok{ }\FunctionTok{text}\KeywordTok{:}\AttributeTok{ }\StringTok{"Proyectos"}
\AttributeTok{        }\FunctionTok{href}\KeywordTok{:}\AttributeTok{ projects/}
\AttributeTok{      }\KeywordTok{{-}}\AttributeTok{ }\FunctionTok{text}\KeywordTok{:}\AttributeTok{ }\StringTok{"Blog"}
\AttributeTok{        }\FunctionTok{href}\KeywordTok{:}\AttributeTok{ blog/}
\AttributeTok{      }\KeywordTok{{-}}\AttributeTok{ }\FunctionTok{text}\KeywordTok{:}\AttributeTok{ }\StringTok{"About"}
\AttributeTok{        }\FunctionTok{href}\KeywordTok{:}\AttributeTok{ about.qmd}
\AttributeTok{    }\FunctionTok{right}\KeywordTok{:}
\AttributeTok{      }\KeywordTok{{-}}\AttributeTok{ }\FunctionTok{icon}\KeywordTok{:}\AttributeTok{ linkedin}
\AttributeTok{        }\FunctionTok{href}\KeywordTok{:}\AttributeTok{ https://linkedin.com/in/jperez}
\AttributeTok{        }\FunctionTok{aria{-}label}\KeywordTok{:}\AttributeTok{ }\StringTok{"LinkedIn"}
\AttributeTok{      }\KeywordTok{{-}}\AttributeTok{ }\FunctionTok{icon}\KeywordTok{:}\AttributeTok{ github}
\AttributeTok{        }\FunctionTok{href}\KeywordTok{:}\AttributeTok{ https://github.com/jperez}
\AttributeTok{        }\FunctionTok{aria{-}label}\KeywordTok{:}\AttributeTok{ }\StringTok{"GitHub"}

\FunctionTok{format}\KeywordTok{:}
\AttributeTok{  }\FunctionTok{html}\KeywordTok{:}
\AttributeTok{    }\FunctionTok{theme}\KeywordTok{:}\AttributeTok{ lux}
\AttributeTok{    }\FunctionTok{page{-}layout}\KeywordTok{:}\AttributeTok{ full}
\end{Highlighting}
\end{Shaded}

\subsection{Caso 4: Sitio
Multilingüe}\label{caso-4-sitio-multilinguxfce}

\textbf{Escenario:} Sitio en español e inglés.

\begin{Shaded}
\begin{Highlighting}[]
\CommentTok{\# ========================================================================}
\CommentTok{\# SITIO MULTILINGÜE}
\CommentTok{\# ========================================================================}
\FunctionTok{project}\KeywordTok{:}
\AttributeTok{  }\FunctionTok{type}\KeywordTok{:}\AttributeTok{ website}
\AttributeTok{  }\FunctionTok{output{-}dir}\KeywordTok{:}\AttributeTok{ \_site}

\FunctionTok{website}\KeywordTok{:}
\AttributeTok{  }\FunctionTok{title}\KeywordTok{:}\AttributeTok{ }\StringTok{"Mi Sitio / My Site"}
\AttributeTok{  }
\AttributeTok{  }\FunctionTok{navbar}\KeywordTok{:}
\AttributeTok{    }\FunctionTok{left}\KeywordTok{:}
\AttributeTok{      }\KeywordTok{{-}}\AttributeTok{ }\FunctionTok{text}\KeywordTok{:}\AttributeTok{ }\StringTok{"Inicio / Home"}
\AttributeTok{        }\FunctionTok{href}\KeywordTok{:}\AttributeTok{ index.qmd}
\AttributeTok{      }\KeywordTok{{-}}\AttributeTok{ }\FunctionTok{text}\KeywordTok{:}\AttributeTok{ }\StringTok{"ES"}
\AttributeTok{        }\FunctionTok{menu}\KeywordTok{:}
\AttributeTok{          }\KeywordTok{{-}}\AttributeTok{ }\FunctionTok{text}\KeywordTok{:}\AttributeTok{ }\StringTok{"Blog (ES)"}
\AttributeTok{            }\FunctionTok{href}\KeywordTok{:}\AttributeTok{ es/blog/}
\AttributeTok{          }\KeywordTok{{-}}\AttributeTok{ }\FunctionTok{text}\KeywordTok{:}\AttributeTok{ }\StringTok{"About (ES)"}
\AttributeTok{            }\FunctionTok{href}\KeywordTok{:}\AttributeTok{ es/about.qmd}
\AttributeTok{      }\KeywordTok{{-}}\AttributeTok{ }\FunctionTok{text}\KeywordTok{:}\AttributeTok{ }\StringTok{"EN"}
\AttributeTok{        }\FunctionTok{menu}\KeywordTok{:}
\AttributeTok{          }\KeywordTok{{-}}\AttributeTok{ }\FunctionTok{text}\KeywordTok{:}\AttributeTok{ }\StringTok{"Blog (EN)"}
\AttributeTok{            }\FunctionTok{href}\KeywordTok{:}\AttributeTok{ en/blog/}
\AttributeTok{          }\KeywordTok{{-}}\AttributeTok{ }\FunctionTok{text}\KeywordTok{:}\AttributeTok{ }\StringTok{"About (EN)"}
\AttributeTok{            }\FunctionTok{href}\KeywordTok{:}\AttributeTok{ en/about.qmd}

\FunctionTok{format}\KeywordTok{:}
\AttributeTok{  }\FunctionTok{html}\KeywordTok{:}
\AttributeTok{    }\FunctionTok{theme}\KeywordTok{:}\AttributeTok{ flatly}
\end{Highlighting}
\end{Shaded}

\subsection{Caso 5: Blog de Empresa}\label{caso-5-blog-de-empresa}

\textbf{Escenario:} Blog corporativo con branding.

\begin{Shaded}
\begin{Highlighting}[]
\CommentTok{\# ========================================================================}
\CommentTok{\# BLOG CORPORATIVO}
\CommentTok{\# ========================================================================}
\FunctionTok{project}\KeywordTok{:}
\AttributeTok{  }\FunctionTok{type}\KeywordTok{:}\AttributeTok{ website}
\AttributeTok{  }\FunctionTok{output{-}dir}\KeywordTok{:}\AttributeTok{ \_site}

\FunctionTok{website}\KeywordTok{:}
\AttributeTok{  }\FunctionTok{title}\KeywordTok{:}\AttributeTok{ }\StringTok{"TechCorp Blog"}
\AttributeTok{  }\FunctionTok{description}\KeywordTok{:}\AttributeTok{ }\StringTok{"Insights y noticias de TechCorp"}
\AttributeTok{  }\FunctionTok{site{-}url}\KeywordTok{:}\AttributeTok{ https://blog.techcorp.com}
\AttributeTok{  }
\AttributeTok{  }\FunctionTok{navbar}\KeywordTok{:}
\AttributeTok{    }\FunctionTok{logo}\KeywordTok{:}\AttributeTok{ assets/logo.png}
\AttributeTok{    }\FunctionTok{background}\KeywordTok{:}\AttributeTok{ }\StringTok{"\#1a1a1a"}
\AttributeTok{    }\FunctionTok{foreground}\KeywordTok{:}\AttributeTok{ light}
\AttributeTok{    }\FunctionTok{left}\KeywordTok{:}
\AttributeTok{      }\KeywordTok{{-}}\AttributeTok{ }\FunctionTok{text}\KeywordTok{:}\AttributeTok{ }\StringTok{"Blog"}
\AttributeTok{        }\FunctionTok{href}\KeywordTok{:}\AttributeTok{ blog/}
\AttributeTok{      }\KeywordTok{{-}}\AttributeTok{ }\FunctionTok{text}\KeywordTok{:}\AttributeTok{ }\StringTok{"Productos"}
\AttributeTok{        }\FunctionTok{href}\KeywordTok{:}\AttributeTok{ https://techcorp.com/products}
\AttributeTok{      }\KeywordTok{{-}}\AttributeTok{ }\FunctionTok{text}\KeywordTok{:}\AttributeTok{ }\StringTok{"Contacto"}
\AttributeTok{        }\FunctionTok{href}\KeywordTok{:}\AttributeTok{ https://techcorp.com/contact}
\AttributeTok{    }\FunctionTok{right}\KeywordTok{:}
\AttributeTok{      }\KeywordTok{{-}}\AttributeTok{ }\FunctionTok{text}\KeywordTok{:}\AttributeTok{ }\StringTok{"Website Principal"}
\AttributeTok{        }\FunctionTok{href}\KeywordTok{:}\AttributeTok{ https://techcorp.com}
\AttributeTok{  }
\AttributeTok{  }\FunctionTok{page{-}footer}\KeywordTok{:}
\AttributeTok{    }\FunctionTok{left}\KeywordTok{:}\AttributeTok{ }\StringTok{"© 2026 TechCorp. Todos los derechos reservados."}
\FunctionTok{    center}\KeywordTok{: }\CharTok{|}
\NormalTok{      \textless{}a href="/privacy"\textgreater{}Privacy\textless{}/a\textgreater{} | }
\NormalTok{      \textless{}a href="/terms"\textgreater{}Terms\textless{}/a\textgreater{}}

\FunctionTok{format}\KeywordTok{:}
\AttributeTok{  }\FunctionTok{html}\KeywordTok{:}
\AttributeTok{    }\FunctionTok{theme}\KeywordTok{:}
\AttributeTok{      }\KeywordTok{{-}}\AttributeTok{ assets/corporate{-}theme.scss}
\AttributeTok{    }\FunctionTok{css}\KeywordTok{:}\AttributeTok{ assets/branding.css}
\end{Highlighting}
\end{Shaded}

\section{Optimización y Best
Practices}\label{optimizaciuxf3n-y-best-practices}

\subsection{1. Estructura y
Organización}\label{estructura-y-organizaciuxf3n}

\textbf{Bien organizado:}

\begin{Shaded}
\begin{Highlighting}[]
\CommentTok{\# ========================================================================}
\CommentTok{\# SECCIÓN CLARA CON COMENTARIO}
\CommentTok{\# ========================================================================}
\FunctionTok{website}\KeywordTok{:}
\CommentTok{  \# {-}{-}{-}{-}{-}{-}{-}{-}{-}{-}{-}{-}{-}{-}{-}{-}{-}{-}{-}{-}{-}{-}{-}{-}{-}{-}{-}{-}{-}{-}{-}{-}{-}{-}{-}{-}{-}{-}{-}{-}{-}{-}{-}{-}{-}{-}{-}{-}{-}{-}{-}{-}{-}{-}{-}{-}{-}{-}{-}{-}{-}{-}{-}{-}{-}{-}{-}{-}{-}{-}}
\CommentTok{  \# Subsección}
\CommentTok{  \# {-}{-}{-}{-}{-}{-}{-}{-}{-}{-}{-}{-}{-}{-}{-}{-}{-}{-}{-}{-}{-}{-}{-}{-}{-}{-}{-}{-}{-}{-}{-}{-}{-}{-}{-}{-}{-}{-}{-}{-}{-}{-}{-}{-}{-}{-}{-}{-}{-}{-}{-}{-}{-}{-}{-}{-}{-}{-}{-}{-}{-}{-}{-}{-}{-}{-}{-}{-}{-}{-}}
\AttributeTok{  }\FunctionTok{title}\KeywordTok{:}\AttributeTok{ }\StringTok{"..."}
\end{Highlighting}
\end{Shaded}

\textbf{Mal organizado:}

\begin{Shaded}
\begin{Highlighting}[]
\FunctionTok{website}\KeywordTok{:}
\FunctionTok{title}\KeywordTok{:}\AttributeTok{ }\StringTok{"..."}
\FunctionTok{navbar}\KeywordTok{:}
\FunctionTok{left}\KeywordTok{:}
\KeywordTok{{-}}\AttributeTok{ }\FunctionTok{text}\KeywordTok{:}\AttributeTok{ }\StringTok{"..."}
\end{Highlighting}
\end{Shaded}

\subsection{2. Comentarios Útiles}\label{comentarios-uxfatiles}

\textbf{Buenos comentarios:}

\begin{Shaded}
\begin{Highlighting}[]
\CommentTok{\# Deshabilitado temporalmente (problemas con CORS)}
\CommentTok{\# google{-}analytics: "G{-}XXXXXX"}

\FunctionTok{site{-}url}\KeywordTok{:}\AttributeTok{ https://misitio.com}\CommentTok{    \# Requerido para Open Graph}
\end{Highlighting}
\end{Shaded}

\textbf{Malos comentarios:}

\begin{Shaded}
\begin{Highlighting}[]
\FunctionTok{title}\KeywordTok{:}\AttributeTok{ }\StringTok{"Mi Sitio"}\CommentTok{    \# Esto es el título}
\end{Highlighting}
\end{Shaded}

\subsection{3. Validación YAML}\label{validaciuxf3n-yaml}

\textbf{Errores comunes:}

\begin{Shaded}
\begin{Highlighting}[]
\CommentTok{\# Incorrecto: falta quote en string con \textquotesingle{}:\textquotesingle{}}
\FunctionTok{title}\KeywordTok{:}\AttributeTok{ Mi Sitio: Un Blog}

\CommentTok{\# Correcto:}
\FunctionTok{title}\KeywordTok{:}\AttributeTok{ }\StringTok{"Mi Sitio: Un Blog"}

\CommentTok{\# Incorrecto: indentación incorrecta}
\FunctionTok{navbar}\KeywordTok{:}
\FunctionTok{left}\KeywordTok{:}
\KeywordTok{{-}}\AttributeTok{ }\FunctionTok{text}\KeywordTok{:}\AttributeTok{ }\StringTok{"Blog"}
\AttributeTok{  }\FunctionTok{href}\KeywordTok{:}\AttributeTok{ blog/}

\CommentTok{\# Correcto:}
\FunctionTok{navbar}\KeywordTok{:}
\AttributeTok{  }\FunctionTok{left}\KeywordTok{:}
\AttributeTok{    }\KeywordTok{{-}}\AttributeTok{ }\FunctionTok{text}\KeywordTok{:}\AttributeTok{ }\StringTok{"Blog"}
\AttributeTok{      }\FunctionTok{href}\KeywordTok{:}\AttributeTok{ blog/}
\end{Highlighting}
\end{Shaded}

\textbf{Validar YAML:}

\begin{Shaded}
\begin{Highlighting}[]
\CommentTok{\# Usar yamllint}
\ExtensionTok{yamllint}\NormalTok{ \_quarto.yml}

\CommentTok{\# O validador online}
\CommentTok{\# https://www.yamllint.com/}
\end{Highlighting}
\end{Shaded}

\subsection{4. Rutas Consistentes}\label{rutas-consistentes}

\textbf{Usar rutas relativas desde la raíz:}

\begin{Shaded}
\begin{Highlighting}[]
\CommentTok{\# Correcto:}
\FunctionTok{image}\KeywordTok{:}\AttributeTok{ /assets/img/logo.png}

\CommentTok{\# Evitar rutas absolutas del sistema:}
\CommentTok{\# Incorrecto:}
\CommentTok{\# image: /home/usuario/proyecto/assets/img/logo.png}
\end{Highlighting}
\end{Shaded}

\subsection{5. Seguridad y Privacidad}\label{seguridad-y-privacidad}

\textbf{Configuración recomendada para Europa (RGPD):}

\begin{Shaded}
\begin{Highlighting}[]
\FunctionTok{website}\KeywordTok{:}
\AttributeTok{  }\FunctionTok{google{-}analytics}\KeywordTok{:}
\AttributeTok{    }\FunctionTok{tracking{-}id}\KeywordTok{:}\AttributeTok{ }\StringTok{"G{-}XXXXXX"}
\AttributeTok{    }\FunctionTok{anonymize{-}ip}\KeywordTok{:}\AttributeTok{ }\CharTok{true}\CommentTok{           \# Anonimizar IPs}
\AttributeTok{    }
\AttributeTok{  }\FunctionTok{cookie{-}consent}\KeywordTok{:}
\AttributeTok{    }\FunctionTok{type}\KeywordTok{:}\AttributeTok{ express}\CommentTok{                \# Bloquear hasta consentimiento}
\AttributeTok{    }\FunctionTok{style}\KeywordTok{:}\AttributeTok{ headline}
\end{Highlighting}
\end{Shaded}

\subsection{6. Accesibilidad}\label{accesibilidad}

\textbf{Siempre incluir \texttt{aria-label} en íconos:}

\begin{Shaded}
\begin{Highlighting}[]
\FunctionTok{navbar}\KeywordTok{:}
\AttributeTok{  }\FunctionTok{right}\KeywordTok{:}
\AttributeTok{    }\KeywordTok{{-}}\AttributeTok{ }\FunctionTok{icon}\KeywordTok{:}\AttributeTok{ github}
\AttributeTok{      }\FunctionTok{href}\KeywordTok{:}\AttributeTok{ https://github.com/usuario}
\AttributeTok{      }\FunctionTok{aria{-}label}\KeywordTok{:}\AttributeTok{ }\StringTok{"GitHub Profile"}\CommentTok{    \# Importante para screen readers}
\end{Highlighting}
\end{Shaded}

\subsection{7. SEO Básico}\label{seo-buxe1sico}

\textbf{Mínimo requerido:}

\begin{Shaded}
\begin{Highlighting}[]
\FunctionTok{website}\KeywordTok{:}
\AttributeTok{  }\FunctionTok{title}\KeywordTok{:}\AttributeTok{ }\StringTok{"..."}\CommentTok{                   \# Máx. 60 caracteres}
\AttributeTok{  }\FunctionTok{description}\KeywordTok{:}\AttributeTok{ }\StringTok{"..."}\CommentTok{             \# 150{-}160 caracteres}
\AttributeTok{  }\FunctionTok{site{-}url}\KeywordTok{:}\AttributeTok{ https://...}\CommentTok{          \# URL completa}
\AttributeTok{  }\FunctionTok{favicon}\KeywordTok{:}\AttributeTok{ assets/favicon.png}
\AttributeTok{  }
\AttributeTok{  }\FunctionTok{open{-}graph}\KeywordTok{:}
\AttributeTok{    }\FunctionTok{image}\KeywordTok{:}\AttributeTok{ /assets/og{-}image.jpg}\CommentTok{  \# 1200x630 px}
\end{Highlighting}
\end{Shaded}

\subsection{8. Versionado con Git}\label{versionado-con-git}

\begin{Shaded}
\begin{Highlighting}[]
\CommentTok{\# Commit cada cambio significativo en \_quarto.yml}
\FunctionTok{git}\NormalTok{ add \_quarto.yml}
\FunctionTok{git}\NormalTok{ commit }\AttributeTok{{-}m} \StringTok{"feat(config): Add dark theme support"}

\CommentTok{\# Antes de cambios grandes, crear backup}
\FunctionTok{cp}\NormalTok{ \_quarto.yml \_quarto.yml.backup}
\end{Highlighting}
\end{Shaded}

\subsection{9. Testing}\label{testing}

\begin{Shaded}
\begin{Highlighting}[]
\CommentTok{\# Siempre probar después de cambios}
\ExtensionTok{quarto}\NormalTok{ preview}

\CommentTok{\# Verificar que no hay errores}
\ExtensionTok{quarto}\NormalTok{ render}

\CommentTok{\# Verificar configuración efectiva}
\ExtensionTok{quarto}\NormalTok{ inspect index.qmd}
\end{Highlighting}
\end{Shaded}

\subsection{10. Documentación}\label{documentaciuxf3n}

\textbf{Mantener un README.md con notas:}

\begin{Shaded}
\begin{Highlighting}[]
\FunctionTok{\# Configuración del Sitio}

\FunctionTok{\# Estructura de Archivos}
\SpecialStringTok{{-} }\InformationTok{\textasciigrave{}\_quarto.yml\textasciigrave{}}\NormalTok{: Configuración principal}
\SpecialStringTok{{-} }\InformationTok{\textasciigrave{}assets/\textasciigrave{}}\NormalTok{: Imágenes, CSS, JS}
\SpecialStringTok{{-} }\InformationTok{\textasciigrave{}blog/\textasciigrave{}}\NormalTok{: Posts del blog}

\FunctionTok{\# Temas Personalizados}
\SpecialStringTok{{-} }\InformationTok{\textasciigrave{}theme\_light.scss\textasciigrave{}}\NormalTok{: Tema claro}
\SpecialStringTok{{-} }\InformationTok{\textasciigrave{}theme\_dark.scss\textasciigrave{}}\NormalTok{: Tema oscuro}

\FunctionTok{\# Deployment}
\SpecialStringTok{{-} }\NormalTok{Netlify: https://methodica.netlify.app}
\SpecialStringTok{{-} }\NormalTok{Dominio: (pendiente)}
\end{Highlighting}
\end{Shaded}

\section{Troubleshooting}\label{troubleshooting}

\subsection{Problema 1: Navbar No
Aparece}\label{problema-1-navbar-no-aparece}

\textbf{Síntomas:}

\begin{itemize}
\tightlist
\item
  No hay barra de navegación
\item
  Solo aparece el título
\end{itemize}

\textbf{Causas y soluciones:}

\begin{Shaded}
\begin{Highlighting}[]
\CommentTok{\# Causa 1: Falta la sección navbar}
\CommentTok{\# Solución: Agregar navbar}
\FunctionTok{website}\KeywordTok{:}
\AttributeTok{  }\FunctionTok{navbar}\KeywordTok{:}
\AttributeTok{    }\FunctionTok{left}\KeywordTok{:}
\AttributeTok{      }\KeywordTok{{-}}\AttributeTok{ }\FunctionTok{text}\KeywordTok{:}\AttributeTok{ }\StringTok{"Inicio"}
\AttributeTok{        }\FunctionTok{href}\KeywordTok{:}\AttributeTok{ index.qmd}

\CommentTok{\# Causa 2: Sintaxis incorrecta}
\CommentTok{\# Incorrecto:}
\FunctionTok{navbar}\KeywordTok{:}
\FunctionTok{left}\KeywordTok{:}
\AttributeTok{  }\KeywordTok{{-}}\AttributeTok{ }\FunctionTok{text}\KeywordTok{:}\AttributeTok{ }\StringTok{"Blog"}
\AttributeTok{  }
\CommentTok{\# Correcto:}
\FunctionTok{navbar}\KeywordTok{:}
\AttributeTok{  }\FunctionTok{left}\KeywordTok{:}
\AttributeTok{    }\KeywordTok{{-}}\AttributeTok{ }\FunctionTok{text}\KeywordTok{:}\AttributeTok{ }\StringTok{"Blog"}
\AttributeTok{      }\FunctionTok{href}\KeywordTok{:}\AttributeTok{ blog/}
\end{Highlighting}
\end{Shaded}

\subsection{Problema 2: Imágenes Open Graph No Se
Muestran}\label{problema-2-imuxe1genes-open-graph-no-se-muestran}

\textbf{Síntomas:}

\begin{itemize}
\tightlist
\item
  Al compartir en redes sociales, no aparece imagen de vista previa
\end{itemize}

\textbf{Checklist:}

\begin{Shaded}
\begin{Highlighting}[]
\CommentTok{\# 1. Verificar que site{-}url está definido}
\FunctionTok{website}\KeywordTok{:}
\AttributeTok{  }\FunctionTok{site{-}url}\KeywordTok{:}\AttributeTok{ https://misitio.com}\CommentTok{    \# Obligatorio}

\CommentTok{\# 2. Usar ruta absoluta desde raíz}
\FunctionTok{website}\KeywordTok{:}
\AttributeTok{  }\FunctionTok{open{-}graph}\KeywordTok{:}
\AttributeTok{    }\FunctionTok{image}\KeywordTok{:}\AttributeTok{ /assets/img/og{-}image.jpg}\CommentTok{  \# Empieza con /}

\CommentTok{\# 3. Verificar dimensiones de imagen}
\CommentTok{\# {-} Tamaño: 1200x630 px}
\CommentTok{\# {-} Ratio: 1.91:1}
\CommentTok{\# {-} Formato: JPG o PNG}
\CommentTok{\# {-} Máximo: 8 MB}

\CommentTok{\# 4. Verificar que el archivo existe}
\AttributeTok{ls assets/img/og{-}image.jpg}
\end{Highlighting}
\end{Shaded}

\textbf{Forzar actualización de caché en redes sociales:}

\begin{itemize}
\tightlist
\item
  Facebook: https://developers.facebook.com/tools/debug/
\item
  Twitter: https://cards-dev.twitter.com/validator
\item
  LinkedIn: https://www.linkedin.com/post-inspector/
\end{itemize}

\subsection{Problema 3: Tema Personalizado No Se
Aplica}\label{problema-3-tema-personalizado-no-se-aplica}

\textbf{Síntomas:}

\begin{itemize}
\tightlist
\item
  CSS personalizado no tiene efecto
\item
  Tema se ve como el predeterminado
\end{itemize}

\textbf{Soluciones:}

\begin{Shaded}
\begin{Highlighting}[]
\CommentTok{\# Verificar rutas de archivos}
\FunctionTok{format}\KeywordTok{:}
\AttributeTok{  }\FunctionTok{html}\KeywordTok{:}
\AttributeTok{    }\FunctionTok{theme}\KeywordTok{:}
\AttributeTok{      }\KeywordTok{{-}}\AttributeTok{ cosmo}
\AttributeTok{      }\KeywordTok{{-}}\AttributeTok{ assets/custom.scss}\CommentTok{    \# Verificar que existe}
\AttributeTok{    }\FunctionTok{css}\KeywordTok{:}\AttributeTok{ assets/styles.css}\CommentTok{    \# Verificar que existe}

\CommentTok{\# Verificar que los archivos existen}
\AttributeTok{ls assets/custom.scss}
\AttributeTok{ls assets/styles.css}

\CommentTok{\# Limpiar caché y re{-}renderizar}
\AttributeTok{rm {-}rf .quarto}
\AttributeTok{quarto render}
\end{Highlighting}
\end{Shaded}

\subsection{Problema 4: Google Analytics No
Funciona}\label{problema-4-google-analytics-no-funciona}

\textbf{Síntomas:}

\begin{itemize}
\tightlist
\item
  Google Analytics no registra visitas
\end{itemize}

\textbf{Checklist:}

\begin{Shaded}
\begin{Highlighting}[]
\CommentTok{\# 1. Verificar ID de tracking}
\FunctionTok{website}\KeywordTok{:}
\AttributeTok{  }\FunctionTok{google{-}analytics}\KeywordTok{:}\AttributeTok{ }\StringTok{"G{-}XXXXXXXXXX"}\CommentTok{  \# Verificar formato}
\CommentTok{  \# GA4: empieza con G{-}}
\CommentTok{  \# Universal Analytics (antiguo): empieza con UA{-}}

\CommentTok{\# 2. Esperar 24{-}48 horas}
\CommentTok{\# Los datos pueden tardar en aparecer}

\CommentTok{\# 3. Verificar con herramientas de desarrollo}
\CommentTok{\# {-} Abrir DevTools (F12)}
\CommentTok{\# {-} Network tab}
\CommentTok{\# {-} Buscar "analytics" o "gtag"}
\CommentTok{\# {-} Debe haber requests a google{-}analytics.com}
\end{Highlighting}
\end{Shaded}

\subsection{Problema 5: Cookie Consent No
Aparece}\label{problema-5-cookie-consent-no-aparece}

\textbf{Síntomas:}

\begin{itemize}
\tightlist
\item
  No se muestra el banner de cookies
\end{itemize}

\textbf{Soluciones:}

\begin{Shaded}
\begin{Highlighting}[]
\CommentTok{\# 1. Verificar configuración}
\FunctionTok{website}\KeywordTok{:}
\AttributeTok{  }\FunctionTok{cookie{-}consent}\KeywordTok{:}\AttributeTok{ }\CharTok{true}\CommentTok{    \# O configuración completa}
\AttributeTok{  }
\CommentTok{\# 2. Limpiar localStorage del navegador}
\CommentTok{\# {-} DevTools \textgreater{} Application \textgreater{} Local Storage}
\CommentTok{\# {-} Eliminar todas las entradas}
\CommentTok{\# {-} Recargar página}

\CommentTok{\# 3. Verificar que no hay conflictos con otros scripts}
\end{Highlighting}
\end{Shaded}

\subsection{Problema 6: Comentarios (Utterances) No
Aparecen}\label{problema-6-comentarios-utterances-no-aparecen}

\textbf{Síntomas:}

\begin{itemize}
\tightlist
\item
  No se muestra la sección de comentarios
\end{itemize}

\textbf{Checklist:}

\begin{Shaded}
\begin{Highlighting}[]
\CommentTok{\# 1. Verificar configuración}
\FunctionTok{website}\KeywordTok{:}
\AttributeTok{  }\FunctionTok{comments}\KeywordTok{:}
\AttributeTok{    }\FunctionTok{utterances}\KeywordTok{:}
\AttributeTok{      }\FunctionTok{repo}\KeywordTok{:}\AttributeTok{ usuario/repositorio}\CommentTok{    \# Formato correcto}
\AttributeTok{      }\FunctionTok{issue{-}term}\KeywordTok{:}\AttributeTok{ title}
\AttributeTok{      }\FunctionTok{theme}\KeywordTok{:}\AttributeTok{ github{-}light}

\CommentTok{\# 2. Verificar que el repositorio existe}
\CommentTok{\# https://github.com/usuario/repositorio}

\CommentTok{\# 3. Verificar que Utterances está instalado}
\CommentTok{\# https://github.com/apps/utterances}
\CommentTok{\# Debe estar instalado en el repositorio}

\CommentTok{\# 4. Verificar permisos}
\CommentTok{\# El repositorio debe ser público o Utterances debe tener acceso}
\end{Highlighting}
\end{Shaded}

\subsection{Problema 7: Error ``Cannot Find
Module''}\label{problema-7-error-cannot-find-module}

\textbf{Síntomas:}

\begin{verbatim}
Error: Cannot find module 'assets/custom.scss'
\end{verbatim}

\textbf{Solución:}

\begin{Shaded}
\begin{Highlighting}[]
\CommentTok{\# Verificar que el archivo existe}
\FunctionTok{ls}\NormalTok{ assets/custom.scss}

\CommentTok{\# Si no existe, crearlo o eliminar la referencia}
\CommentTok{\# Si existe, verificar la ruta en \_quarto.yml}

\CommentTok{\# Ruta correcta (relativa a \_quarto.yml):}
\ExtensionTok{theme:}
  \ExtensionTok{{-}}\NormalTok{ cosmo}
  \ExtensionTok{{-}}\NormalTok{ assets/custom.scss    }\CommentTok{\# Correcto}

\CommentTok{\# Ruta incorrecta:}
\CommentTok{\# {-} /assets/custom.scss   \# NO usar / al inicio en theme}
\end{Highlighting}
\end{Shaded}

\subsection{Problema 8: Sitio No Se Actualiza en
Netlify}\label{problema-8-sitio-no-se-actualiza-en-netlify}

\textbf{Síntomas:}

\begin{itemize}
\tightlist
\item
  Cambios locales funcionan
\item
  Netlify muestra versión antigua
\end{itemize}

\textbf{Soluciones:}

\begin{Shaded}
\begin{Highlighting}[]
\CommentTok{\# 1. Verificar que cambios están en Git}
\FunctionTok{git}\NormalTok{ status}
\FunctionTok{git}\NormalTok{ add \_quarto.yml}
\FunctionTok{git}\NormalTok{ commit }\AttributeTok{{-}m} \StringTok{"Update config"}
\FunctionTok{git}\NormalTok{ push}

\CommentTok{\# 2. Verificar build en Netlify}
\CommentTok{\# {-} Ir a Netlify Dashboard}
\CommentTok{\# {-} Site \textgreater{} Deploys}
\CommentTok{\# {-} Ver logs del último deploy}
\CommentTok{\# {-} Buscar errores}

\CommentTok{\# 3. Limpiar caché de Netlify}
\CommentTok{\# {-} Netlify Dashboard}
\CommentTok{\# {-} Site settings \textgreater{} Build \& deploy}
\CommentTok{\# {-} Clear cache and retry deploy}

\CommentTok{\# 4. Verificar configuración de build}
\CommentTok{\# netlify.toml o configuración en UI:}
\CommentTok{\# Build command: quarto render}
\CommentTok{\# Publish directory: \_site}
\end{Highlighting}
\end{Shaded}

\section{Comandos Útiles}\label{comandos-uxfatiles}

\subsection{Comandos Básicos}\label{comandos-buxe1sicos}

\begin{Shaded}
\begin{Highlighting}[]
\CommentTok{\# Previsualizar sitio localmente}
\ExtensionTok{quarto}\NormalTok{ preview}

\CommentTok{\# Renderizar sitio completo}
\ExtensionTok{quarto}\NormalTok{ render}

\CommentTok{\# Renderizar solo archivo específico}
\ExtensionTok{quarto}\NormalTok{ render index.qmd}

\CommentTok{\# Renderizar con ejecución de código}
\ExtensionTok{quarto}\NormalTok{ render }\AttributeTok{{-}{-}execute}
\end{Highlighting}
\end{Shaded}

\subsection{Comandos de Inspección}\label{comandos-de-inspecciuxf3n}

\begin{Shaded}
\begin{Highlighting}[]
\CommentTok{\# Ver configuración efectiva de un archivo}
\ExtensionTok{quarto}\NormalTok{ inspect index.qmd}

\CommentTok{\# Ver metadata del proyecto}
\ExtensionTok{quarto}\NormalTok{ inspect}

\CommentTok{\# Verificar versión de Quarto}
\ExtensionTok{quarto} \AttributeTok{{-}{-}version}
\end{Highlighting}
\end{Shaded}

\subsection{Comandos de Limpieza}\label{comandos-de-limpieza}

\begin{Shaded}
\begin{Highlighting}[]
\CommentTok{\# Limpiar caché}
\FunctionTok{rm} \AttributeTok{{-}rf}\NormalTok{ .quarto}

\CommentTok{\# Limpiar output}
\FunctionTok{rm} \AttributeTok{{-}rf}\NormalTok{ \_site}

\CommentTok{\# Limpiar todo y re{-}renderizar}
\FunctionTok{rm} \AttributeTok{{-}rf}\NormalTok{ .quarto \_site }\KeywordTok{\&\&} \ExtensionTok{quarto}\NormalTok{ render}
\end{Highlighting}
\end{Shaded}

\subsection{Comandos de Publicación}\label{comandos-de-publicaciuxf3n}

\begin{Shaded}
\begin{Highlighting}[]
\CommentTok{\# Publicar en Quarto Pub}
\ExtensionTok{quarto}\NormalTok{ publish quarto{-}pub}

\CommentTok{\# Publicar en GitHub Pages}
\ExtensionTok{quarto}\NormalTok{ publish gh{-}pages}

\CommentTok{\# Publicar en Netlify}
\ExtensionTok{quarto}\NormalTok{ publish netlify}

\CommentTok{\# Publicar en RStudio Connect}
\ExtensionTok{quarto}\NormalTok{ publish connect}
\end{Highlighting}
\end{Shaded}

\subsection{Comandos de Desarrollo}\label{comandos-de-desarrollo}

\begin{Shaded}
\begin{Highlighting}[]
\CommentTok{\# Modo preview con auto{-}reload}
\ExtensionTok{quarto}\NormalTok{ preview }\AttributeTok{{-}{-}watch}

\CommentTok{\# Renderizar solo archivos modificados}
\ExtensionTok{quarto}\NormalTok{ render }\AttributeTok{{-}{-}render{-}modified}

\CommentTok{\# Renderizar con perfil específico}
\ExtensionTok{quarto}\NormalTok{ render }\AttributeTok{{-}{-}profile}\NormalTok{ production}
\end{Highlighting}
\end{Shaded}

\section{Conclusión}\label{conclusiuxf3n}

\subsection{\texorpdfstring{Checklist Mínimo para
\texttt{\_quarto.yml}}{Checklist Mínimo para \_quarto.yml}}\label{checklist-muxednimo-para-_quarto.yml}

\begin{Shaded}
\begin{Highlighting}[]
\CommentTok{\# Obligatorio}
\FunctionTok{project}\KeywordTok{:}
\AttributeTok{  }\FunctionTok{type}\KeywordTok{:}\AttributeTok{ website}
\AttributeTok{  }
\FunctionTok{website}\KeywordTok{:}
\AttributeTok{  }\FunctionTok{title}\KeywordTok{:}\AttributeTok{ }\StringTok{"..."}
\AttributeTok{  }
\FunctionTok{format}\KeywordTok{:}
\AttributeTok{  }\FunctionTok{html}\KeywordTok{:}
\AttributeTok{    }\FunctionTok{theme}\KeywordTok{:}\AttributeTok{ cosmo}

\CommentTok{\# Altamente recomendado}
\FunctionTok{website}\KeywordTok{:}
\AttributeTok{  }\FunctionTok{description}\KeywordTok{:}\AttributeTok{ }\StringTok{"..."}
\AttributeTok{  }\FunctionTok{site{-}url}\KeywordTok{:}\AttributeTok{ https://...}
\AttributeTok{  }\FunctionTok{favicon}\KeywordTok{:}\AttributeTok{ assets/favicon.png}
\AttributeTok{  }\FunctionTok{navbar}\KeywordTok{:}\AttributeTok{ }\KeywordTok{\{\}}
\AttributeTok{  }
\CommentTok{\# Recomendado para SEO}
\FunctionTok{website}\KeywordTok{:}
\AttributeTok{  }\FunctionTok{open{-}graph}\KeywordTok{:}\AttributeTok{ }\KeywordTok{\{\}}
\AttributeTok{  }\FunctionTok{twitter{-}card}\KeywordTok{:}\AttributeTok{ }\KeywordTok{\{\}}
\AttributeTok{  }
\CommentTok{\# Opcional pero útil}
\FunctionTok{website}\KeywordTok{:}
\AttributeTok{  }\FunctionTok{google{-}analytics}\KeywordTok{:}\AttributeTok{ }\StringTok{"..."}
\AttributeTok{  }\FunctionTok{cookie{-}consent}\KeywordTok{:}\AttributeTok{ }\KeywordTok{\{\}}
\AttributeTok{  }\FunctionTok{comments}\KeywordTok{:}\AttributeTok{ }\KeywordTok{\{\}}
\AttributeTok{  }\FunctionTok{page{-}footer}\KeywordTok{:}\AttributeTok{ }\KeywordTok{\{\}}
\end{Highlighting}
\end{Shaded}

\subsection{Template Rápido
(Copiar/Pegar)}\label{template-ruxe1pido-copiarpegar}

\begin{Shaded}
\begin{Highlighting}[]
\CommentTok{\# ========================================================================}
\CommentTok{\# TEMPLATE RÁPIDO \_quarto.yml}
\CommentTok{\# ========================================================================}

\FunctionTok{project}\KeywordTok{:}
\AttributeTok{  }\FunctionTok{type}\KeywordTok{:}\AttributeTok{ website}
\AttributeTok{  }\FunctionTok{output{-}dir}\KeywordTok{:}\AttributeTok{ \_site}

\FunctionTok{website}\KeywordTok{:}
\AttributeTok{  }\FunctionTok{title}\KeywordTok{:}\AttributeTok{ }\StringTok{"TU TÍTULO"}
\AttributeTok{  }\FunctionTok{description}\KeywordTok{:}\AttributeTok{ }\StringTok{"Tu descripción del sitio"}
\AttributeTok{  }\FunctionTok{site{-}url}\KeywordTok{:}\AttributeTok{ https://tusitio.com}
\AttributeTok{  }\FunctionTok{favicon}\KeywordTok{:}\AttributeTok{ assets/favicon.png}
\AttributeTok{  }
\AttributeTok{  }\FunctionTok{navbar}\KeywordTok{:}
\AttributeTok{    }\FunctionTok{left}\KeywordTok{:}
\AttributeTok{      }\KeywordTok{{-}}\AttributeTok{ }\FunctionTok{text}\KeywordTok{:}\AttributeTok{ }\StringTok{"Inicio"}
\AttributeTok{        }\FunctionTok{href}\KeywordTok{:}\AttributeTok{ index.qmd}
\AttributeTok{      }\KeywordTok{{-}}\AttributeTok{ }\FunctionTok{text}\KeywordTok{:}\AttributeTok{ }\StringTok{"Blog"}
\AttributeTok{        }\FunctionTok{href}\KeywordTok{:}\AttributeTok{ blog/}
\AttributeTok{    }\FunctionTok{right}\KeywordTok{:}
\AttributeTok{      }\KeywordTok{{-}}\AttributeTok{ }\FunctionTok{icon}\KeywordTok{:}\AttributeTok{ github}
\AttributeTok{        }\FunctionTok{href}\KeywordTok{:}\AttributeTok{ https://github.com/usuario}
\AttributeTok{        }\FunctionTok{aria{-}label}\KeywordTok{:}\AttributeTok{ }\StringTok{"GitHub"}
\AttributeTok{  }
\AttributeTok{  }\FunctionTok{page{-}footer}\KeywordTok{:}
\AttributeTok{    }\FunctionTok{center}\KeywordTok{:}\AttributeTok{ }\StringTok{"© 2026 Tu Nombre"}

\FunctionTok{format}\KeywordTok{:}
\AttributeTok{  }\FunctionTok{html}\KeywordTok{:}
\AttributeTok{    }\FunctionTok{theme}\KeywordTok{:}\AttributeTok{ cosmo}
\AttributeTok{    }\FunctionTok{toc}\KeywordTok{:}\AttributeTok{ }\CharTok{true}
\AttributeTok{    }\FunctionTok{code{-}fold}\KeywordTok{:}\AttributeTok{ }\CharTok{true}
\AttributeTok{    }\FunctionTok{code{-}copy}\KeywordTok{:}\AttributeTok{ }\CharTok{true}
\end{Highlighting}
\end{Shaded}

\subsection{Recursos Adicionales}\label{recursos-adicionales}

\textbf{Documentación oficial:}

\begin{itemize}
\tightlist
\item
  Quarto Websites: https://quarto.org/docs/websites/
\item
  Quarto Reference: https://quarto.org/docs/reference/
\end{itemize}

\textbf{Herramientas:}

\begin{itemize}
\tightlist
\item
  YAML Validator: https://www.yamllint.com/
\item
  Open Graph Debugger: https://developers.facebook.com/tools/debug/
\item
  Twitter Card Validator: https://cards-dev.twitter.com/validator
\end{itemize}

\section{Publicaciones Similares}\label{publicaciones-similares}

Si te interesó este artículo, te recomendamos que explores otros blogs y
recursos relacionados que pueden ampliar tus conocimientos. Aquí te dejo
algunas sugerencias:

\begin{enumerate}
\def\labelenumi{\arabic{enumi}.}
\tightlist
\item
  \href{https://methodica.netlify.app/posts/2023-06-03-ideas-de-investigacion-para-economia/index.pdf}{\faIcon{file-pdf}}
  \href{https://methodica.netlify.app/posts/2023-06-03-ideas-de-investigacion-para-economia}{Ideas
  De Investigacion Para Economia}
\item
  \href{https://methodica.netlify.app/posts/2023-06-03-pautas-de-presentacion-del-informe-de-investigacion/index.pdf}{\faIcon{file-pdf}}
  \href{https://methodica.netlify.app/posts/2023-06-03-pautas-de-presentacion-del-informe-de-investigacion}{Pautas
  De Presentacion Del Informe De Investigacion}
\item
  \href{https://methodica.netlify.app/posts/2025-01-12-recursos-de-bibliografia-y-documentacion/index.pdf}{\faIcon{file-pdf}}
  \href{https://methodica.netlify.app/posts/2025-01-12-recursos-de-bibliografia-y-documentacion}{Recursos
  De Bibliografia Y Documentacion}
\item
  \href{https://methodica.netlify.app/posts/2025-02-09-recursos-para-traducción-y-correccion/index.pdf}{\faIcon{file-pdf}}
  \href{https://methodica.netlify.app/posts/2025-02-09-recursos-para-traducción-y-correccion}{Recursos
  Para Traducción Y Correccion}
\item
  \href{https://methodica.netlify.app/posts/2025-04-23-tipos-de-elementos-en-zotero/index.pdf}{\faIcon{file-pdf}}
  \href{https://methodica.netlify.app/posts/2025-04-23-tipos-de-elementos-en-zotero}{Tipos
  De Elementos En Zotero}
\end{enumerate}

Esperamos que encuentres estas publicaciones igualmente interesantes y
útiles. ¡Disfruta de la lectura!






\end{document}
